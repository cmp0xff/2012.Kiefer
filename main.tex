\documentclass[a4paper]{article}


\input{./preambles/preamble}
\input{./preambles/unicode}
\setdefaultlanguage{english}
\setotherlanguages{german,greek}

\input{./preambles/math-single}
\input{./preambles/math-brac}
\input{./preambles/math-thm}
\input{./preambles/phys-chem}

%\newcommand{\RomaN}[1]{%
%  \textup{\uppercase\expandafter{\romannumeral#1}}%
%}


\usepackage[style=authoryear-icomp,
			backend=biber]{biblatex}
\addbibresource{main.bib}
\title{Notes on \cite{Kiefer_2012}}
\author{Yi-Fan Wang}


\begin{document}
\maketitle

%\begin{abstract}
%\cite
%\end{abstract}

\tableofcontents

\section{Parametrised and relational systems}

\subsection{Particle systems}

\cite{Gitman_1990,Prokhorov_2009,Rothe_2010} can also be helpful.

\subsubsection{Parametrised non-relativistic particle}

The advantage of the complication below is to clarify the steps at which the 
constraints are not to be imposed.

\paragraph{Un-parametrised system}

The action reads
\begin{equation}
\sfun{S}{q} \coloneqq \int_{t_1}^{t_2}\dd t\,\rfun{L}{q, q'},
\label{eq:action-ori}
\end{equation}
where
\begin{equation}
q' \coloneqq \frde{q}{t}.
\end{equation}
It can be rephrased (see below) in the velocity formalism 
\cite[ch.~2]{Gitman_1990}
\begin{equation}
\sfun{S}{q, p, v} \coloneqq \int_{t_1}^{t_2}\dd t\,\cbr{p\rbr{q'-v}+
\rfun{L^\text{v}}{q, v}},
\label{eq:action-v}
\end{equation}
in which
\begin{equation}
\rfun{L^\text{v}}{q, v} \coloneqq \rfun{L}{q, q'}.
\end{equation}
Variation of \cref{eq:action-v} with respect to $\rbr{q, p, v}$ gives
\begin{equation}
\rfun{\frpa{L^\text{v}}{q}}{q, v} = p',\quad
\rfun{\frpa{L^\text{v}}{v}}{q, v} = p,\quad
q' = v,
\label{eq:ext-euler-lag}
\end{equation}
respectively. Eliminating $\rbr{p, v}$ in \cref{eq:ext-euler-lag} recovers the 
Euler--Lagrange equation.

To move to the Hamiltonian formalism, transform the action in 
\cref{eq:action-v}
\begin{align}
\sfun{S}{q, p, v} &\equiv \int_{t_1}^{t_2}\dd t\,\cbr{pq' - 
\rbr{pv-\rfun{L^\text{v}}{q, v}}} \nonumber \\
&\eqqcolon \int_{t_1}^{t_2}\dd t\,\cbr{pq' - \rfun{H^\text{v}}{q, p, v}}.
\end{align}
One may solve $v$ by $\rbr{q, p}$ from the second equation in 
\cref{eq:ext-euler-lag} by partially inverse the function 
$\rfun{\frpa{L^\text{v}}{v}}{q, v}$, and denote the solution by
\begin{equation}
v = \rfun{\bar v}{q, p}.
\label{eq:v-bar-v}
\end{equation}
In the current case, all velocity can be solved, and one passes to the 
\emph{canonical Hamiltonian} in one leap
\begin{equation}
\rfun{H^\text{c}}{q, p} \coloneqq %\fat{H^\text{v}}{v = \rfun{\bar v}{q, p}}
%\equiv
\rfun{H^\text{v}}{q, p, \rfun{\bar v}{q, p}} \equiv
p\rfun{\bar v}{q, p} - \rfun{L^\text{v}}{q, \rfun{\bar v}{q, p}}.
\label{eq:H^c}
\end{equation}

\paragraph{Parametrised system}

Parametrising $t$ in \cref{eq:action-ori} gives
\begin{equation}
\sfun{S}{q, t} = \int_{t_1}^{t_2} \dd \tau\,\rfun{\tilde L}{q, t, \dot 
q, \dot t},
\end{equation}
where
\begin{equation}
\rfun{\tilde L}{q, t, \dot q, \dot t} \coloneqq \dot t \rfun{L}{q, 
\frac{\dot q}{\dot t}},\quad
\dot \rfun{f}{q,t} \coloneqq \rfun{\frde{f}{\tau}}{q,t}.
\end{equation}
The corresponding velocity formalism reads
\begin{align}
\sfun{S}{q, t, p_q, p_t, u, N} &\coloneqq \int_{t_1}^{t_2}\dd \tau\,
\cbr{p_q\rbr{\dot q-u}+p_t\rbr{\dot t-N} +
\rfun{\tilde L^\text{v}}{q, t, u, N}} \nonumber \\
&=\int_{t_1}^{t_2}\dd \tau\,
\cbr{p_q\rbr{\dot q-u}+p_t\rbr{\dot t-N} +
N \rfun{L^\text{v}}{q, \frac{u}{N}}}.
\label{eq:action-Nu}
\end{align}

Let us try to move to the Hamiltonian formalism. Variation of 
\cref{eq:action-Nu} with respect to $u$ gives
\begin{equation}
p_q = \frpa{\tilde L^\text{v}}{u} \equiv
\rfun{\frpa{L^\text{v}}{v}}{q, \frac{u}{N}},
\label{eq:p_q-u-N}
\end{equation}
whereas the variation with respect to $N$ leads to
\begin{equation}
p_t = \frpa{\tilde L^\text{v}}{N} \equiv
\rfun{L^\text{v}}{q, \frac{u}{N}} - \frac{u}{N}\cdot
\rfun{\frpa{L^\text{v}}{v}}{q,\frac{u}{N}}.
\label{eq:p_t-u-N}
\end{equation}
Can one in this case solve both $\rbr{u, N}$ by $\rbr{q, t, p_q, p_t}$ from 
\cref{eq:p_t-u-N,eq:p_q-u-N}? The answer is negative. With the help of 
\cref{eq:v-bar-v}, one finds
\begin{equation}
\frac{u}{N} = \rfun{\bar v}{q, p_q} \quad \Leftrightarrow \quad
u = \rfun{\bar u}{q, p_q, N} \coloneqq N \rfun{\bar v}{q, p_q}.
\label{eq:u/N}
\end{equation}
from \cref{eq:p_q-u-N} (compare it with the second equation in 
\cref{eq:ext-euler-lag}!). Inserting \cref{eq:v-bar-v,eq:u/N} into
\cref{eq:p_t-u-N} yields
\begin{equation}
p_t = \rfun{L^\text{v}}{q, \rfun{\bar v}{q, p_q}} -
\rfun{\bar v}{q, p_q}\cdot p_q
\equiv -\rfun{H^\text{c}}{q, p_q},
\label{eq:p_t-H^c}
\end{equation}
where used is made of \cref{eq:H^c} in the last step. One sees that 
\cref{eq:p_t-H^c} contains no more velocity, and $N$ cannot be solved by the 
positions and momenta.

Inserting the \emph{partially solved set of velocities} (and not 
\cref{eq:p_t-H^c}, which is incompetent to eliminate the velocity), the action
\cref{eq:action-Nu} reads
\begin{align}
\sfun{S}{q, t, p_q, p_t, N} &\coloneqq \sfun{S}{q, t, p_q, p_t, \bar 
u, N} \nonumber \\
&= \int_{t_1}^{t_2}\dd \tau\,\cbr{p_q \dot q + p_t \dot t - \sbr{ p_q \bar u + 
p_t N - \rfun{\tilde L^\text{v}}{q, t, \bar u, N}}} \nonumber \\
& \eqqcolon \int_{t_1}^{t_2}\dd \tau\,
\cbr{p_q \dot q + p_t\dot t - \rfun{\tilde H^\text{p}}{q, t, p_q, p_t, N}},
\end{align}
in which
\begin{equation}
\tilde H^\text{p} = N \rfun{\tilde H_\perp}{q, t, p_q, p_t},\qquad
\tilde H_\perp \coloneqq p_t + \rfun{H^\text{c}}{q, p_q}
\end{equation}
are the \emph{Hamiltonian with primary constraints} and \emph{Hamiltonian 
constraint}, respectively. The canonical Hamiltonian in this case vanishes, 
since there are only constraints in $\tilde H^\text{p}$ and nothing else left. 
\Cref{eq:p_t-H^c} is a \emph{(primary) constraint} and is to be solved along 
with the Hamiltonian equations of motion, which can be found in 
\cite[ch.~2]{Gitman_1990}.






% Let's print the overall heading of the bibliography first:
%\printbibheading
\printbibliography

\end{document}