% begin 15.11.2016
\section{Foundations of Cosmology}
\label{sec:found-cosmo}

\subsection{A brief history of cosmology}

Cosmology is the science of origin and evolution of the Universe as a whole.

cosmogenesis: part of cosmology, dealing with origin and evolution of celestial 
bodies.

κόσμος, kósmos: universe; order decoration, embellishment (cosmetics!)

Begin of `rational' cosmology with ancient philosophers:

\begin{itemize}
\item Anaximander (Ἀναξίμανδρος Anaxímandros; \textborn c.\ 610, \textdied c.\ 
545 BC)
\item Empedokles (Ἐμπεδοκλῆς Empedoklḗs; \textborn c.\ 485, \textdied c.\ 425 
BC)
\item Aristoteles (Ἀριστοτέλης Aristotélēs; \textborn 384, 
\textdied 322 BC)
\item Galileo Galilei (\textborn 1564, \textdied 1642)
\item Johannes Kepler (\textborn 1571, \textdied 1630)
\item Isaac Newton (\textborn 1642, \textdied 1727)
\item Immanuel Kant (\textborn 1724, \textdied 1804)
\item Harlow Shapley (\textborn 1885, \textdied 1972)
\item Vesto Slipher (\textborn 1875, \textdied 1969)
\item Albert Einstein (\textborn 1879, \textdied 1955)
\item Willem de Sitter (\textborn 1872, \textdied 1934)
\item Edwin Hubble (\textborn 1889, \textdied 1953)
\item Carl Wilhelm Wirtz (\textborn 1876, \textdied 1939)
\item Alexander Friedmann (\textborn 1888, \textdied 1925)
\item Georges Edouard Lemaître (\textborn 1894, \textdied 1966)
\end{itemize}

\begin{itemize}
\item Maarten Schmidt (\textborn 1929) \cite{SCHMIDT1963}
\item Alpher, Bethe, Gamov (1948)
\item Penzias and Wilson (1965) 
\cite{Dicke1965,Penzias1965,Penzias1965a}
\item COBE satellite (1989)
\item Further satellite missions
\item Supernova observations
\item Important theoretical developments
\end{itemize}
% end 15.11.2016


\subsection{Structures in the Universe and the Cosmological Principle}
% begin 17.11.2016
We observe \emph{structures} in the Universe. Appropriate scale: 
\emph{Megaparsec} (\si{\mega\parsec}),
\begin{equation}
\SI{1}{\mega\parsec} = \SI{e6}{\parsec} \approx \SI{3.09e24}{\centi\metre}.
\end{equation}

\paragraph{Up to $\sim\SI{100}{\mega\parsec}$} galaxies 
($\sim\SI{30}{\kilo\parsec}$), clusters of galaxies 
($\sim\SI{5}{\mega\parsec}$), super-clusters ($\sim\SI{50}{\mega\parsec}$) and 
voids ($\sim$ \SIrange{100}{200}{\mega\parsec})

\textbf{diagram} (Milky way) (mainly from red-shift surveys)

dispute about structures beyond

$\gtrsim\SI{100}{\mega\parsec}$: mainly repetition of these structures

\paragraph{observable Universe} $\sim$ \emph{Hubble length}
\begin{equation}
\lc \hH_0^{-1} \approx 3\hh^{-1}\,\si{\giga\parsec},
\end{equation}
where
\begin{equation}
\hH_0 \eqqcolon 100\hh\,\si{\kilo\metre\per\second\per\mega\parsec}
\end{equation}

\cite{Ade2014}:
\begin{equation}
\hH_0 = \SI{67.4 +- 1.4}{\kilo\metre\per\second\per\mega\parsec},
\end{equation}
so that $\hh\approx \num{.67}$, and
\begin{equation}
\lc \hH_0^{-1} \approx \SI{4.4}{\giga\parsec}.
\end{equation}

$M_\text{Milky Way} \sim \num{1.4e11}\,M_\astrosun$; we observe about 
$\num{e11}$ galaxies.

Milky Way belongs to the cluster `Local group' and to the super-cluster `Virgo' 
whose centre in about \SI{20}{\mega\parsec} distance.

Volume filled by galaxies $\lesssim \num{e-6}$ total volume, so that treat 
galaxies in cosmological models as `points'.

There is also \emph{Dark Matter}.

\textbf{diagram}
\cite{1933AcHPh...6..110Z}

If matter were only in stars: $v_\text{rad} \propto r^{-1/2}$ outside visible 
part,
\begin{equation}
v = \sqrt{\frac{\nG \rfun{M}{r}}{r}}
\end{equation}
where $\rfun{M}{r}$ mass from centre to $r$. For large $r$, $\rfun{M}{r} 
\propto r$ fits curves. See \cite{GottIII2005} for a cartography of our 
universe.

\paragraph{CMB} highly isotropic (up to $\rfun{\Omicron}{\num{e-5}}$); Universe 
also looks roughly the same in every direction.


\paragraph{Cosmological Principle} dates back to 
\cite{einstein1917kosmologische}; in his paper 
\citetitle{einstein1930kosmologischen} he wrote:
\begin{quote}
Alle Stellen des Universums sind gleichwertig; im speziellen soll also auch die 
\"ortlich gemittelte Dichte der Sternmaterie \"uberall gleich sein.
\end{quote}
(``All places in the Universe are alike'') (name \emph{cosmological principle} 
due to Edward Milne 1933 \textbf{cannot find citation})
% end 17.11.2016
% begin 18.11.2016

Harrison compares it to Rudyard Kipling's ``The Cat That Walked by Himself'': 
``I am the Cat who walks by himself, and all places are alike to me.''

In \citeyear{einstein1930kosmologischen}, Einstein mentions the following 
second principle of his \citeyear{einstein1917kosmologische} paper, which he 
now gives up in view of Hubble's observation:
\begin{quote}
R\"aumliche Struktur und Dichte sollen zeitlich konstant sein.
\end{quote}
(Later called ``perfect cosmological principle'' and revived in the theory of 
the steady-state universe \cite{Hoyle1948,Bondi1948})

Somewhat different from the cosmological principle is the \emph{Copernican 
principle} (Bondi 1960 \textbf{cannot find citation}) (Copernicus: Sunis at 
the centre, but we are not there),
\begin{quote}
We are not at the centre of the Universe.
\end{quote}
where ``We'' refers to ``us'' and seems to perpetuate the belief that a centre 
exists (but we are not there), so that better to talk about the cosmological 
principle only!

Important consequences:

\emph{Isotropy around our position (CMB, etc.).} Combined with cosmological 
principle one has isotropy about every point; according to a theorem 
\cite{Schur1886} (see also \cref{ssec:constant-curvature}), this 
leads also to homogeneity, or spaces of constant curvature; see 
\cref{ssec:constant-curvature}.

Only with assumed homogeneity can one draw conclusions such as the following:

\textbf{diagram}

Statement ``our Galaxy is at $t = t_2$ similarly structured than the galaxy 
(of the same class) that we see today as it was at $t=t_2$'' can be drawn only 
if homogeneity holds.

\subsection{Spaces of constant curvature}
\label{ssec:constant-curvature}
\paragraph{Isotropy around one point}
At the origin of a Riemann Normal Coordinate System, the Riemann tensor must be 
a tensor invariant under arbitrary rotations

(we talk about space here; in space-time, this would refer to Lorentz 
transformations, which would only be relevant for the perfect cosmological 
principle)

\begin{equation}
R_{iklm} = a\delta_{ik}\delta_{lm} + b\delta_{il}\delta_{km} + c\delta_{im} 
\delta_{kl} + d\epsilon_{iklm},
\end{equation}
where $\epsilon_{iklm}$ is a pseudo-tensor, so we set $d=0$.

Antisymmetry properties of $R_{iklm}$ entail
\begin{equation}
a = 0, \qquad b = -c,
\end{equation}
so that
\begin{equation}
R_{iklm} = b\rbr{\delta_{il}\delta_{km} - \delta_{im}\delta_{kl}}
\end{equation}
at origin of the Riemann Normal Coordinate System. In an arbitrary coordinate 
system,
\begin{equation}
R_{iklm} = \rfun{b}{x}\rbr{g_{il}g_{km} - g_{im}g_{kl}}.
\end{equation}
But if we had $\rfun{b}{x}$, then the gradient $\rightarrow{\nabla}\rfun{b}{x}$ 
would specify a \emph{distinguished direction}, in contradiction to the 
cosmological principle\footnote{Result also follows from $G^i{}_{k;i}=0$.}, so 
that
\begin{equation}
\rfun{b}{x} = b = \text{const.},
\end{equation}
leading to a \emph{space of constant curvature} (\cite{Schur1886}: 
isotropy everywhere leads to homogeneity).

Line element?
section 5.9: Two metrics are conformal to each other if
\begin{equation}
\rfun{\ol{g}_{ab}}{x} = \rfun{\Omega^2}{x}\rfun{g_{ab}}{x}.
\end{equation}
If this is the case, the Weyl tensor $C^\cdot{}_{\cdot\cdot\cdot}$ coincide; 
conformally flat means $\rfun{g_{ab}}{x} = \rfun{\Omega^{-2}}{x}\delta_{ab}$, 
where the Weyl tensor for the metric $\delta_{ab}$ vanishes, so that 
$C^\cdot{}_{\cdot\cdot\cdot} = 0$ for $\rfun{g_{ab}}{x}$ as well.

\paragraph{Special case of $n=3$}: $C^\cdot{}_{\cdot\cdot\cdot} \equiv 0$; 
necessary and sufficient for conformal flatness is the vanishing of the 
\emph{Cotton tensor} (Emile Cotton (\textborn 1982, \textdied 1950) 
[\textbf{unrecognisable}]
\begin{equation}
C_{ijk} = R_{ij;k} - R_{ik;j} + \frac{1}{4}\rbr{g_{ik}R_{,j} - g_{ij}R_{,k}}.
\end{equation}
Use $\rfun{R_{iklm}}{x} = b\rbr{g_{il}g_{km} - g_{im}g_{kl}}$, $b = 
\text{const.}$ [space of a constant curvature] to show that in this case 
$C^\cdot{}_{\cdot\cdot\cdot} = 0$ ($n=3$: $C_{\cdot\cdot\cdot} = 0$).

So that \emph{every space of constant curvature is conformally flat}.

Invert $\rfun{g_{ab}}{x} = \rfun{\Omega^{-2}}{x}\delta_{ab}$ into 
$\rfun{R_{inlm}}{x} = \rfun{b}{\ldots}$, where general expression for 
$R_{inlm}$ is also inserted, one can derive
\begin{empheq}[box=\fbox]{equation}
\dd l^2 = \Omega^{-2} \delta_{ab}\,\dd x^a\,\dd x^b = 
\frac{\delta_{ab}\,\dd x^a\,\dd x^b}{\rbr{1+\frac{b}{4}r^2}^2},\quad
r^2 \coloneqq \sum_{a=1}^d \rbr{x^a}^2,
\end{empheq}
where $b = 0$ corresponds to flat space.

We see that two spaces of constant curvature (same $b$) and with the same 
dimension are \emph{locally isometric}.

$n=3$: make explicit coordinate transformation
\begin{equation}
r \mapsto \ol{r},\qquad k\ol{r}^2 \coloneqq \frac{b r^2}{1+\frac{b}{4} r^2},
\end{equation}
where $k = \sgn b$. Then,
\begin{empheq}[box=\fbox]{equation}
\dd l^2 = a^2\rbr{\frac{\dd r^2}{1-kr^2}+r^2\,\dd\Omega^2},\qquad
a^2\coloneqq \frac{k}{b} = \frac{6k}{R},
\end{empheq}
where $\ol{r}$ is called $r$ again, $\dd \Omega^2 = \dd \theta^2 + \sin^2\theta
\,\dd \phi^2$, and $R$ Ricci scalar. This is the form usually used in cosmology.
Space-time: $a \to \rfun{a}{t}$ (see later), where $a$ is called the 
\emph{scale factor}.


\paragraph{Geometric realisation of spaces with constant curvature $b \neq 0$
\newline{}}

1. $b > 0$: $n$-dimensional space can be embedded in an \emph{Euclidean space} 
of dimension $\rbr{n+1}$:
\begin{equation}
\BbbR^{n+1} \supset S_\rho \coloneqq \cbr{\ol{x}\in\BbbR^{n+1} 
\Bigg| \sum_{i=1}^{n+1} \rbr{\ol{x}^i}^2 = \rho^2},\qquad b = \rho^{-2},
\end{equation}
where $S_\rho \equiv S_\rho^n$ is $n$-dimensional surface of 
$\rbr{n+1}$-dimensional ball, or sphere with radius $\rho$.

Metric on $S_\rho$ is the one induced by the Euclidean metric of $\BbbR^{n+1}$.
\begin{proof}
stereographic projection

\textbf{diagramme}

\begin{equation}
\tan \phi = \frac{\rho}{x^\alpha} = \frac{\overline{x}^{n+1}}{x^\alpha - 
\overline{x}^\alpha} \hookrightarrow \overline{x}^\alpha = 
x^\alpha \rbr{1-\frac{\overline{x}^{n+1}}{\rho}}.
\label{eq:starred}
\end{equation}

$S_\rho$:
\begin{equation}
\sum_{i=1}^{n+1}\rbr{}^2 = \rho^2 = \sum_{\alpha = 1}^n\rbr{\ol{x}^\alpha}^2
+\rbr{\ol{x}^{n+1}}^2,
\end{equation}
where by \cref{eq:starred},
\begin{align}
\sum_{\alpha = 1}^n\rbr{\ol{x}^\alpha}^2 &=
\rbr{1-\frac{\ol{x}^{n+1}}{\rho}}^2 \sum_{\alpha=1}^n \rbr{x^\alpha}^2
\eqqcolon \rbr{1-\frac{\ol{x}^{n+1}}{\rho}}^2 r^2 \nonumber \\
&= \rho^2 - \rbr{\ol{x}^{n+1}}^2 = \rho^2 \rbr{1 -\rbr{\frac{\ol{x}^{n+1} 
}{\rho}}^2} \nonumber \\
&= \rho^2\rbr{1+\frac{\ol{x}^{n+1}}{\rho}} 
\rbr{1-\frac{\ol{x}^{n+1}}{\rho}}.
\end{align}
Assume $\ol{x}^{n+1} \neq \rho$, or the north pole $N$ excluded. Divide by
$\rbr{1-\ol{x}^{n+1}/\rho}$:
\begin{equation}
\rbr{1-\frac{\ol{x}^{n+1}}{\rho}}r^2 = \rho^2 \rbr{1+\frac{\ol{x}^{n+1}}{\rho}},
\end{equation}
so that
\begin{equation}
\ol{x}^{n+1} = \rho\frac{r^2-\rho^2}{r^2+\rho^2}.
\end{equation}
Insert this into \cref{eq:starred}
\begin{equation}
\ol{x}^\alpha = x^\alpha\rbr{1-\frac{\ol{x}^{n+1}}{\rho}} =
\frac{2\rho^2}{r^2+\rho^2}x^\alpha.
\end{equation}

Instead of $\rbr{n+1}$ embedding coordinates (with the constraint 
$\sum_{i=1}^{n+1}\rbr{x^i}^2 = \rho^2$) we now have the $n$ intrinsic 
coordinates $x^\alpha$ on $S_\rho$ without constraint.

Line element? Induced metric on $S_\rho$:
\begin{equation}
\sum_{i=1}^{n+1} \rbr{\dd \ol{x}^i}^2 = \sum_{\alpha = 1}^{n} \rbr{\dd 
\ol{x}^\alpha}^2 + \rbr{\dd \ol{x}^{n+1}}^2.
\end{equation}
Replacing $\ol{x}$'s with $x$:
\begin{align}
\dd \ol{x}^{n+1} &= \rho^2 \frac{4\rho r \,\dd r}{\rbr{r^2+\rho^2}^2},
\label{eq:(1)again}\\
\dd \ol{x}^\alpha &= 2\rho^2 \frac{\rbr{r^2+\rho^2}\,\dd x^\alpha - 2x^\alpha 
r \,\dd r}{\rbr{r^2+\rho^2}^2}
\label{eq:(2)again}
\end{align}
so that
\begin{align}
\sum_{i=1}^{n+1} \rbr{\dd \ol{x}^i}^2 &= 
\frac{4\rho^4}{\rbr{r^2+\rho^2}^2} \sum_{\alpha = 1}^{n} \rbr{\dd x^\alpha}^2
\nonumber \\
&= \rbr{1+\frac{r^2}{4\rho^2}}^{-2} \sum_{\alpha = 1}^{n} \rbr{\dd x^\alpha}^2
\nonumber \\
&= \rbr{1+\frac{b}{4}r^2}^{-2} \sum_{\alpha = 1}^{n} \rbr{\dd x^\alpha}^2
\nonumber \\
&= \dd l^2 = \frac{\delta_{ab}\,\dd x^a\,\dd x^b}{\rbr{1+\frac{b}{4}r^2}^2}
\end{align}
where $r^2 = \sum_{\alpha=1}^n\rbr{x^\alpha}^2$ is used in the first line,
$x^\alpha \to x^\alpha/2$ on the second line, $b \coloneqq \rho^{-2}$ on the 
third line.
\end{proof}

2. $b < 0$: no smooth embedding possible into a Euclidean space $\BbbR^{n+1}$, 
but possible to embed in higher-dimensional space.

But embedding is possible into $\BbbR^{n+1}$ with \emph{Minkowski signature},
\begin{equation}
\BbbR^{n+1} \supset H_\rho \coloneqq \cbr{\ol{x}\in\BbbR^{n+1} \Bigg| 
\sum_{\alpha=1}^n \rbr{\ol{x}^\alpha}^2 - \rbr{\ol{x}^{n+1}}^2 = -\rho^2},
\end{equation}
which is a \emph{pseudosphere} or \emph{mass-shell}, which consists of 
\emph{two} simply connected hyperboloids, and can be embedded into 
$\BbbR^{n+1}$ with Euclidean signature when allowing an `edge'.

\textbf{diagrams}

Calculation analogous to the one above (same stereographic coordinates) gives 
instead of \cref{eq:(1)again,eq:(2)again},
\begin{align}
\ol{x}^{n+1} &= \rho\frac{r^2+\rho^2}{r^2-\rho^2},\\
\ol{x}^{\alpha} &= -\frac{2\rho^2}{r^2+\rho^2}x^\alpha,
\end{align}
so that
\begin{equation}
\sum_{\alpha=1}^n\rbr{\dd \ol{x}^\alpha}^2 - \rbr{\dd \ol{x}^{n_1}}^2
= \rbr{1+\frac{b}{4}r^2}^{-2}\sum_{\alpha=1}^n\rbr{\dd x^\alpha}^2,
\end{equation}
where $b\coloneqq -\rho^{-2} < 0$, and the result is the metric for space of 
constant negative curvature.

\paragraph{Summary} The $n$-dimensional Riemannian spaces of constant curvature 
with $b = \pm\rho^{-2}, 0$ are locally isometric to the $n$-dimensional sphere, 
pseudo-sphere and flat space, respectively.

\paragraph{Remarks}
\begin{enumerate}
\item On $S_\rho$ or $H_\rho$, the groups $\rfun{\SO}{n+1}$ or 
$\rfun{\SO}{n,1}$ operate naturally as isometry groups
\item $S_\rho$ and $H_\rho$ are spaces of \emph{maximal symmetry} and thus 
possess $n\rbr{n+1}/2$ independent Killing fields, which is $6$ for $n = d = 3$.
\item Global structure \emph{not} fixed; there are many other spaces with 
constant curvature) (Einstein equations determine local geometry, not global 
topology)
\end{enumerate}

Concerning topology, one has the following theorem: Two simply connected spaces 
of the same constant curvature are \emph{globally} isomorphic to each other, 
i.e.\ there is no additional global freedom.

$S_\rho$: simply connected; $H_\rho$: two simply connected pieces.

$n=3$: proof of Poincaré conjecture by Grigori Jakowlewitsch Perelman
(Григорий Яковлевич Перельман) \cite{math/0211159,math/0303109,math/0307245}, 
awarded Fields Medal in 2006: $S^3$ is the \emph{only} finite (compact without 
boundary) simply connected $3$-dimensional manifold.

$n=2, b=0$: five possibilities: plane (simply connected case), cylinder, torus 
(closed), Möbius strip (not orientable), Klein bottle (closed and not 
orientable).

Many more possibilities in higher dimensions; e.g.\ in $n=3$: apart from $S^3$, 
$S^3$ with opposite points identified, Poincaré manifold 
(`football')\footnote{follows from dodecahedron (12 pentagons): rotate every 
pentagon by $\pp/5$ and identify it with the opposite pentagon}, which was 
suggested in 2003 as model for our Universe (but soon falsified) ($M = 
S^3/\Gamma$, where $\Gamma$ is a finite subgroup of $\rfun{\SO}{4}$ which 
acts on $S^3$).

History: \cite{schwarzschild1900ueber,Reboucas2005}


\subsection{Robertson--Walker line element}
We have seen: observation plus cosmological principle lead to spaces of 
constant curvature (relevant for Universe at a given time). But what about 
\emph{space-time}?

\paragraph{1) Homogeneity of space}
There exists a one-parameter family of space-like hypersurfaces $\Sigma_t$ in 
space-time with constant $t$, which foliate space-time and whose elements are 
\emph{homogeneous}, which means there exists an isometry of the spatial metric 
that maps for any two given points $P, Q \in \Sigma_t$ the point $P$ into $Q$.

\paragraph{2) Isotropy of space}
Consider congruence of time-like curves (observers, galaxies, etc.)

\textbf{diagram}
$u^\mu$: distinguished $4$-velocity (cosmology: at rest with CMB), which is 
defined as co-moving observer.

$S^\mu_1$, $S^\mu_2$: spatial tangential vectors, orthogonal to $u^\mu$.

For every point $P$ and for any two spatial unit tangential vectors $S^\mu_1$ 
and $S^\mu_2$, there is an isometry of the metric which fixes $P$ and $u^\mu$, 
but rotates $S^\mu_1$ into $S^\mu_2$, i.e.\ there is no geometrically preferred 
tangential vector orthogonal to $u^\mu$.

(remark: isotropy only holds for co-moving observer)

Homogeneous surface $\Sigma_t$ must be \emph{orthogonal} to the world-lines, 
because otherwise one could construct a geometrically preferred spatial vector.

\textbf{diagram} Gaussian synchronous coordinates

\paragraph{Space-time line element}
\begin{empheq}[box=\fbox]{equation}
 \dd s^2 = -\dd t^2 + \rfun{a^2}{t}h_{ab}\,\dd x^a\,\dd x^b,
\end{empheq}
where $\rfun{a}{t}$ the `scale factor' or `expansion factor'; 
$\rfun{h_{ab}}{x^1, x^2, x^3}$ spatial metric (apart from $\rfun{a}{t}$), which 
is the same on each $\Sigma_t$, and describes the spaces of constant curvature 
\cref{ssec:constant-curvature}.
\begin{empheq}[box=\fbox]{equation}
 \dd s^2 = -\dd t^2 + \rfun{a^2}{t}\rbr{\frac{\dd r^2}{1-kr^2} + 
r^2\,\dd\Omega^2},
\end{empheq}
which is called the `\emph{Robertson--Walker line element}', or R--W line 
element; Howard Percy Robertson (\textborn 1903, \textdied 1961); Arthur 
Geoffrey Walker (\textborn 1909, \textdied 2001)

Isotropy is defined with respect to the cosmological microwave background; one 
can prove (use of Einstein equation is made)

\emph{EGS Theorem} \cite{Ehlers1968} If for a family of freely falling 
observers the cosmological microwave background is everywhere exactly 
isotropic, space-time is decided by the Robinson--Walker line element (thus 
definitely holds since the decoupling of radiation)

Remains true if one writes `almost isotropic' and `almost Robinson--Walker' 
\cite{Stoeger1995}

$k=1$: We already had in \cref{ssec:energy-density}
\begin{align}
 \dd l^2 &= a^2 \rbr{\frac{\dd r^2}{1-k r^2} + r^2\,\dd \phi^2} \nonumber \\
 &= a^2\rbr{\dd \chi^2 + \sin^2\chi\,\dd \phi^2},
\end{align}
where on the second line $r = \sin\chi$ was used. For $r=1$: coordinate 
singularity (`equator')

\textbf{diagram}

\paragraph{Robinson--Walker line element} For $\theta = \pp / 2$, we have 
$S^2$; therefore,
\begin{empheq}[box=\fbox]{equation}
 \dd s^2 = -\dd t^2 + \rfun{a^2}{t} \sbr{\dd\chi^2 + \sin^2\chi \, \rbr{\dd 
\theta^2 + \sin^2\theta \,\dd\phi^2}},
\end{empheq}
where $\rbr{\dd \theta^2 + \sin^2\theta \,\dd\phi^2}$ took place of $\dd\phi^2$ 
for $S^2$.

Spatial volumes is finite:
\begin{equation}
 V = \int_0^\pp \dd\chi\,\int_0^\pp\dd\theta\,\int_0^{2\pp}\dd\phi\, \sqrt{h} 
a^3,
\end{equation}
where $h \coloneqq \det h_{ab}$, and $\sqrt{h} a^3 \equiv \fat{\sqrt{\det 
g_{\mu\nu}}}{t=\text{const.}}$. Take
\begin{equation}
h_{ab} = \rfun{\diag}{1, \sin^2\chi, 
sin^2\chi\,\sin^2\theta},
\end{equation}
so that $\sqrt{h} = \sin^2\chi\,\sin\theta$, one has
\begin{equation}
 V = 4\pp a^3\int_0^\pp\dd\chi\,\sin^2\chi = 2\pp^2 a^3,
\end{equation}
which is finite, but unbounded universe! (not imaginable in Newtonian cosmology)

Consider now two-dimensional sphere $S^2$ \emph{in} $S^3$, i.e.\ around $\chi = 
0$.

\textbf{diagram}

connection of area and radius of $S^2$ \emph{in} $S^3$?

radius
\begin{equation}
 R = a \int_0^\chi \dd\tilde{\chi} = a\chi,
\end{equation}
where $\dd\theta = 0 = \dd\phi$.

surface
\begin{align}
 A &= a^2\sin^2\chi \int_0^\pp \dd\theta\,\sin\theta\,\int_0^{2\pp}\dd\phi
 \nonumber \\
 &= 4\pp a^2\sin^2\chi = 4\pp a^2\rfun{\sin^2}{\frac{R}{a}},
\end{align}
instead of $4\pp R^2$ for flat space.

With increasing $R$ (as seen from $\chi = 0$), $A$ increases up to 
$A_\text{max} = 4\pp a^2$ (for $R/a = \pp/2$: `equator') and then decreases to 
reach $A = 0$ at $R = \pp a$ (`south pole')

$S^2$ in Euclidean space:
\begin{equation}
A = 4\pp R^2 = \pp^3 a^2 > A_\text{max} = 4\pp a^2,
\end{equation}
where the first equation follows from $4\pp a^2\rfun{\sin^2}{\frac{R}{a}}$ for 
small $R/a$, the second equations follows from taking $R = \pp/2\cdot a$, 
resulting in surface increases faster.

$k = -1$: $0 \le r < +\infty$, $r = \sinh \chi$. So that
\begin{empheq}[box=\fbox]{equation}
\dd s^2 = -\dd t^2 + \rfun{a^2}{t}\sbr{\dd\chi^2 + \sinh^2\chi\,\rbr{\dd 
\theta^2 + \sin^2\theta\,\dd\phi^2}}.
\end{empheq}
Surface of $S^2$ now is $A = 4\pp a^2 \sinh^2\frac{R}{a}$.

\textbf{diagram}

Dependence of $A$ on distance -> information about $k$ (counting of galaxies as 
a function of brightness -> too imprecise; better via anisotropies of 
cosmological microwave background, see later)

\paragraph{Cosmological Red-shift} We had (Robertson--Walker line element)
\begin{equation}
\dd s^2 = -\dd t^2+\rfun{a^2}{t}\rbr{\dd\chi^2+\rfun{f^2}{\chi}\,\dd\Omega^2},
\end{equation}
where $\dd\Omega^2 = \rbr{\dd \theta^2 + \sin^2\theta \,\dd\phi^2}$, 
$\rfun{f}{\chi} = \sin\chi, \sinh\chi, \chi$ for $k = 1, -1, 0$, respectively.

\textbf{diagram}

spatial distance: $\rfun{d}{t} = \rfun{a}{t}\cdot\rbr{\chi-\chi_0}$, so 
that relative velocity
\begin{equation}
\rfun{\dot{d}}{t} = \frac{\dot{a}}{a} \rfun{d}{t} \eqqcolon \rfun{\hH}{t},
\label{eq:above-436}
\end{equation}
which is called the \emph{Hubble parameter}.

Most important: propagation of electromagnetic radiation (light), where $\dd 
s^2 = 0$, which is conveniently described by introducing a new time coordinate 
$\eta$ with
\begin{empheq}[box=\fbox]{equation}
\dd t \eqqcolon \rfun{a}{t}\,\dd\eta,
\end{empheq}
which is called the \emph{conformal time}. Then, the Robertson--Walker line 
element reads
\begin{equation}
\dd s^2 = \rfun{a^2}{t}\rbr{-\dd\eta^2+\dd\chi^2+\rfun{f^2}{\chi}\,\dd\Omega^2}.
\label{eq:conformal-time}
\end{equation}

Radial light propagation: $\theta, \phi = \text{const.}$, so that $\dd s^2 = 0$ 
yields
\begin{empheq}[box=\fbox]{equation}
 \dd\eta = \pm \dd\chi,
\end{empheq}
where information about $a$ has been contained in $\eta$. The solution reads
\begin{equation}
 \chi = \eta_0 - \eta,
\end{equation}
which describes light propagates towards observer who is at $\chi_0 = 0$.

\textbf{diagram; corresponds to `now'}

From $\dd\eta_s = \dd\eta_0$, one has
\begin{equation}
 \frac{\dd t_s}{\rfun{a}{t_s}} = \frac{\dd t_0}{\rfun{a}{t_0}}.
\end{equation}
Define $a_0 \coloneqq \rfun{a}{t_0}$, which is today's value of the scale 
factor. Angular frequency of light $\omega \propto \rbr{\dd t}^{-1}$  because 
$t$ corresponds to the local proper time for a co-moving observer ($\eta$ does 
not correspond to proper time)

\begin{equation}
 \omega_s \rfun{a}{t_s} = \omega_0 a_0,
\end{equation}
which means emitted frequency does not equal to received frequency.

Define
\begin{equation}
z \coloneqq \frac{\lambda_0 - \lambda_s}{\lambda_s} = \frac{\omega_s}{\omega_0} 
- 1 = \frac{a_0}{\rfun{a}{t_s}} - 1,
\end{equation}
which means
\begin{empheq}[box=\fbox]{equation}
 1+z = \frac{a_0}{\rfun{a}{t_s}},
\end{empheq}
which is greater than $1$ in observation, meaning the Universe is expanding. 
This is the central kinematical relations; connects observed $z$ with relative 
size of the Universe.

Let us consider ``neighbouring galaxies'', for which one can write
\begin{equation}
 \rfun{a}{t} = a_0 + \rbr{t-t_0} \dot{a}_0 + \rfun{\Omicron}{\rbr{t-t_0}^2}.
\end{equation}
Define $t-t_0 \eqqcolon -\Dva t$, so that $\Dva t > 0$. Then
\begin{equation}
z = \frac{a_0}{a_0 - \Dva t \dot{a}_0} - 1 \approx \Dva t \frac{\dot{a}_0}{a_0}
\approx \frac{d}{\lc}\frac{\dot{a}_0}{a_0} = \frac{\dot{d}_0}{\lc},
\end{equation}
where $\lc\,\Dva t \approx d$ (during $\Dva t$, distance does not change much), 
and we had above \cref{eq:above-436}. This yields
\begin{equation}
\dot{d}_0 = \lc z = d \hH_0,
\end{equation}
in \cite{Hubble1929}, and
\begin{equation}
 \hH_0 \coloneqq \frac{\dot{a}_0}{a_0},
\end{equation}
the \emph{Hubble constant}. These are the relations between \emph{red-shift} 
and \emph{distance}. For `neighbouring galaxies' (and negligible peculiar 
velocities), one can interpret the red-shift as a `Doppler effect' (relative 
velocity between sender and receiver) This is \emph{not} possible for finite 
$\Dva t$ (mixture of gravitational part), see below!

\subsection{Friedmann--Lemaître equations}
To derive the Friedmann--Lemaître (F--L) equation, we write
\begin{align}
\dd s^2 &= -\dd t^2 + \rfun{a^2}{t}\sbr{\frac{\dd r^2}{1-kr^2} + r^2 \, 
\dd\theta^2 + r^2\sin^2\theta\,\dd\phi^2} \nonumber \\
&= -\theta^0\otimes\theta^0 + \theta^1\otimes\theta^1 + \theta^2\otimes\theta^2 
+ \theta^3\otimes\theta^3 \equiv \eta_{\mu\nu}\,\theta^\mu\otimes\theta^\nu.
\end{align}
with the co-basis
\begin{equation}
 \theta^0 = \dd t,\quad \theta^1 = \frac{a}{w}\,\dd r,\quad
 \theta^2 = ar\,\dd \theta,\quad \theta^3 = a r \sin\theta\,\dd\phi,
\end{equation}
and $w\coloneqq \sqrt{1-kr^2}$.

Cartan 1:
\begin{equation}
 \dd\theta^\mu + \omega^\mu{}_\nu\wedge\theta^\nu = 0,
\end{equation}
where $\omega^\mu{}_\nu$ is to be determined.

Cartan 2:
\begin{align}
 \Omega^\mu{}_\nu &\coloneqq \dd\omega^\mu{}_\nu + \omega^\mu{}_\rho\wedge 
\omega^\rho{}_\nu \nonumber \\
&\eqqcolon \frac{1}{2}R^\mu{}_{\nu\rho\sigma}\,\theta^\rho\wedge\theta^\sigma,
\end{align}
where $R^\mu{}_{\nu\rho\sigma}$ is to be determined.

From there, $R_{\mu\nu}$ and $R$; the Einstein equations then yield the 
Friedmann--Lemaître equations. To be done in the exercises!

\begin{equation}
R_{\mu\nu} - \frac{1}{2} g_{\mu\nu} R + \Lambda g_{\mu\nu} = 8\pp\nG T_{\mu\nu},
\end{equation}
where $T_{\mu\nu} = \rbr{\rho+p}u_\mu u_\nu + pg_{\mu\nu}$ for ideal fluid. 
Then yields
\begin{empheq}[box=\fbox]{align}
 \ddot{a} &= -\frac{4\pp\nG}{3}\rbr{\rho+3p}a + \frac{\Lambda}{3}a
 \label{eq:FL(1)}\\
 \dot{a}^2 &= \frac{8\pp\nG}{3}\rho a^2 + \frac{\Lambda}{3}a^2 - k \nonumber \\
 &= \frac{8\pp\nG}{3}a^2\rbr{\rho+\frac{\Lambda}{8\pp\nG}} - k,
 \label{eq:FL(2)}
\end{empheq}
where $\Lambda/8\pp\nG$ is called $\rho_\text{V}$, `energy density of the 
vacuum'. These are the `Friedmann--Lemaître equations'.

From \cref{eq:FL(2)} one has
\begin{equation}
\dot{a}^2 a = \frac{8\pp\nG}{3}\rho a^3 + \frac{\Lambda}{3}a^3-ka.
\end{equation}
Solve this with respect to $\rho a^3$, perform the time derivative and use 
\cref{eq:FL(1)}
\begin{equation}
 \frde{}{t}\rbr{\rho a^3} + p\frde{}{t}\rbr{a^3} = 0.
 \label{eq:FL(3)}
\end{equation}
\Cref{eq:FL(3)} can also be written in the form
\begin{equation}
 \dot{\rho} + 3H\rbr{\rho+p} = 0,
\end{equation}
where $\rbr{\rho+p}$ is called the `positive gravitational mass'.

Take for $\rho$ the vacuum density $\rho_\text{V} = \text{const.}$, so that
\begin{equation}
 \dot{\rho}_\text{V} = 0 \quad\Rightarrow\quad \rho_\text{V} = -p_\text{V},
\end{equation}
which is the equation of state for `vacuum'.

Therefore, there is no need to write $\Lambda$ explicitly; this case is 
included by the equation of state $\rho = -p$ (modern way of writing the 
Friedmann--Lemaître equations)

Choice of equation of state $p = \rfun{f}{\rho, t}$ gives the dynamics of a 
cosmological model

Today's Universe: ordinary matter has $p/\rho\lc \lesssim \num{e-4}$, where 
$\rho$ has the dimension of mass density, which allows one to set $p = 0$ 
(`dust'). But we also have a vacuum contribution with $\rho = -p$.

Consider $p = 0$. \Cref{eq:FL(3)} leads to
\begin{equation}
 \rho a^3 = \text{const.} \eqqcolon \frac{3M}{4\pp}
\end{equation}
from analogy with Newtonian physics (mass conservation). \Cref{eq:FL(2)} then 
assumes the form
\begin{equation}
\frac{1}{2}\dot{a}^2 - \frac{\nG M}{a} - \frac{\Lambda a^2}{6} = -\frac{k}{2}
\eqqcolon \frac{E}{m}
\end{equation}
`energy law' in Newtonian cosmology (there, of course, $\Lambda = 0$) (for a 
co-moving volume), was discussed in one of the exercises.

Other important equations of state:
\begin{equation}
p = \frac{\rho}{3}
\end{equation}
for radiation, inserting into \cref{eq:FL(3)} yields
\begin{equation}
 \rho a^4 = \text{const.}
\end{equation}
(additional factor $a$ from the red-shift). Since $\rbr{\rho+3p}$ enters 
\cref{eq:FL(1)}, radiation contributes \emph{twice as much as dust} to 
$\ddot{a}$.

We had (\cref{eq:FL(2)})
\begin{equation}
 \dot{a}^2 = \frac{8\pp\nG}{3} a^2 \rbr{\rho+\rho_\text{V}} - k,
\end{equation}
divided by $a^2$ yields
\begin{equation}
 \hH^2 = \frac{8\pp\nG}{3} \rbr{\rho+\rho_\text{V}} - \frac{k}{a^2}.
 \label{eq:H^2}
\end{equation}
Define the \emph{critical density}
\begin{equation}
 \rho_\text{c} \coloneqq \frac{3\hH^2}{8\pp\nG},
\end{equation}
\cref{eq:H^2} turns to be
\begin{equation}
 1 = \frac{\rho+\rho_\text{V}}{\rho_\text{c}} - \frac{k}{a^2\hH^2}.
\end{equation}
Define
\begin{equation}
 \Omega \coloneqq \frac{\rho}{\rho_\text{c}},\quad
 \Omega_\Lambda \equiv \Omega_\text{V} \coloneqq 
\frac{\rho_\text{V}}{\rho_\text{c}},\quad
 \Omega_k \coloneqq -\frac{k}{a^2 \hH^2},
\end{equation}
one has
\begin{empheq}[box=\fbox]{equation}
 \Omega + \Omega_\text{V} + \Omega_\text{k} = 1
 \label{eq:FL(4)}
\end{empheq}
(if evaluated `today', write $\Omega_0 + \Omega_{\text{V}, 0} + 
\Omega{\text{k}, 0} = 1$; $\rho_{\text{c},0} \coloneqq 3\hH_0^2 / 8\pp\nG$)

Often, one writes $\Omega = \Omega_\text{m} + \omega_\text{r}$, where m and r 
stands for matter and radiations, respectively.

From \cref{eq:FL(4)}
\begin{equation}
 \Omega + \Omega_\text{V} = 1+\frac{k}{a^2\hH^2},
\label{eq:p445(4):}
\end{equation}
where $k/a^2\hH^2$ can be $0$, positive or negative, so that
\begin{equation}
 k = \begin{cases}
  0 & \rho+\rho_\text{V} = \rho_\text{c},\\
  +1 & \rho+\rho_\text{V} > \rho_\text{c},\\
  -1 & \rho+\rho_\text{V} < \rho_\text{c}.\\
 \end{cases}
\end{equation}
Total energy density $\rbr{\rho+\rho_\text{V}}$ fixes geometry (but not 
topology) of the Universe.

Today: $\Omega_{\text{r},0} \ll \Omega_{\text{m},0}$, so that $\Omega_0 \approx 
\Omega_{\text{m},0}$.

Often, the following diagrams are considered:

\textbf{diagram}

Our Universe corresponds to a point there (to be found from observation)

Historically, one also introduced the \emph{deceleration parameter}
\begin{equation}
 \dq \coloneqq -\frac{\ddot{a}a}{\dot{a}^2},
\end{equation}
which is a dimensionless version of $\ddot{a}$; $a > 0$ for $\ddot{a} < 0$ (as 
as historically believed to hold)

From \cref{eq:FL(1)}
\begin{equation}
\frac{\Lambda}{3} = \frac{4\pp\nG}{3}\rbr{\rho+3p} - \dq\hH^2
\label{eq:p446}
\end{equation}
for example, for $\Lambda = 0$, $p = 0$, one has $\dq = \Omega/2$.

This can be used to eliminate $\Lambda$ from \cref{eq:FL(2)}
\begin{align}
\frac{k}{a^2} &= \frac{8\pp\nG}{3}\rho + \sbr{\frac{4\pp\nG}{3}\rbr{\rho+3p} - 
\dq\hH^2} - \hH^2 \nonumber \\
&= \frac{\hH^2}{2}\sbr{3\rbr{\Omega+\frac{p}{\rho\lc}} - 2\dq - 2}
\end{align}
Evaluated today ($p = 0$ for matter part):
\begin{equation}
\frac{k}{a_0^2} = \frac{\hH_0^2}{2}\rbr{3\Omega_0 - 2\dq_0 - 2},
\end{equation}
where $\rbr{3\Omega_0 - 2\dq_0 - 2}$ determines the sign of $k$, because 
$\Omega_0$ and $\dq_0$ are in principle observable.

A Friedmann--Lemaître model is fixed by the \emph{four} parameters $\hH_0$, 
$\Omega_{\text{r},0}$, $\Omega_{\text{m},0}$, $\Omega_{\text{V},0}$ (or 
$\dq_0$), 
and the equation of state. 

Hubble parameter: has dimension $\rbr{\text{time}}^{-1}$; in astronomy, one 
uses \si{\kilo\metre\per\second\per\mega\parsec} (because $v = \lc z = d 
\hH_0$, 
where $\sbr{v} = \si{\kilo\metre\per\second}$, $\sbr{d} = \si{\mega\parsec}$)

Often: dimensionless Hubble constant,
\begin{equation}
 \hh \coloneqq \frac{\hH_0}{\SI{100}{\kilo\metre\per\second\per\mega\parsec}}
\end{equation}

\emph{Hubble time}:
\begin{equation}
 t_\text{H} \coloneqq \hH_0^{-1} \approx \num{9.78}\,\hh^{-1}\,\si{Gyr}.
\end{equation}
For $\hh\approx \num{.67}$\footnote{which is the current value, but see debates 
about possible higher value from Cepheid (\textbf{?})}, this gives $t_\text{H} 
\approx \SI{14.6}{Gyr}$, which coincides approximately (by chance!) with the 
age of the Universe of $\approx \SI{13.8}{Gyr}$.

Critical density `today':
\begin{equation}
\rho_{\text{c}, 0}  = \frac{3\hH_0^2}{8\pp\nG} \approx 
\num{1.88e-29}\,\hh^2\,\si{\gram\per\cubic\centi\metre}.
\end{equation}

Today, $p \approx p_\text{m} \approx 0$. Did this always hold?

From observation of cosmological microwave background,
\begin{equation}
\Omega_{\text{r}, 0} \eqqcolon \Omega_\text{CMB} \approx \num{2.5e-5}\,\hh^{-2} 
\approx \num{5.6e-5} \ll 1,
\end{equation}
where $\hh\approx \num{.67}$ has been used. But because $\rho_\text{r} a^4 = 
\text{const.}$, this radiation has dominated in the past (\emph{hot Big Bang}). 
Details later (\textbf{section 13 ?})

What can generally be said about the past? From \cref{eq:FL(1)}
\begin{equation}
 \ddot{a} = -\frac{4\pp\nG}{3} a \rbr{\rho + 3p - \frac{\Lambda}{4\pp\nG}},
\end{equation}
where ${\Lambda}/{4\pp\nG} = 2\rho_\text{V} = -\rho_\text{V} - 3p_\text{V}$. 
\emph{If}
\begin{equation}
\rho + 3p - \frac{\Lambda}{4\pp\nG} \ge 0\quad\hookrightarrow\quad 
\ddot{a} \le 0
\end{equation}

\textbf{diagram}

Then, $\rfun{a}{t} = 0$ (singularity in space-time) at a finite time in the 
past ($\rho \to +\infty$ as $a \to 0^+$, curvature invariants $\to \infty$ as 
$a \to 0^+$). But we observe that `today' we have
\begin{equation}
 \rho + 3p - 2\rho_\text{V} \approx \rho - 2\rho_\text{V} < 0
\end{equation}
(converse graph). Fro times $t < \tilde{t}$, however, a concave graph is valid, 
and thus we have a \emph{singularity} in spite of today having $\rho + 3p - 
2\rho_\text{V} < 0$, c.f.\ \emph{singularity theorems} \textbf{section 10.7 ?} 
There, the strong energy condition is assumed,
\begin{equation}
 T_{\mu\nu}\xi^\mu\xi^\nu /ge \frac{T}{2},\qquad\forall\text{ time-like 
}\xi^\mu,
\end{equation}
which means
\begin{equation}
 \rho \ge \frac{1}{2}\rbr{\rho-3p} \qquad\text{or}\qquad \rho+3p \ge 0.
\end{equation}
This can be violated by $\Lambda$, but the \emph{observed} value for $\Lambda$ 
is not large enough to ensure singularity avoidance, see later.


\subsection{Distance measure}
Observation: given relation between red-shift $z$ and energy flux
(\emph{apparent magnitude}). Above we had $\lc z = d \hH_0$. What is $d$?
(needed for $\hH_0$-determination). For instance, in \emph{flat space}:
\begin{equation}
l = \frac{L}{4\pp d^2},
\label{eq:l-L-flat}
\end{equation}
where $l$ is the observed energy flux in
\si{erg\per\second\per\square\centi\metre}, $L$ source luminosity in
\si{erg\per\second}, $d$ distance in flat space, and $\SI{1}{erg} =
\SI{e-7}{\joule}$.

Hubble Law (if $t$-dependence of $a$ is neglected)
\begin{equation}
z = d \hH_0 \quad \Rightarrow \quad \frac{l}{L} = \frac{\hH_0^2}{4\pp z^2},
\end{equation}
which is the relation between red-shift and brightness.

Generalisation to Friedmann models?
\begin{equation}
\dd s^2 = -\dd t^2 + \rfun{a^2}{t}\rbr{\frac{\dd r^2}{1-k r^2}
+ r^2\,\dd\Omega^2}.
\end{equation}

\textbf{diagram}

$D$: transversal physical extension of a remote object, $\delta$: angular
extension of the object.

Define the following measures
\begin{enumerate}
\item angular distance
\begin{equation}
d_\text{A} \coloneqq \frac{D}{\delta},
\end{equation}
which would --- for small $\delta$ --- be the ordinary distance in flat space.
\item proper motion distance
\begin{equation}
d_\text{P} \coloneqq \frac{v_\perp}{\dd\delta/\dd t_0},
\end{equation}
c.f.\ $v = \omega r$; $\dd\delta/\dd t_0$ is the apparent angular velocity.
\item luminosity distance, which is the most important
\begin{equation}
d_\text{L} \coloneqq \sqrt{\frac{L}{4\pp\l}},
\end{equation}
which is motivated by the above flat-space relation, \cref{eq:l-L-flat}.
\end{enumerate}

These three measures are connected with each other: from $\dd s^2 = \ldots$ one
has for $t = t_1$, $r = r_1$,
\begin{equation}
D = \rfun{a}{t_1}r_1\delta = d_\text{A}\delta \quad \Rightarrow \quad
d_\text{A} = \rfun{a}{t_1}r_1,
\end{equation}
the infinitesimal version of which reads $dD = \rfun{a}{t_1}r_1d\delta$. Also
\begin{equation}
v_\perp \coloneqq \frde{D}{t_1} \quad\Rightarrow\quad
\dd D = v_\perp\,\dd t_1 = v_\perp\,\dd t_0 \frac{\rfun{a}{t_1}}{a_0},
\end{equation}
so that
\begin{equation}
\dd\delta = \frac{\dd D}{\rfun{a}{t_1}r_1} = \frac{v_\perp\,\dd t_0}{a_0 r_1},
\end{equation}
resulting in
\begin{equation}
d_\text{P} = \frac{v_\perp}{\dd \delta/\dd t_0} = a_0 r_1
= d_\text{A} \frac{a_0}{\rfun{a}{t_1}} = d_\text{A}\cdot\rbr{1+z}.
\end{equation}

What about $d_\text{L}$? object emits in time $\dd t_1$ the energy $L\,\dd t_1$;
this energy is \emph{red-shifted} by the factor $\rfun{a}{t_1}/a_0$ and
\emph{distributed} over the surface $4\pp\rbr{r_1 a_0}^2$, which comes from
$\dd s^2 = \ldots a^2 r^2 \,\dd \Omega^2$, so that
\begin{align}
l &= \underbrace{L\,\dd t_1 \frac{\rfun{a}{t_1}}{a_0}
\frac{1}{4\pp r_1^2 a_0^2}}_\text{incoming energy per surface}
\frac{1}{\dd t_0} \nonumber \\
&= \frac{L}{4\pp}\frac{\rfun{a^2}{t_1}}{a_0^4 r_1^2} \quad \Rightarrow \quad
4\pp l = L \frac{\rfun{a^2}{t_1}}{a_0^4 r_1^2},
\end{align}
and
\begin{align}
d_\text{L} &= \sqrt{\frac{L}{4\pp l}} = \frac{a_0^2 r_1}{\rfun{a}{t_1}}
= d_\text{P}\frac{a_0}{\rfun{a}{t_1}} \nonumber \\
&= d_\text{P}\cdot\rbr{1+z} = d_\text{A}\cdot\rbr{1+z}^2.
\end{align}

Discuss now the relation between $d_\text{L}$ and $z$ independent of the
dynamics, i.e.\ purely kinematically. (\cref{ssec:12.7}: derived with the use
of the Friedmann equations).

\textbf{diagram}

radial light rays: $\dd s^2 = 0$, $\dd \Omega = 0$,
\begin{equation}
\int_{t_1}^{t_0}\frac{\dd t}{\rfun{a}{t}} = \int_0^{r_1}
\frac{\dd t}{\sqrt{1-kr^2}}.
\label{eq:hH*}
\end{equation}
Assume that $z$, $\rbr{t_0 - t_1}$ and $r_1$ are `small enough' to justify a
Taylor expansion
\begin{equation}
\rfun{a}{t} = a_0\sbr{1 + \hH_0\rbr{t-t_0} - \frac{1}{2}\dq_0 \hH_0^2
\rbr{t-t_0}^2 + \rfun{\Omicron}{\rbr{t-t_0}^3} },
\end{equation}
where $\hH_0 = \dot{a}_0/a_0$, $\dq_0 = -\ddot{a}_0 a_0 / \dot{a}_0^2$. So that
\begin{align}
z &= \frac{a_0}{\rfun{a}{t_1} } - 1
= \rbr{1 + \hH_0\rbr{t_1-t_0} - \frac{1}{2}\dq_0 \hH_0^2 \rbr{t_1 - t_0}^2
+ \rfun{\Omicron}{\rbr{t-t_0}^3}}^{-1} - 1 \nonumber \\
&\approx \hH_0\rbr{t_0-t_1} + \rbr{1+\frac{\dq_0}{2} }\hH_0^2\rbr{t_0 - t_1}^2
+ \rfun{\Omicron}{\rbr{t-t_0}^3}.
\end{align}
Solve recursively
\begin{equation}
t_0 - t_1 = \hH_0^{-1}\sbr{z - \rbr{1+\frac{\dq_0}{2}}z^2 + 
\rfun{\Omicron}{z^3}}.
\label{eq:hH**}
\end{equation}
From above,
\begin{equation}
d_\text{L} = \rbr{1+z} d_\text{P} = \rbr{1+z}r_1 a_0.
\label{eq:p455dL}
\end{equation}
Use
\cref{eq:hH*}: left hand side
\begin{equation}
\int_{t_1}^{t_0} \frac{\dd t}{\rfun{a}{t} } =
\frac{1}{a_0}\int_{t_1}^{t_0}
\sbr{1+\hH_0\rbr{t_0}+\rbr{1+\frac{\dq_0}{2} }\hH_0^2\rbr{t_0 - t_1}^2
+ \rfun{\Omicron}{\rbr{t-t_0}^3}}\,\dd t;
\end{equation}
right hand side
\begin{equation}
\int_0^{r_1} \frac{\dd t}{\sqrt{1-kr^2}} = \frac{1}{\sqrt{k}}
\rfun{\arcsin}{r_1\sqrt{k}} \approx r_1 + \frac{1}{6} r_1^3 k,
\end{equation}
where the second term is also neglected (i.e.\ in this approximation, result
is independent of $k$). One has
\begin{align}
r_1 &= \frac{1}{a_0}\sbr{\rbr{t_0-t_1} + \frac{1}{2}\hH_0\rbr{t_0 - t_1}^2 +
\rfun{\Omicron}{\rbr{t-t_0}^3} } \nonumber \\
&= \frac{1}{a_0}\sbr{z - \frac{1}{2}\rbr{1+\dq_0}z^2 + \rfun{\Omicron}{z^3} }
\end{align}
by using \cref{eq:hH**}. Thus,
\begin{align}
d_\text{L} &= \rbr{1+z}r_1 a_0 = \frac{1}{\hH_0} \rbr{1+z}
\sbr{z+\frac{1}{2}\rbr{1+\dq_0}z^2 +\rfun{\Omicron}{z^3} } \nonumber \\
&= \hH_0^{-1}\sbr{z + \frac{1}{2}\rbr{1-\dq_0}z^2 + \rfun{\Omicron}{z^3} },
\label{eq:eq-in-455}
\end{align}
which includes the original Hubble Law and the first correction, only the former
one of which can be interpreted as a Doppler shift.

\begin{equation}
l = \frac{L}{4\pp d_\text{L}^2} = \frac{L \hH_0^2}{4\pp z^2}
\sbr{1 + \rbr{\dq_0 - 1}z + \rfun{\Omicron}{z^2}},
\end{equation}
where the leading term corresponds to the flat-space relation 
\cref{eq:l-L-flat}. One can continue the expansion to higher orders in $z$: 
\cite{Visser2005} has given expressions up to $\rfun{\Omicron}{z^4}$, which can 
be relevant for observations; his result reads up to $\rfun{\Omicron}{z^3}$ as 
follows
\begin{equation}
\rfun{d_\text{L}}{z} = \frac{z}{\hH_0} \sbr{1+\frac{1}{2}\rbr{1-\dq_0}z 
- \frac{1}{6}\rbr{1-\dq_0-3\dq_0^2+j_0+\frac{k}{\hH_0^2 a_0^2}}z^2 + 
\rfun{\Omicron}{z^3}},
\end{equation}
where $j_0 \coloneqq a_0^{-1} \dddot{a}_0\rbr{\dot{a}_0/a_0}^{-3}$ is the 
\emph{jerk} (Ruck). In astronomy, use is made instead of $l$ and $L$ the 
\emph{apparent} and \emph{absolute} magnitudes $m$ and $M$,
\begin{equation}
L \eqqcolon \text{const.}\times 10^{-\frac{2}{5}M},\qquad
l \eqqcolon \text{const.}\times 10^{-\frac{2}{5}m}
\end{equation}
so that Sun has $M = \num{4.72}$ and $m = \num{-26.85}$. One also has
\begin{equation}
M = \num{4.72} - \num{2.5}\ln\frac{L}{L_\astrosun},\qquad
m_1 - m_2 = -\num{2.5}\ln\frac{l_1}{l_2}.
\end{equation}
$M$ is the apparent magnitude which the object would have at a distance of 
\SI{10}{\parsec}, i.e.\ 
\begin{equation}
d_\text{L} = \sqrt{\frac{L}{4\pp l}} = 10^{1+\rbr{m-M}/5}\,\si{\parsec},
\end{equation}
where the \emph{distance modulus} (Entfernungsmodul)
\begin{equation}
m - M = 5\ln\frac{d_\text{L}}{\SI{10}{\parsec}}
= 5\sfun{\ln}{d_\text{L}\,\si{\mega\parsec}} + 25.
\end{equation}
Inserting \cref{eq:eq-in-455} yields
\begin{equation}
m-M = 25-5\sfun{\ln}{\hH_0\,\si{\kilo\metre\per\second\per\mega\parsec}}
+ 5\sfun{\ln}{\lc z\,\si{\kilo\metre\per\second}}
+ \frac{2.5}{\ln 10}\rbr{1-\dq_0} z + \ldots,
\end{equation}
where $2.5/\ln 10 \approx \num{1.086}$. In principle, one can determine from 
this $\hH_0$, $\dq_0$ from observation at small $z$ (supernovae Ia, see later). 
For very small $z$, only $\hH_0$ enters, but problems with peculiar velocities.



\subsection{Temporal evolution of the scalar factor ($\Lambda = 0$)}
\label{ssec:12.7}
We had
\begin{equation}
\Omega + \Omega_\Lambda + \Omega_k = 1,
\tag{\ref{eq:FL(4)} revisited}
\end{equation}
where $\Omega_k = -k / a^2 \hH^2$. For $\Omega_\Lambda = 0$, it reduces to
\begin{equation}
\Omega - \frac{k}{a^2 \hH^2} = 1.
\end{equation}
For $\Lambda \neq 0$, see \cref{ssec:12.9}, which is more realistic `today'.

When evaluated `today', $\Omega_0 - k/a_0^2 \hH_0^2 = 1$, so that
\begin{equation}
k = a_0^2 \hH_0^2\rbr{\Omega_0 -1},
\label{eq:k=a_0^2}
\end{equation}
where
\begin{equation}
\Omega_0 = \Omega_{\text{m},0} + \Omega_{\text{r},0} \approx 
\Omega_{\text{m},0},\quad \Rightarrow \quad p = 0.
\end{equation}
We had
\begin{equation}
\frac{\Lambda}{3} = \frac{4\pp\nG}{3}\rbr{\rho+3p} - \dq\hH^2,
\tag{\ref{eq:p446} revisited}
\end{equation}
where $q = - \ddot{a} a / \dot{a}^2$. For $\Lambda = 0$ and $p = 0$, $q = 
\Omega / 2$. From Friedmann--Lemaître equation \eqref{eq:FL(2)},
\begin{equation}
\dot{a}^2 = \frac{8\pp\nG}{3}a^2 \rho - k.
\end{equation}
Divided by $a_0^2$ yields
\begin{equation}
\rbr{\frac{\dot{a}}{a_0}}^2 = \frac{8\pp\nG}{3}a^2 \rho_0 \rbr{\frac{a_0}{a}}^3 
- a_0^2 \hH_0^2\rbr{\Omega_0 -1}. 
\label{eq:divided-by-a02}
\end{equation}
where $\rho = \rho_0 \rbr{a_0/a}^3$ and \cref{eq:k=a_0^2} are used. Use
$8\pp\nG/3 = \hH_0^2/\rho_{\text{c},0}$:
\begin{equation}
\rbr{\frac{\dot{a}}{a_0}}^2 = \hH_0^2\rbr{1-\Omega_0+\Omega_0\frac{a_0}{a}}.
\end{equation}
Set
\begin{equation}
x \coloneqq \frac{a}{a_0} = \rbr{1+z}^{-1},
\end{equation}
so that
\begin{equation}
\dot{x}^2 = \hH_0^2 \rbr{1 - \Omega_0 + \frac{\Omega_0}{x}},
\label{eq:p459-dotx^2}
\end{equation}
where everything is related to \emph{present} values. Integration yields
\begin{equation}
t_0 - t_* = \hH_0^{-1}\int_{x_*}^1 \dd \tilde{x}\,\rbr{1-\Omega_0 + 
\frac{\Omega_0}{\tilde{x}}}^{-1/2}
\label{eq:p459(*)},
\end{equation}
which is the age of a `matter-dominated universe' from an early time $t_*$ up 
to `today'. The general case would be in \cref{eq:p459(*)} $\Omega_0$ 
\emph{plus} $\Omega_\Lambda$ and $\Omega_\text{r}$.

Special solutions for $\rfun{a}{t}$ and $p = 0$:

\paragraph{a) $k = 0$} $\Omega = 1$, i.e.\ $\rho = \rho_\text{c}$ at all 
times (\emph{fixed point}), so that
\begin{equation}
\rbr{\frac{\dot{a}}{a_0}} = \hH_0^2 \frac{a_0}{a},
\end{equation}
and
\begin{empheq}[box=\fbox]{equation}
\rfun{a}{t} = a_0\cdot\rbr{\frac{3}{2}\hH_0 t}^{2/3} \propto t^{2/3}.
\end{empheq}
This is called the \emph{Einstein--de Sitter} or \emph{EdS Universe} 
\cite{Einstein1932}.

\begin{equation}
\rfun{\hH}{t} = \frac{\dot{a}}{a} = \frac{2}{3t}\quad\Rightarrow\quad
\rfun{\hH^{-1}}{t} = \frac{3}{2} t,
\end{equation}
and
\begin{equation}
\rfun{\rho}{t} = \rho_0 \rbr{\frac{a_0}{a}}^3 = \frac{4\rho_0}{9 \hH_0^2 t^2}
= \frac{1}{6\pp\nG t^2},
\end{equation}
where $\rho_0 = \rho_{\text{c},0}$ has been used.

Age of this hypothetical universe: set $a = a_0$,
\begin{equation}
t_0 = \frac{2}{3}\hH_0^{-1},
\end{equation}
which also follows from \cref{eq:p459(*)}. Putting $\hh_0 \approx \num{.67}$,
$t_0 \approx \SI{9.9}{Gyr}$, which is less than the age of globular cluster, 
$\approx \SI{12}{Gyr}$. Einstein--de Sitter universe is in conflict with 
observation!

\paragraph{b) $k = +1$} $\rho_0 > \rho_{\text{c},0}$, so that $\Omega_0 > 1$, 
$\dq_0 > 1/2$. One can show 
that the solution curves for $\rfun{a}{t}$ are \emph{cycloids}, given in 
parameter representation by
\begin{align}
\rfun{a}{\eta} = \frac{\Omega_0}{2\hH_0\rbr{\Omega_0-1}^{3/2}} 
\rbr{1-\cos\eta} &\eqqcolon \frac{a_\text{max}}{2}\rbr{1-\cos\eta}, \\
\rfun{t}{\eta} = \frac{\Omega_0}{2\hH_0\rbr{\Omega_0-1}^{3/2}} 
\rbr{\eta-\sin\eta} &\eqqcolon \frac{a_\text{max}}{2}\rbr{\eta-\sin\eta},
\end{align}
where $0 \le \eta \le 2\pp$, and $a_\text{max}$ is reached for $\eta = \pp$. 
One sees $\dd t/\dd \eta = a$, so that $\eta$ is \emph{conformal time}.

\textbf{diagram}

is a model for an \emph{expanding} and \emph{re-collapsing} universe, where
\begin{equation}
a_\text{max} \coloneqq \frac{\Omega_0}{\hH_0\rbr{\Omega_0-1}^{3/2}}
= a_0 \frac{\Omega_0}{\Omega_0 - 1},
\end{equation}
where in the second step, $k = 1 = a_0^2\hH_0^2\rbr{\Omega_0 - 1}$ is used.

For instance, when $\Omega_0 = 2$, $a_\text{max} = 2a_0$. $a_\text{max}$ is 
reached at $\rfun{t}{a_\text{max}} = \pp/2\lc \cdot a_\text{max}$ for $\eta = 
\pp$. Big crunch is reached at $t_\text{f} = \pp a_\text{max}$. Age of this 
model universe would be ($\Omega = \Omega_\text{m}$)
\begin{align}
t_0 &= \hH_0^{-1}\int_0^1\dd\tilde{x}\, 
\rbr{1-\Omega_0+\frac{\Omega_0}{\tilde{x}}}^{-1/2} \nonumber\\
&= \frac{\Omega_0}{2\hH_0\rbr{\Omega_0-1}^{3/2}}
\sbr{\rfun{\arccos}{\frac{2}{\Omega_0} - 1} - \frac{2}{\Omega_0} 
\rbr{\Omega_0 - 1}^{1/2}} < \frac{2}{3}\hH_0^{-1}.
\end{align}
When $t_0 < 2/3\cdot \hH_0^{-1}$, age is \emph{smaller} than the age of the 
corresponding Einstein--de Sitter universe, and is thus observationally `even 
more' excluded.

Intuitively, $k = +1$ universe is more decelerated than $k = 0$ universe, i.e.\ 
in direction of the past it is more accelerated, so it has a shorter age.

\begin{equation}
\rfun{\dq}{\eta} = \frac{1}{1+\cos\eta},\qquad \frac{1}{2} \le \dq < +\infty
\end{equation}


\paragraph{c) $k = -1$} $\rho_0 < \rho_{\text{c},0}$, so that $\Omega_0 < 1$, 
$\dq_0 < 1/2$. Solution follows from the $k = +1$ solution by analytic 
continuation $\eta \mapsto -\ii \eta$:
\begin{align}
\frac{a}{a_0} = \frac{\Omega_0}{2\rbr{\Omega_0-1}} \rbr{1-\cos\eta}
&\xrightarrow{\cos\ii\eta = \cosh\eta}
\frac{\Omega_0}{2\rbr{\Omega_0-1}} \rbr{1-\cosh\eta}, \\
\hH_0 t = \frac{\Omega_0}{2\rbr{\Omega_0-1}^{3/2}} \rbr{\eta-\sin\eta}
&\xrightarrow{\sin\ii\eta = \ii\sinh\eta}
\frac{\Omega_0}{2\rbr{\Omega_0-1}} \rbr{-\ii\eta+\ii\sinh\eta} \nonumber \\
&= \frac{\Omega_0}{2\ii\rbr{1-\Omega_0}} \rbr{-\ii\eta+\ii\sinh\eta}.
\end{align}
One has
\begin{align}
\rfun{a}{\eta} = \frac{\Omega_0}{2\hH_0\rbr{1-\Omega_0}^{3/2}} 
\rbr{\cosh\eta-1},\\
\rfun{t}{\eta} = \frac{\Omega_0}{2\hH_0\rbr{1-\Omega_0}^{3/2}} 
\rbr{\sinh\eta-\eta},
\end{align}
where $0 \le \eta < +inf$. Eternal expansion; for $\eta \to +\infty$, 
$\rfun{a}{t} \propto t$.

Age of this model universe:
\begin{equation}
t_0 = \hH_0^{-1}\sbr{\frac{1}{1-\Omega_0} - 
\frac{\Omega_0}{2\rbr{1-\Omega_0}^{3/2}}
\rfun{\arccosh}{\frac{2}{\Omega_0}-1}},
\end{equation}
obeys $2/3\hH_0 < t_0 < 1/\hH_0$, where $1/\hH_0$ is the age of `empty' 
universe ($\dot{a}^2 = -k = 1\quad\hookrightarrow\quad\rfun{a}{t}\propto t$)

Age is bigger than $k = 0$ age,
\begin{equation}
 \rfun{\dq}{\eta} = \frac{1}{1+\cosh\eta}\qquad 0 \le \dq \le \frac{1}{2}.
\end{equation}

\textbf{diagram},
Corresponds to the three possibilities of Newtonian cosmology, where $k = +1$ 
corresponds to negative, $k=0$ vanishing, and $k = -1$ positive energy.

Connection between emission time $t_\text{E}$ and red-shift $z$: we had
\begin{equation}
\dot{x}^2 = \hH_0^2 \rbr{1 - \Omega_0 + \frac{\Omega_0}{x}},
\qquad x = \frac{a}{a_0},
\tag{\ref{eq:p459-dotx^2} revisited}
\end{equation}
so that
\begin{equation}
t_\text{E} = \hH_0^{-1}\int_0^{\rbr{1+z}^{-1}}\dd y\,
\rbr{1 - \Omega_0 + \frac{\Omega_0}{y}}^{-1/2},
\end{equation}
where $t_\text{E}$ is the age of universe at time of emission, and the lower 
integration limit $0$ resp.\ from validity of $p = 0$ on.

$\rbr{1+z}^{-1} = \rfun{a}{t_\text{E}}/a_0$; `look back time' $t_0 - 
t_\text{E}$.

For instance $\Omega_0 = 1$ ($\dq_0 = 1/2$):
\begin{align}
t_0 - t_\text{E} &= t_0 - \hH_0^{-1} \int_0^{\rbr{1+z}^{-1}}\dd y\,\sqrt{y}
\nonumber \\
&= \frac{2}{3}\hH_0^{-1} - \frac{2}{3} \hH_0^{-1}\rbr{1+z}^{-3/2}
\nonumber \\
&= \frac{2}{3}\hH_0^{-1}\sbr{1-\frac{1}{\rbr{1+z}^{3/2}}}.
\end{align}
$k=+1$: smaller ($k=-1$: larger) look back time (evaluation numerically).

\paragraph{Exact relationship between distance and red-shift}

Take here $\Omega_\Lambda$ into account (see \cref{ssec:12.9}).
\begin{align}
\hH^2 &= \frac{8\pp\nG}{3} \rho - \frac{k}{a^2}
 \tag{\ref{eq:H^2} revisited} \\
&= \frac{8\pp\nG}{3} \rbr{\rho_{\text{m},0}\rbr{\frac{a_0}{a}}^3 + 
\rho_\text{\Lambda}} - \frac{k}{a^2},
\end{align}
where
\begin{equation}
\rho_\Lambda = \frac{\Lambda}{8\pp\nG}.
\end{equation}
Note that
\begin{equation}
\Omega_{\Lambda,0} = \frac{\rho_\Lambda}{\rho_{\text{c},0}}
= \rho_\Lambda \frac{8\pp\nG}{3\hH_0^2}\quad\Rightarrow\quad
\rho_\Lambda = \frac{3\hH_0^2}{8\pp\nG}\Omega_{\Lambda,0},
\end{equation}
one has
\begin{equation}
\hH^2 = \hH_0^2\rbr{\frac{a_0}{a}}^3\Omega_{\text{m},0} + 
\hH_0^2 \Omega_{\Lambda,0} - \frac{k}{a^2}.
\end{equation}
Divided by $\hH_0^2$ yields
\begin{equation}
\rbr{\frac{\dot{a}a_0}{a\dot{a}_0}}^2
\rbr{\frac{\hH}{\hH_0}}^2 = \rbr{\frac{a_0}{a}}^3\Omega_{\text{m},0} + 
\Omega_{\Lambda,0} - \frac{k}{a^2\hH_0^2}.
\end{equation}
Multiplying with $a^2/a_0^2$ yields
\begin{equation}
\rbr{\frac{\dot{a}}{\dot{a}_0}}^2 = \frac{a_0}{a}\Omega_{\text{m},0} + 
\rbr{\frac{a}{a_0}}^2\Omega_{\Lambda,0} - \frac{k}{a_0^2\hH_0^2},
\end{equation}
where
\begin{equation}
- \frac{k}{a_0^2\hH_0^2} = \Omega_k
= 1 - \Omega_{\text{m},0} - \Omega_{\Lambda,0}.
\end{equation}
One therefore has
\begin{equation}
\rbr{\frac{\dot{a}}{\dot{a}_0}}^2 = \hH_0^2\rbr{1-\Omega_{\text{m},0} - 
\Omega_{\Lambda,0}+\frac{a_0}{a}\Omega_{\text{m},0} + 
\rbr{\frac{a}{a_0}}^2\Omega_{\Lambda,0}}.
\end{equation}
Setting $y \coloneqq a/a_0$, one finally has
\begin{empheq}[box=\fbox]{equation}
\dot{y} = \hH_0\rbr{1-\Omega_{\text{m},0}-\Omega_{\Lambda,0} + 
\frac{\Omega_{\text{m},0}}{y} + 
y^2\Omega_{\Lambda,0}}^{1/2}.
\end{empheq}
Noting $\dd s^2 = 0$, one has
\begin{align}
\int_{t_1}^{t_0}\frac{\dd t}{\rfun{a}{t}} &=
\int_0^{r_1}\frac{\dd r}{\sqrt{1-kr^2}} =
a_0^{-1}\int_{a_1/a_0}^1\frac{\dd y}{y\dot{y}} \nonumber \\
&= \rbr{a_0\hH_0}^{-1} \int_{\rbr{1+z}^{-1}}^1 \frac{\dd y}%
{y\sqrt{1-\Omega_{\text{m},0}-\Omega_{\Lambda,0} + 
\frac{\Omega_{\text{m},0}}{y} + y^2\Omega_{\Lambda,0}}},
\label{eq:p465star}
\end{align}
where $\Omega_{\text{m},0}/y$ is an effective `Coulomb potential', whereas 
$y^2\Omega_{\Lambda,0}$ `oscillator potential'.

\begin{equation}
d_\text{L} = \rbr{1+z}r_1 a_0.
\tag{\ref{eq:p455dL} revisited}
\end{equation}
From \cref{eq:p465star}: follows from
\begin{equation}
\int \frac{\dd r}{\sqrt{1-k r^2}} =
\begin{cases}
\arcsin \frac{r}{r_0} + C & k = +1,\\
\arcsinh \frac{r}{r_0} = \rfun{\ln}{\frac{r}{r_1} + 
\sqrt{1+\rbr{\frac{r}{r_1}}^2}} & k = -1,\\
r + r_0 & k = 0,
\end{cases}
\end{equation}
one has
\begin{equation}
\int_0^{r_1} \frac{\dd r}{\sqrt{1-k r^2}} =
\int_{a_1}^{a_0}\frac{\dd a}{a \dot{a}}.
\end{equation}

Alternative from involving $\rfun{\hH}{z}$: use $\tilde{z} = a_0/a - 1$ as 
integration variable
\begin{equation}
\dd \tilde{z} = -\frac{a_0}{a^2}\,\dd a \quad\Rightarrow\quad
\dd a = -\frac{a^2}{a_0}\,\dd\tilde{z},
\end{equation}
so that
\begin{equation}
\int_{a_1}^{a_0}\frac{\dd a}{a \dot{a}} =
-\frac{1}{a_0} = \int_z^0\dd\tilde{z}\,\frac{a}{\dot{a}} = 
\frac{1}{a_0}\int_0^z\frac{\dd \tilde{z}}{\rfun{\hH}{\tilde{z}}}.
\end{equation}
For $k = 0$,
\begin{equation}
\fat{\int_0^{r_1}\frac{\dd r}{\sqrt{1-k r^2}}}{k = 0} = r_1 =
\frac{1}{a_0} \int_0^z\frac{\dd\tilde{z}}{\rfun{\hH}{\tilde{z}}}.
\end{equation}
One has
\begin{empheq}[box=\fbox]{equation}
\rfun{d_\text{L}}{z} = \rbr{1+z}\int_0^z 
\frac{\dd\tilde{z}}{\rfun{\hH}{\tilde{z}}},
\label{eq:p465a-useful}
\end{empheq}
which is often useful. For $k \neq 0$,
\begin{equation}
\frac{1}{\sqrt{k}} \rfun{\arcsin}{r_1\sqrt{k}} = 
\frac{1}{a_0} = \int_0^z \frac{\dd\tilde{z}}{\rfun{\hH}{\tilde{z}}},
\end{equation}
which is more complicated.

We had
\begin{equation}
 \Omega + \Omega_\Lambda = 1+\frac{k}{a^2\hH^2}.
\tag{\ref{eq:p445(4):} revisited}
\end{equation}
For $\Omega_\Lambda = 0$ (and evaluated `today'):
\begin{equation}
\frac{k}{a_0^2} = \rbr{\Omega_0 - 1}\hH_0^2,
\end{equation}
where $p = 0$, $\Omega_0 = 2\dq_0$. Integral of \cref{eq:p465star} gives
\begin{equation}
r_1 = \frac{z\dq_0 + \rbr{q_0-1}\rbr{\sqrt{2q_0 z + 1}-1}}{\hH_0 a_0 \dq_0^2 
\rbr{1+z}},
\end{equation}
which is valid for all $k$. So that
\begin{equation}
d_\text{L} = r_1 a_0 \rbr{1+z} = \rbr{\hH_0 \dq_0^2}^{-1}\sbr{z\dq_0 + 
\rbr{q_0-1}\rbr{\sqrt{2\dq_0 z +1}-1}}.
\end{equation}

\paragraph{Remark}
\cite{Mattig1957} was the first to derive such formulae. Expansion for small 
$z$ gives our old result,
\begin{equation}
d_\text{L} = \hH_0^{-1}\sbr{z + \frac{1}{2}\rbr{1-\dq_0}z^2 + 
\rfun{\Omicron}{z^3} }.
\tag{\ref{eq:eq-in-455} revisited}
\end{equation}
For $k = 0$ ($q = 1/2$):
\begin{equation}
d_\text{L} = \frac{2\rbr{z+1}}{\hH_0 q_0^2}\rbr{1-\frac{1}{\sqrt{1+z}}}.
\end{equation}
Generally, one has for $k = 0$
\begin{equation}
\rfun{d_\text{L}}{z} = \rbr{1+z}\int_0^z 
\frac{\dd\tilde{z}}{\rfun{\hH}{\tilde{z}}}.
\tag{\ref{eq:p465a-useful} revisited}
\end{equation}

\paragraph{Remark}
Yields also exact relation between red-shift, apparent and absolute magnitude:
\begin{equation}
m-M = \num{25} + 5\ln\frac{\lc\,\sbr{\si{\kilo\metre\per\second}}}%
{\hH_0\,\sbr{\si{\kilo\metre\per\second\per\mega\parsec}}} + 5\rfun{\ln}%
{\dq_0^{-2}\sbr{z\dq_0 + \rbr{\dq_0-1}\rbr{\sqrt{2\dq_0 z+1}-1}}}.
\end{equation}


\subsection{Horizons}
Study of \emph{causal structure}: `conformal time' is useful.
\begin{equation}
\dd s^2 = \rfun{a^2}{\eta} 
\rbr{-\dd\eta^2+\dd\chi^2+\rfun{f^2}{\chi}\,\dd\Omega^2}.
\tag{\ref{eq:conformal-time} revisited}
\end{equation}
One can generally show: Robinson--Walker space-times are \emph{conformal} to 
\emph{part} of the \emph{Einstein cylinder} (and are thus conformally flat), 
see for instance \cite[sec.\ 5.3]{Hawking1973}.

\paragraph{Einstein cylinder} for $\theta, \phi = \text{const.}$, embedding of 
cylinder $x^2+y^2 = 1$ into three-dimensional Minkowski space $\dd s^2 = -\dd 
t^2 + \dd x^2 + \dd y^2$.

From $x^2 +  y^2 = 1$, one has $y\,\dd y = -x\,\dd x$, so that
\begin{equation}
\dd y = -\frac{x\,\dd x}{\sqrt{1-x^2}}.
\end{equation}
One has
\begin{equation}
\dd s^2 = -\dd t^2 + \dd x^2 + \frac{x^2\,\dd x^2}{1-x^2} = -\dd t^2 +
\frac{\dd x^2}{1-x^2} = -\dd t^2 + \dd \chi^2,
\end{equation}
where in the last step $x = \sin\chi$ has been used. The last form corresponds 
to $-\dd \eta^2 + \dd\chi^2$ from \cref{eq:conformal-time} above, which is 
essentially `flat', but the whole four-dimensional space-time is not.

$k = +1$: \textbf{diagram} $\eta = 0$: `big bang'; $\eta = 2\pp$: `big crunch'

$k = 0$, $k = -1$: conformally related to a more complicated part of the 
Einstein cylinder (see \cite{Hawking1973})

\paragraph{Penrose diagram related to $k = +1$ ($p = 0 = \Lambda$)}

\textbf{diagram} (world-line of a co-moving observer)

$\chi = 0$ and $\chi = \pp$ are coordinate singularities; $\mscrI^\pm$ are 
\emph{space-like}!

$k = 0$, $k = -1$: \textbf{diagram} \textbf{diagram at p 468} $\mscrI^+$: 
light-like

$k = +1$: we recognise from this Penrose diagram the presence of a `particle 
horizon' (perhaps better: world-line horizon), which limits the world-lines 
that can be seen `today' or at any other time. \textbf{diagram}

Different from this: `event horizon', which limits the world-lines that cannot 
be seen at all, in the whole evolution of the universe.

\textbf{diagram} `future event horizon' this observer can never see events in 
the shaded region

\textbf{diagram} `past event horizon' this observer can never reach events in 
the shaded region

Quantitative details for `particle horizons': \textbf{diagram}

$\rfun{r_\mscrh}{t}$: largest co-moving coordinate radius visible at $t$

$r_1$: event happening at $t_1$ are at $t$ visible if they are at $r \le r_1$.

$\dd s^2 = 0$ or $\dd \Omega^2 = 0$:
\begin{equation}
\int_0^{r_1} \frac{\dd r}{\sqrt{1-kr^2}} = \int_{t_1}^t \frac{\dd\tilde{t}}%
{\rfun{a}{\tilde{t}}}.
\label{eq:int-event-horizon}
\end{equation}

$t_1 \to 0$, $r_1 \to r_\mscrh$:
\begin{equation}
\int_0^{\rfun{r_\mscrh}{t}} \frac{\dd r}{\sqrt{1-kr^2}} = \int_0^t 
\frac{\dd\tilde{t}}{\rfun{a}{\tilde{t}}},
\end{equation}
where $r_\mscrh$ is finite if the last integral converges.

Physical length of particle horizon at time $t$:
\begin{equation}
\rfun{d_\mscrh}{t} = \rfun{a}{t}\int_0^{\rfun{r_\mscrh}{t}} \frac{\dd 
r}{\sqrt{1-kr^2}} = \rfun{a}{t}\int_0^t \frac{\dd 
\tilde{t}}{\rfun{a}{\tilde{t}}},
\end{equation}
which comes from Robertson--Walker line element. $\rfun{d_\mscrh}{t}$ is 
`radius' or distance to horizon, not a `diameter', which contains also other 
direction.

For Einstein--de Sitter Universe, $k = 0$, $p = 0$:
\begin{equation}
\rfun{a}{t} = a_0 \rbr{\frac{3}{2}\hH_0 t}^{2/3}\quad\Rightarrow\quad
\rfun{d_\mscrh}{t} = 3t.
\label{eq:p471above}
\end{equation}
Evaluated `today':
\begin{align}
\rfun{d_\mscrh}{t_0} &= 3t_0 = 3\lc t_0 = 3\cdot\frac{2}{3}\hH_0^{-1} = 
2\hH_0^{-1} \nonumber \\
\approx 2\times \SI{14.6}{Gyr}\cdot\lc \approx \SI{9}{\giga\parsec},
\end{align}
where in the second line $\hh \approx \num{.67}$ has been used, and the result 
is the radius of visible universe today.

\paragraph{Remark} for our real Universe, we find (because of $\Lambda > 0$)
\begin{equation}
\rfun{d_\mscrh}{t_0} \approx \SI{14}{\giga\parsec},
\end{equation}
see \cref{ssec:12.9}.

\begin{align}
\rfun{d_\mscrh}{t_0}
&= \rfun{a}{t} \int_0^{\rfun{a}{t}} \frac{\dd t}{a\dot{a}}
\label{eq:p472line1} \\
&= \rfun{\rfun{a}{t}}{a_0} \int_0^{\frac{\rfun{a}{t}}{a_0}}
\frac{\dd y}{y\dot{y}} \label{eq:p472line2} \\
&= \rfun{\rfun{a}{t}}{a_0\hH_0} \int_0^{\frac{\rfun{a}{t}}{a_0}}
\frac{\dd y}{y\sqrt{1-\Omega_0+\Omega_0/y}}, \label{eq:p472line3} \\
\end{align}
where in \cref{eq:p472line1} $\dd t = \dd a/\dot{a}$, in \cref{eq:p472line2} $y 
= a/a_0$, and in \cref{eq:p472line3} $\dot{y}^2 = \hH_0^2 \rbr{1 - \Omega_0 + 
\Omega_0/y}$ have been used.

Using $a_0/\rfun{a}{t} = 1+z$, one finds for $t \le t_0$
\begin{equation}
\rfun{d_\mscrh}{t} =
\begin{cases}
\frac{2}{\hH_0\rbr{1+z}^{3/2}} \propto \rfun{a^{3/2}}{t} & k = 0 \\
\frac{1}{\hH_0\rbr{1+z}\sqrt{\Omega_0-1}}\sfun{\arccos}{1- 
\frac{2\rbr{\Omega_0-1}}{\Omega_0\rbr{1+z}}} & k = -1 \\
\frac{1}{\hH_0\rbr{1+z}\sqrt{1-\Omega_0}}\sfun{\arccosh}{1+ 
\frac{2\rbr{1-\Omega_0}}{\Omega_0\rbr{1+z}}} & k = +1.
\end{cases}
\end{equation}
One also notes that $\rfun{d_\mscrh}{t; k = 0} = \frac{2}{\hH_0} 
\rbr{\frac{a}{a_0}}^{3/2} = 3t$ as in \cref{eq:p471above}; furthermore, 
$\rfun{d_\mscrh}{t; k = +1} < \rfun{d_\mscrh}{t; k = 0} < \rfun{d_\mscrh}{t; k 
= -1}$, which can be seen from the integral.

For $\rfun{a}{t} \ll a_0$ so that $y \ll 1$: $\Omega_0/y$ dominates, so the 
behaviour is in all cases like the $k=0$ case:
\begin{align}
\rfun{d_\mscrh}{t} &\approx \rfun{\rfun{a}{t}}{a_0\hH_0} 
\int_0^{\frac{\rfun{a}{t}}{a_0}}\frac{\dd y}{\sqrt{y\Omega_0}} =
2\hH_0^{-1}\Omega_0^{-1/2}\rbr{\frac{a}{a_0}}^{3/2} \nonumber \\
&= 2\hH_0^{-1}\Omega_0^{-1/2}\rbr{1+z}^{-3/2} \approx 3t,
\end{align}\
where the last expression is exact only for $k = 0$.

For $\Omega \le 1$ ($k = 0, -1$): $\rfun{d_\mscrh}{t}$ increases \emph{faster} 
than $\rfun{a}{t}$, and $\rfun{a}{t} \to +\infty$ as $t \to +\infty$; thus, 
every particle will be visible at some time (no future event horizon), which is 
clear from \textbf{diagram at p 468}.

For $\Omega > 1$ ($k = 1$): $S^3$, so has a circumference of
\begin{equation}
\rfun{L}{t} = \rfun{a}{t} \int_0^{2\pp}\dd\phi = 2\pp \rfun{a}{t},
\end{equation}
where $\chi = \pp/2 = \theta$ (equator) has been used.

Question: what is $\rfun{d_\mscrh}{t}/\rfun{L}{t}$, i.e.\ which part of 
$\rfun{L}{t}$ is observable at $t$?

With the above expression for $\rfun{d_\mscrh}{t}$ one finds
\begin{equation}
\frac{\rfun{d_\mscrh}{t}}{\rfun{L}{t}} = \frac{1}{2\pp\rfun{a}{t}\hH_0\rbr{1+z} 
\sqrt{\Omega_0-1}}\sfun{\arccos}{1-\frac{2\rbr{\Omega_0-1}}{\Omega_0\rbr{1+z}}}.
\end{equation}
For k = 1:
\begin{equation}
a_0^{-2} = \hH_0^2\rbr{\Omega_0-1}\quad\Rightarrow\quad
a_0\hH_0\sqrt{\Omega_0-1} = 1,
\end{equation}
so that
\begin{equation}
\frac{\rfun{d_\mscrh}{t}}{\rfun{L}{t}} = \frac{1}{2\pp} 
\sfun{\arccos}{1-\frac{2\rbr{\Omega_0-1}\rfun{a}{t}}{\Omega_0 a_0}},
\end{equation}
where the constant $1$ in the parameter of $\arccos$ takes the second branch of 
the inverse function, see \textbf{diagram}.

$a = 0$: $\rfun{d_\mscrh}{t}/\rfun{L}{t} = 0$.

\begin{equation}
a = a_\text{max} = a_0 \Omega_0/\rbr{\Omega_0 - 1} \quad\Rightarrow\quad
\frac{\rfun{d_\mscrh}{\rfun{t}{a_\text{max}}}}{\rfun{L}{\rfun{t}{a_\text{max}}}} 
= \frac{1}{2\pp} \rfun{\arccos}{-1} = \frac{1}{2}.
\end{equation}
We can look to the `antipode': \textbf{diagram}
Visible from both directions, can see `everything'

After that, $d_\mscrh/L$ increases monotonously (second branch of $\arccos$) 
and reaches
\begin{equation}
\frac{d_\mscrh}{L} = \frac{1}{2\pp}\fat{\arccos 1}{\text{second branch}} = 1.
\end{equation}
In this limit, on can `see oneself from the back'.

\paragraph{example} $\Omega_0 = 2$, $\hh_0 \approx \num{.67}$ so that 
$\hH_0^{-1} \approx \SI{14.6}{Gyr}$.
\begin{equation}
a_0^2 = \frac{1}{\hH_0\rbr{\Omega_0-1}} = \hH_0^{-2}\text{ for }\Omega_0 = 2.
\end{equation}
One also has
\begin{equation}
L_0 = 2\pp a_0 \approx \SI{91.7e9}{Lyr} \approx \SI{28.1}{\giga\parsec}.
\end{equation}
Since
\begin{equation}
\frac{\rfun{d_\mscrh}{t_0}}{L_0} = 
\frac{1}{2\pp}\rfun{\arccos}{1-\frac{a_0}{a_0}} = \frac{1}{4},
\end{equation}
where $\Omega_0 = 2$, so that
\begin{equation}
\rfun{d_\mscrh}{t_0} \approx \SI{22.9e9}{Lyr} \approx \SI{7.0}{\giga\parsec}
\end{equation}

\paragraph{Event horizon}
recall \textbf{diagram}
\begin{equation}
\int_0^{r_1} \frac{\dd r}{\sqrt{1-kr^2}} = \int_{t_1}^t \frac{\dd\tilde{t}}%
{\rfun{a}{\tilde{t}}}.
\tag{\ref{eq:int-event-horizon} rivisited}
\end{equation}
If the right hand side diverges for $t \to +\infty$ (or $t \to t_\text{e}$ for 
big crunch), there is no event horizon. If it converges,
\begin{equation}
\int_0^{r_1} \frac{\dd r}{\sqrt{1-kr^2}} \le \int_{t_1}^{t_\text{max}} 
\frac{\dd\tilde{t}}{\rfun{a}{\tilde{t}}},
\end{equation}
where $t_\text{max}$ corresponds to $+\infty$ or $t_\text{e}$.

$\Omega_0 \le 1$: $\rfun{a}{t} \propto t^{2/3}$ for $\Omega_0 = 1$ or 
$\rfun{a}{t} \sim t$ for $\Omega_0 < 1$ and $t \to +\infty$. $t$-integral 
diverges and there is no (future) event horizon.

$\Omega_0 > 1$:
\begin{align}
\rfun{d_{\text{e}\mscrh}}{t_1} &= \rfun{a}{t_1}\int_{t_1}^{t_\text{max}}
\frac{\dd\tilde{t}}{\rfun{a}{\tilde{t}}} \nonumber \\
&= \frac{\rfun{a}{t_1}}{a_0\hH_0\sqrt{\Omega_0-1}}\sbr{2\pp
-\sfun{\arccos}{1-\frac{2\rbr{\Omega_0-1}\rfun{a}{t_1}}{\Omega_0 a_0}}}.
\end{align}

\textbf{diagram} $\rfun{d_{\text{e}\mscrh}}{t_1}$: events at $t_1$ that become 
visible before the big crunch.

\paragraph{Example} $\Omega_0 = 2$, $\hh \approx \num{.67}$, thus $\hH_0^{-1} 
\approx \SI{14.6}{Gyr}$, and
\begin{equation}
\frac{\rfun{d_{\text{e}\mscrh}}{t_0}} = \hH_0^{-1}\rbr{2\pp-\arccos 0} = 
\frac{3\pp}{2}\hH_0^{-1} \approx \SI{68.8e9}{Lyr}\approx\SI{21.1}{\giga\parsec},
\end{equation}
where $\rfun{d_{\text{e}\mscrh}}{t_0}$: events happening `today' that 
become visible before the big crunch. Recall above, $L_0 \approx 
\SI{28.1}{\giga\parsec}$ and $\frac{\rfun{d_{\mscrh}}{t_0}} \approx 
\SI{7.0}{\giga\parsec}$. But one can see everything by looking in `the other 
direction'.


\subsection{Cosmological constant and dark energy}
\label{ssec:12.9}

We had
\begin{equation}
\Omega + \Omega_\Lambda + \Omega_k = 1,
\tag{\ref{eq:FL(4)} revisited}
\end{equation}
where $\Omega = \Omega_\text{m} + \Omega_\text{r}$, $\Omega + \Omega_\Lambda 
\eqqcolon \Omega_\text{tot}$, and $\Omega_k = -k/\rbr{a\hH}^2$; respectively,
\begin{equation}
\dot{a}^2 = \frac{8\pp\nG}{3}a^2\rbr{\rho+\rho_\Lambda} - k,
\tag{\ref{eq:FL(2)} revisited}
\end{equation}
where $\rho_\Lambda = \Lambda/8\pp\nG$, and $\rho \coloneqq 
\rho_{\text{m},0}\rbr{a_0/a}^3 + \rho_{\text{r},0}\rbr{a_0/a}^4$.

Division by $a_0^2$ leads to
\begin{equation}
\rbr{\frac{\dot{a}}{a_0}}^2 = \frac{8\pp\nG}{3} \rbr{\rho_{\text{m},0} 
\frac{a_0}{a} + \rho_{\text{r},0} \rbr{\frac{a_0}{a}}^2 + \rho_\Lambda 
\rbr{\frac{a}{a_0}}^2} - \frac{k}{a_0^2}.
\label{eq:division-by-a02}
\end{equation}
Introducing Dimensionless variables
\begin{equation}
\tau \coloneqq \hH_0 t,\quad x \coloneqq a/a_0,\quad \Omega_{\text{m},0}
\coloneqq \rho_{\text{m},0}/\rho_{\text{c},0},\qquad\text{etc.},
\end{equation}
one has
\begin{equation}
\rbr{\frde{x}{\tau}}^2 = \frac{\Omega_{\text{m},0}}{x} + 
\frac{\Omega_{\text{r},0}}{x^2} + \Omega_{\Lambda,0} x^2 - 
\frac{k}{a_0^2\hH_0^2},
\end{equation}
where
\begin{equation}
 - \frac{k}{a_0^2\hH_0^2} = 1 - \Omega_{\text{m},0} - \Omega_{\text{r},0} - 
\Omega_{\Lambda,0}
\end{equation}
by \cref{eq:FL(4)}. Then,
\begin{empheq}[box=\fbox]{equation}
\rbr{\frde{x}{\tau}}^2 = 1 + \Omega_{\text{m},0}\rbr{\frac{1}{x}-1} + 
\Omega_{\text{r},0}\rbr{\frac{1}{x^2}-1} + \Omega_{\Lambda,0} \rbr{x^2-1},
\label{eq:p479the}
\end{empheq}
in which $\Omega_{\text{r},0} \approx \num{8e-5}$, so we neglect this in the 
following (relevant only for the early Universe); then, $\Omega_{\text{m},0} = 
\Omega_0$. One has
\begin{equation}
\rbr{\frde{x}{\tau}}^2 + \rfun{V_\text{eff}}{x} = 1,\qquad
\rfun{V_\text{eff}}{x} = \Omega_0\rbr{1-\frac{1}{x}} + \Omega_{\Lambda, 0}
\rbr{1-x^2},
\end{equation}
where the right hand side of the first equation can be interpreted as an 
`energy', and $\Omega_{\Lambda, 0} = \Lambda/3\hH_0^2$.

Various cases:

a) $\Omega_{\Lambda, 0} < 0$ ($\Lambda < 0$)

\textbf{diagram} no extremal value; turning point at $x^3 = 
-\Omega_0/\Omega_{\Lambda, 0}$.

Always a re-collapsing universe (independent of $k$)

symmetry around maximum, because only $\rbr{\dd x/\dd \tau}^2$ occurs

b) $\Omega_{\Lambda, 0} > 0$ ($\Lambda > 0$)

Extremum of $\rfun{V_\text{eff}}{x}$ is at $x = \rbr{\Omega_0 / 
2\Omega_{\Lambda,0}}^{1/3}$; distinguishes between b1) and b2):

b1) Extremum (maximum) of $\rfun{V_\text{eff}}{x}$ is less than one:

\textbf{diagram}

universe expands forever (independent of $k$) [describes our Universe after the 
radiation phase]

for $k=0$, which is indicated by observations, there is a simple analytical 
solution
\begin{align}
\rfun{a}{t} &= a_0\rbr{\frac{\Omega_0}{\Omega_{\Lambda, 0}}}^{1/3} 
\sinh^{2/3}\frac{3\sqrt{\Omega_{\Lambda,0}}\hH_0 t}{2} \nonumber \\
&= a_0\rbr{\frac{3\Omega_0 \hH_0^2}{\Lambda}}^{1/3} 
\rfun{\sinh^{2/3}}{\frac{3}{2}\sqrt{\frac{\Lambda}{3}}t}
\sim \ee^{\sqrt{\Lambda/3}t},
\end{align}
which approaches de Sitter space; see \cref{ssec:dS-AdS}.

\paragraph{Remark} setting
\begin{equation}
\frac{a}{a_0} = \frac{1}{1+z} = \rbr{\frac{\Omega_0}{\Omega_{\Lambda,0}}}^{1/3}
\sinh^{2/3}\frac{2\sqrt{\Omega_{\Lambda,0}}\hH_0 t}{2},
\end{equation}
one finds
\begin{equation}
\hH_0 t = \frac{2}{3}\Omega^{-1/2}_{\Lambda,0}\rfun{\arcsinh}{\rbr{1+z}^{-3/2}
\rbr{\frac{\Omega_{\Lambda,0}}{\Omega_0}}^{1/2}}.
\end{equation}

b2) Extremum (maximum) of $\rfun{V_\text{eff}}{x}$ is greater than one:

\textbf{diagrams}

singularity-free solution

Interesting limiting case:

`Eddington--Lemaître' solution; static solution, see below 
\cite{einstein1917kosmologische}

Maximum `slightly' below $\Lambda$: `quasi-static Lemaître model'

\textbf{diagram}

historically used to explain a peak in the quasar abundance at $z \approx 
\num{2.5}$, today explained by processes in galactic nuclei.

\paragraph{Static Einstein universe} \cite{einstein1917kosmologische}

Motivated by Mach's Principle (matter determines geometry uniquely)

Staticity: $\dot{a} = 0 = \ddot{a}$, only matter: $p = 0$.
\begin{equation}
\ddot{a} = -\frac{4\pp\nG}{3}\rbr{\rho + 3p} a + \frac{\Lambda a}{3} 
\overset{!}{=} 0,
\end{equation}
so that $\Lambda = 4\pp\nG\rho$, $\rho_\Lambda = \rho/2$;
\begin{equation}
\dot{a}^2 = \frac{8\pp\nG}{3}\rho a^2 + \frac{\Lambda a^2}{3} - 1 
\overset{!}{=} 0,
\end{equation}
where $k = 1$ (otherwise no solution), thus
\begin{equation}
a_\text{s}^2\rbr{\frac{8\pp\nG}{3}\rho + \frac{\Lambda}{3}} = 1,
\end{equation}
where $\Lambda = 4\pp\nG\rho$, and $a_\text{s}$ is the radius of static 
Einstein universe, therefore
\begin{equation}
a_\text{s}^2 = \frac{1}{4\pp\nG\rho} = \frac{1}{\Lambda},
\end{equation}
where the balance between matter and $\Lambda$-repulsion is exact.

Total mass:
\begin{equation}
M = \rho V = \frac{1}{4\pp\nG a_\text{s}^2}2\pp^2 a_\text{s}^3 = 
\frac{\pp}{2\nG} a_\text{s},
\end{equation}
so that
\begin{equation}
a_\text{s} = \frac{2\nG M}{\pp},
\end{equation}
where $2\nG M = \rSch$.

Small $M$ corresponds to small universe; no empty universe would exist, in the 
spirit of Mach's Principle. But: model is \emph{unstable} \cite{Eddington1930}

Einstein gave $\Lambda$ definitely up (had doubts already in 1923; cf.\ 
\cite{nussbaumer2014})

\emph{Age} of the universe: recall the formula
\begin{equation}
\rbr{\frde{x}{\tau}}^2 = 1 + \Omega_{\text{m},0}\rbr{\frac{1}{x}-1} + 
\Omega_{\text{r},0}\rbr{\frac{1}{x^2}-1} + \Omega_{\Lambda,0} \rbr{x^2-1},
\tag{\ref{eq:p479the} revisited}
\end{equation}
where $x = a/a_0$, $\tau = \hH_0 t$. One finds that
\begin{equation}
\tau = \int_0^1\dd y\,\sbr{1 + \Omega_{\text{m},0}\rbr{\frac{1}{y}-1}+
\Omega_{\text{r},0}\rbr{\frac{1}{y^2}-1}+\Omega_{\Lambda,0}\rbr{y^2-1}}^{-1/2},
\end{equation}
where $\Omega_{\text{r}{0}} \approx \num{8e-5}$, thus has little influence on 
the age. One has
\begin{equation}
\rfun{\tau}{\Omega_{\text{m},0},\Omega_{\text{r},0},\Omega_{\Lambda,0}}\approx
\rfun{\tau}{\Omega_{\text{m},0},\Omega_{\Lambda,0}}.
\end{equation}
The various possibilities of models can be depicted (after some algebra) in the 
following diagram (which plays a central role)

\textbf{diagram}

$k = 0$: $\Omega_{\text{m},0} + \Omega_{\Lambda,0} = 1$.

Observations should determine the `point' that corresponds to our real Universe.

\paragraph{Particle horizon for general $\Lambda$?}

neglect curvature, i.e.\ $k = 0$,
\begin{equation}
\rbr{\frde{x}{\tau}}^2 = \frac{\Omega_{\text{m},0}}{x} + 
\frac{\Omega_{\text{r},0}}{x^2} + \Omega_{\Lambda,0} x^2,
\end{equation}
thus
\begin{align}
\rfun{d_\mscrh}{t} &= \rfun{a}{t}\int_0^t \frac{\dd\tilde{t}}%
{\rfun{a}{\tilde{t}}} = \frac{\rfun{a}{t}}{a_0}\int_0^{\rfun{a}{t}/a_0}
\rfun{\dd y}{y \dot{y}} \nonumber \\
&= \frac{1}{\hH_0\cdot\rbr{1+z}}\int_0^{\rbr{1+z}^{-1}}\frac{\dd y}%
{y\sqrt{\frac{\Omega_{\text{m},0}}{y}+\frac{\Omega_{\text{r},0}}{y^2} + 
\Omega_{\Lambda,0} y^2}},
\end{align}
where $a_0/\rfun{a}{t} = 1+z$ has been used in the second line.

`today': $z = 0$,
\begin{equation}
\rfun{d_\mscrh}{t_0} = \frac{1}{\hH_0}\int_0^1
\frac{\dd y}{y\sqrt{\frac{\Omega_{\text{m},0}}{y} + 
\frac{\Omega_{\text{r},0}}{y^2} + \Omega_{\Lambda,0} y^2}}.
\end{equation}
Present values: $\hH_0 \approx \SI{67}{\kilo\metre\per\second\per\mega\parsec}$ 
or $\hh_0 \approx \num{.67}$, $\Omega_{\Lambda,0} \approx \num{.68}$, 
$\Omega_{\text{m},0} \approx \num{.32}$, $\Omega_{\text{r},0} \approx 
\num{8e-5}$, leading to $\rfun{d_\mscrh}{t_0} \approx \SI{14}{\giga\parsec} 
\approx \SI{47e9}{Lyr} > \lc\cdot\SI{13.8e9}{yr}$, where $\SI{13.8e9}{yr}$ is 
the age of Universe. $\rfun{d_\mscrh}{t_0}$ is the distance in today's universe 
up to which we can see objects (i.e.\ photons have had enough time since the 
big bang to reach us)

\subsection{de Sitter and anti-de Sitter space}
\label{ssec:dS-AdS}

Exact vacuum solution of Einstein equation for $\Lambda > 0$: de Sitter 
solution (\emph{de Sitter} or \emph{dS space}); $\Lambda < 0$: \emph{Anti-de 
Sitter} or \emph{AdS space}. Willem de Sitter (\textborn 1982, \textdied 1934); 
solution in \cite{de1917relativity,de1917curvature,Levi-Civia1917}.

\begin{equation}
R_{\mu\nu} - \frac{1}{2} g_{\mu\nu} R + \Lambda g_{\mu\nu} = 8\pp\nG T_{\mu\nu},
\label{eq:einstein's-way}
\end{equation}
which is Einstein's way of writing, and equivalent to
\begin{equation}
R_{\mu\nu} - \frac{1}{2} g_{\mu\nu} R = 8\pp\nG\rbr{T_{\mu\nu} - 
\rho_\Lambda g_{\mu\nu}},
\end{equation}
where $\rho_\lambda \coloneqq \Lambda/8\pp\nG$, which is the modern way of 
writing,
\begin{equation}
T_{\mu\nu, \Lambda} \equiv T_{\mu\nu}^\text{vac} = -\frac{\Lambda}{8\pp\nG} 
g_{\mu\nu} \equiv -\rho_\Lambda g_{\mu\nu}.
\end{equation}

Taking trace of \cref{eq:einstein's-way} gives
\begin{equation}
R - 2R + 4\Lambda = 8\pp\nG T \quad\Rightarrow\quad R = 4\Lambda-8\pp\nG T.
\label{eq:p486this}
\end{equation}
Inserting \cref{eq:p486this} back to \cref{eq:einstein's-way} gives
\begin{equation}
R_{\mu\nu} = \Lambda g_{\mu\nu} + 8\pp\nG\rbr{T_{\mu\nu} - \frac{1}{2} 
g_{\mu\nu} T}.
\end{equation}
For vacuum,
\begin{empheq}[box=\fbox]{equation}
R_{\mu\nu} = \Lambda g_{\mu\nu} \quad\Rightarrow\quad R=4\Lambda.
\end{empheq}
Solutions to these equations are called \emph{Einstein spaces}. Among the 
solutions, there are solutions where $R_{\mu\nu\lambda\rho}$ is not of the form 
$\propto\rbr{g_{\mu\lambda} g_{\nu\rho} - g_{\mu\rho} g_{\nu\lambda}}$, but 
here we consider only spaces of constant curvature.

Space-times with \emph{maximal symmetry} ($10$ Killing vectors) are spaces with 
constant curvature: $\Lambda = 0$ Minkowski, $\Lambda > 0$ de Sitter, and 
$\Lambda < 0$ Anti-de Sitter.

\paragraph{de Sitter space} because of symmetry, can be written in many simple 
coordinate systems. First possibility or \emph{static form}:
\begin{equation}
\dd s^2 = -\rbr{1-\frac{\Lambda r^2}{3}}\,\dd t^2 + \rbr{1-\frac{\Lambda 
r^2}{3}}^{-1}\,\dd r^2 + r^2\,\dd\Omega^2,
\label{eq:p487star}
\end{equation}
in which $r = 0$ is completely regular. Solution with additional 
spherically-symmetric mass $M$:
\begin{equation}
\dd s^2 = -\rbr{1-\frac{2\nG M}{r} - \frac{\Lambda r^2}{3}}\,\dd t^2 + 
\rbr{1-\frac{2\nG M}{r}-\frac{\Lambda r^2}{3}}^{-1}\,\dd r^2 + r^2\,\dd\Omega^2,
\end{equation}
which is called the \emph{Schwarzschild--de Sitter} or \emph{Kottler solution} 
in \cite{ANDP:ANDP19183611402}, where for small $r$ $\approx$ Schwarzschild, 
while for big $r$ it approaches de Sitter.

\paragraph{remark} including an electric charge, one has
\begin{equation}
\dd s^2 = -\rbr{1-\frac{2\nG M}{r} - \frac{\Lambda r^2}{3} + \frac{\nG 
Q^2}{r^2}}\,\dd t^2 + \ldots
\end{equation}

\paragraph{Back to de Sitter} \Cref{eq:p487star} has singularity at $r = 
\sqrt{3/\Lambda}$, which must be a coordinate singularity, since space has 
constant curvature (``looks everywhere the same'')

$r = \sqrt{3/\Lambda}$ is an `event horizon', in analogy to $\rSch = 2\nG M$, 
but here the static region is `inside' ($r < \sqrt{3/\Lambda}$), while the 
dynamical region ($r$ time-like) is `outside' ($r > \sqrt{3/\Lambda}$).

One can show (exercise) that de Sitter space can be \emph{mapped isometrically} 
onto the following four-dimensional subspace (hyperboloid) of five-dimensional 
Minkowski space with $\dd s^2 = -\dd T^2 + \dd x^2 + \dd y^2 + \dd z^2 + \dd 
w^2$:
\begin{equation}
x^2 + y^2 + z^2 + w^2 - T^2 = \frac{3}{\Lambda}
\end{equation}

Symmetry group is $\rfun{\SO}{1,4}$ with ten parameters: corresponds to Lorentz 
transformation in five dimension, \emph{not} Poincaré transformation, which is 
de Sitter group, isomorphic to $\rfun{\SO}{5}$ in the Euclidean case, which 
leaves $S^4$ invariant.
