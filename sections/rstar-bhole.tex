\section{Relativistic Stars and Black Holes}

\subsection{Elementary theory of degenerate stars}

\label{ssec:elem-th-ds}

\subsubsection*{Hydrostatic equilibrium of \emph{normal stars} (stars on main
sequence) in Newtonian theory}
\begin{figure}
\centering
\includegraphics{graphics/11/stars1}
\caption{Need a title}
\end{figure}
Radial symmetry assumed. Gas dominates for normal stars.

Force difference: $\rfun{p}{r}\dd\, A - \rfun{p}{r+\dd r}\dd\, A =
-p'\dd\, r\dd\, A \equiv -\dd p\dd\, A$

\emph{Equilibrium}: outward pointing pressure force must be compensated by
inward pointing gravitational force.
\begin{align}
-\dd p\dd\, A &\eeq \rfun{g}{r}\dd\, M \nonumber \\
&= \frac{\nG M\rbr{r}}{r^2} \rho\rbr{r}\,\underbrace{\dd A\dd\, r}_{\dd%
\text{volume}}
\end{align}
\begin{empheq}[box=\fbox]{equation}
\hookrightarrow\quad
\frde{\rfun{p}{r}}{r} = -\frac{\nG M\rbr{r}}{r^2}\rho\rbr{r} =
-\rfun{g}{r}\rho\rbr{r}
\end{empheq}
So called \emph{hydrostatic equilibrium}:
\begin{equation}
M\rbr{r}\coloneqq 4\pp\int_0^r \dd\tld{r}\,\tld{r}^2\rfun{\rho}{\tld{r}}
\label{eq:stellare-mass}
\end{equation}
pressure decreases with increasing $r$.
\begin{figure}
\centering
\includegraphics{graphics/11/stars2}
\caption{Also need a title}
\end{figure}

\subsection*{Heuristic consideration}
Let $\rho = \text{const}$:
\begin{align}
\frde{\rfun{p}{r}}{r} &= -\frac{\nG M\rbr{r}}{r^2}\rho \nonumber \\
&= -\frac{4\pp}{3}\nG \rho^2 r \quad
\text{(Inserting }M\rbr{r} = \frac{4\pp}{3}\rho r^3\text{)}
\end{align}
so that
\begin{equation}
\rfun{p}{r} = \rfun{p}{0} - \frac{2\pp}{3}\nG \rho^2 r^2
\end{equation}
where $\rfun{p}{0} \equiv p_0$ is the pressure at stellar centre.

Vanishing pressure at stellar surface:
\begin{equation}
\rfun{p}{R} \eeq 0 \qquad \hookrightarrow \quad
p_0 = \frac{2\pp}{3}\nG \rho^2 R^2
\end{equation}

Dimensions:
\begin{align}
\sbr{p} &= \frac{\text{force}}{\text{area}} = \mathsfup{\frac{M}{LT^2}}, \\
\sbr{\rho} &= \frac{\text{mass}}{\rbr{\text{length}}^3} =
\frac{\sbr{p}}{\mathsfup{L^2/T^2}}
\end{align}
\begin{equation}
\hookrightarrow\quad \frac{p_0}{\rho\lc^2} = \frac{2\pp}{3}
\frac{\nG \rho R^2}{\lc^2} = \frac{\rSch}{4 R}
\label{eq:heu-estimate}
\end{equation}
where $\rho\lc^2$ is dimensionless, $\rSch = 2\nG M/\lc^2$ Schwarzschild
radius, $\rho = M / \rbr{\frac{4\pp}{3}R^3}$.

Order of magnitude estimate:
\begin{empheq}[box=\fbox]{equation}
\frac{p}{\rho\lc^2}\sim\frac{\nG M}{R\lc^2},
\end{empheq}
where $p$ is the average pressure. The right-hand side is important in General
Relativity, which is the ratio determined by ratio of pressure to energy
density. Higher pressure leads to more relativistic.

For the following need the \emph{equation of state}
\begin{equation}
\frac{p}{\rho\lc^2} \eqqcolon f\rbr{\rho, T}.
\end{equation}

\paragraph{\emph{normal stars}}
Ideal gas
\begin{equation}
pV = N\bk T = n\gR T = n\bk\aN T
\end{equation}
where $N$ is the number of molecules, $n$ chemical amount of substance (mole
in SI), $\gR$ gas constant, $\bk$ Boltzmann constant, $\aN$ Avogadro number,
with $N = n\aN$.

Now the equation of state reads
\begin{equation}
\rfun{f}{\rho, T} = \frac{p}{\rho\lc^2} = \frac{N}{\rho V}\frac{\bk T}{\lc^2}
=\frac{\bk T}{m\lc^2},
\label{eq:eos-star}
\end{equation}
where $m\coloneqq \rho V / N$ molecule mass; later proton mass.

\eqref{eq:eos-star} leads to
\begin{equation}
\frac{\nG M}{R \lc^2} \sim f = \frac{\bk T}{m \lc^2} = \sigma
\rbr{\si{\kilo\electronvolt/\giga\electronvolt}}\sim\num{e-6},
\end{equation}
where $\si{\kilo\electronvolt}$ nuclear reaction (interior),
$\si{\giga\electronvolt}$ rest energy of hydrogen. So that relativistic
effects are of order \num{e-6}! We already know this from section 9
\textbf{not good, to be replaced with \texttt{\\ref}}.

`Smallness' of relativistic effects is due to `nuclear physics', independent
of stellar dimension.

\begin{equation}
\frac{3}{2} \bk T = \frac{1}{2} m \ol{v}^2,
\end{equation}
$\ol{v}$ average molecule velocity.
\begin{equation}
\hookrightarrow \frac{\bk T}{m\lc^2} = \frac{\ol{v}^2}{3\lc^2}.
\end{equation}

\paragraph{Degenerate matter (general)}
Nuclear fuel exhausted leads to collapse of star; for $\rho \gtrsim
\SI{e4}{\gram\per\cubic\centi\metre}$, \emph{degenerate matter}:
\begin{equation}
\frac{p}{\rho\lc^2} \equiv \rfun{f}{\rho},
\end{equation}
which is independent of $T$.

Heuristic derivation:
\begin{equation}
\fp \cdot t \sim \phs
\label{eq:heis-fp}
\end{equation}
where $\fp$
Fermi momentum, and the equation means electrons confined to region with
$V\sim d^3$ by Pauli principle. This leads to
\begin{equation}
\fE \coloneqq \frac{\fp}{2m_{\ce{e}}},
\end{equation}
where $\fE$ (non-relativistic) Fermi energy, and the lightest electron mass
contributes.

Above:
\begin{equation}
f = \frac{p}{\rho\lc^2} \sim \frac{\bk T}{m\lc^2}.
\end{equation}
Replacing $\bk T$ by $\fE$ yields
\begin{align}
f &\sim\frac{1}{m\lc^2}\rbr{\sqrt{\fp^2\lc^2+m_{\ce{e}}^2\lc^4}
-m_{\ce{e}} \lc^2} \\
&\sim \frac{1}{m\lc^2}\cdot
\begin{cases}
\frac{\fp^2}{m_{\ce{e}}},\qquad\text{non-relativistic} \\
\fp\lc,\qquad\text{relativistic},
\end{cases}
\end{align}
where $m$ stands for proton mass, and the relativistic transition at
$\fp \sim m_{\ce{e}}\cdot\lc$.

Then from \eqref{eq:heis-fp}:
\begin{equation}
d \sim \frac{\phs}{m_{\ce{e}}\lc} \eqqcolon\lambda_{\ce{e}} \approx
\SI{4e-11}{\centi\metre},
\end{equation}
where $\lambda_{\ce{e}}$ reduced Compton wavelength. Define density
\begin{equation}
\rho_0\sim\frac{m}{\lambda_{\ce{e}}^3} \approx
\SI{3e7}{\gram\per\cubic\centi\metre}.
\end{equation}
\begin{align}
\fp\sim\frac{\phs}{d},\qquad d\sim\rbr{\frac{m}{\rho}}^{\frac{1}{3}} \\
\hookrightarrow \fp \sim \phs\rbr{\frac{\rho}{m}}^{\frac{1}{3}}
\end{align}
which leads to
\begin{empheq}[box=\fbox]{equation}
f = \frac{p}{\rho\lc^2} \sim \frac{m_{\ce{e}}}{m} \cdot
\begin{cases}
\rbr{\frac{\rho}{\rho_0}}^\frac{2}{3},\qquad\rho\lesssim\rho_0\quad
\text{(non-relativistic)}, \\
\rbr{\frac{\rho}{\rho_0}}^\frac{1}{3},\qquad\rho\gtrsim\rho_0\quad
\text{(relativistic)},
\end{cases}
\label{eq:heu-eos-fermi-gas}
\end{empheq}
which works for $\SI{e4}{\gram\per\cubic\centi\metre}\lesssim\rho\lesssim
\SI{e8}{\gram\per\cubic\centi\metre}$.

\paragraph{Exact treatment (statistical mechanics)}

$T = 0$: electrons occupy all states with energies from $0$ to $\fE$, which is
the maximal energy (determined by the \emph{number} of electrons).

Number of quantum states for electron in $V$ with momentum whose magnitude
lies between $k$ and $k+\dd k$:
\begin{equation}
2\times\frac{4\pp k^2\dd\, k}{\rbr{2\pp\phs}^3}V
= \frac{V}{\pp^2 \phs^3} k^2\dd\, k,
\label{eq:elec-diff-num}
\end{equation}
where $2$ is the degenerate number of spin, $\rbr{2\pp\phs}^3$ is the
phase-space volume. This yields the total number
\begin{equation}
N = \frac{V}{\pp^2\phs^3}\int_0^{\fp} k^2 \dd\, k
= \frac{V\fp^3}{3\pp^2\phs^3},
\end{equation}
where $\fp$ is the maximal momentum, equals to the radius of Fermi ball in
momentum space. Now that
\begin{equation}
\fp = \rbr{3\pp^2}^\frac{1}{3}\rbr{\frac{N}{V}}^\frac{1}{3}\phs
\label{eq:fermi-mom}
\end{equation}
so the (non-relativistic) Fermi energy reads
\begin{equation}
\fE = 
\frac{\fp^2}{2m_{\ce{e}}}=\rbr{3\pp^2}^\frac{2}{3}\frac{\phs^2}{2m_{\ce{e}}}
\rbr{\frac{N}{V}}^\frac{2}{3},
\end{equation}
and the total energy is
\begin{align}
E &= \int_0^{\fp}\frac{k^2}{2m_{\ce{e}}}\rbr{\frac{V}{\pp^2\phs^3}k^2\dd\, k}
\nonumber \\
&= \frac{3\rbr{3\pp^2}^\frac{2}{3}}{10}\frac{\phs^2}{m_{\ce{e}}}
\rbr{\frac{N}{V}}^\frac{2}{3}N
\end{align}
where the bracketed part in the integrand of the first line comes from
\cref{eq:elec-diff-num}, and on the second line \cref{eq:fermi-mom} is
inserted. Further considering that Fermi gas obeys
\begin{equation}
pV = \frac{2}{3}E,
\end{equation}
we have
\begin{equation}
\frac{p}{\rho\lc^2}=\frac{\rbr{3\pp^2}^\frac{2}{3}}{5}\frac{m_{\ce{e}}}{m}
\rbr{\frac{\rho}{\rho_0}}^\frac{2}{3}.
\end{equation}
Recall that $\rho_0 = m/\lambda_{\ce{e}}^3$. Though the numerical factor 
$\rbr{3\pp^2}^{2/3}/5 \approx 1.9$ was missing before in
\cref{eq:heu-eos-fermi-gas}, the heuristic estimation was excellent!

The relativistic case is similar.

\paragraph{Special case: White Dwarfs} We had
\begin{equation}
\frac{\nG M}{R\lc^2} = \frac{p}{\rho \lc^2} \equiv \rfun{f}{\rho}.
\end{equation}
With $R\sim\rbr{M/\rho}^{1/3}$:
\begin{equation}
\frac{\nG M^\frac{2}{3} \rho^\frac{1}{3}}{\lc^2} = f,
\end{equation}
which leads to
\begin{equation}
M = \frac{\lc^3 \rfun{f^\frac{3}{2}}{\rho}}{\rho^\frac{1}{2}\nG^\frac{3}{2}}
\eqqcolon\rfun{M}{\rho},
\end{equation}
so $\rho\leftrightarrow\rfun{M}{\rho}$ for degenerate star in equilibrium.

% begin 20.07.2016

\subparagraph{Non-relativistic case}
\begin{equation}
\rfun{f}{\rho} = \frac{m_{\ce{e}}}{m}\rbr{\frac{\rho}{\rho_0}}^\frac{2}{3}.
\end{equation}
Thus
\begin{align}
\rbr{M}{\rho} &= \frac{\lc^3}{\rho^\frac{1}{2}\nG^\frac{3}{2}}
\frac{\rho}{\rho_0}\rbr{\frac{m_{\ce{e}}}{m}}^\frac{3}{2} \nonumber \\
&\equiv \rbr{\frac{\rho}{\rho_0}}^\frac{1}{2} M_\text{C},\qquad
\rho < \rho_0,
\end{align}
where the \emph{Chandrasekhar mass}
\begin{equation}
M_\text{C} \coloneqq \rbr{M}{\rho_0} =
\frac{\lc^3}{\rho_0^\frac{1}{2}\nG^\frac{3}{2}}
\rbr{\frac{m_{\ce{e}}}{m}}^\frac{3}{2},
\end{equation}
and $\rho_0 = m/\lambda_{\ce{e}}^3$.

\subparagraph{Relativistic limit}
\begin{equation}
\rfun{f}{\rho} = \frac{m_{\ce{e}}}{m}\rbr{\frac{\rho}{\rho_0}}^\frac{1}{3}.
\end{equation}
Thus
\begin{equation}
\rfun{M}{\rho} = M_\text{C},\qquad \rho > \rho_0
\end{equation}
which is achieved for relativistic degeneracy, and no $\rho$-dependence of
$M$!

\emph{White draft}: degeneracy pressure counterbalances gravity.

Using $m/\lambda_{\ce{e}}^3 = m m_{\ce{e}}^3 \lc^3 / \phs^3$, one can write
\begin{empheq}[box=\fbox]{equation}
M_\text{C} = m\rbr{\frac{\phs\lc}{m^2\nG}}^\frac{3}{2} \eqqcolon
m\apG^{-\frac{3}{2}} \approx \num{1.8} M_\astrosun,
\end{empheq}
where
\begin{equation}
\apG \coloneqq \frac{\nG m^2}{\phs\lc}\equiv\rbr{\frac{m}{\plm}}^2
\approx \num{5.91e-38}
\end{equation}
is the \emph{fine structure constant of gravitation}, similar to
the electromagnetic fine structure constant $\apE = \ec^2/\phs\lc$, and
$\plm$ Planck mass. More precise calculation yields $M_\text{C}
\approx 1.44 M_\astrosun$.

$\apG$ central number for size and mass of stars, so that $\phs$ not only
relevant for atoms, but also for stars!

\subparagraph{Number of protons in a White Dwarf}
\begin{equation}
A \sim \frac{M_\text{C}}{m} = \apG^{-\frac{3}{2}} \approx \num{2e54},
\end{equation}
which gives correct order of magnitude for the sun.

General-relativistic effect are of order
\begin{equation}
\frac{\nG M}{R\lc^2} \sim \frac{p}{\rho\lc^2}
\approx \frac{m_{\ce{e}}}{m}\rbr{\frac{\rho}{\rho_0}}^p \sim 
\frac{m_{\ce{e}}}{m}
\sim \num{e-4}
\end{equation}
in which $p = 2/3$ for non-relativistic case, $1/3$ for relativistic case.
So that the general-relativistic effect is determined by the ratio
$m_{\ce{e}}/m$! Above: $\bk T$, here $m_{\ce{e}} \lc^2$.

% end 20.07.2016

% begin 21.07.2016

\subparagraph{Radii of White Dwarfs}
\begin{align}
R &\sim \rbr{\frac{M}{\rho}}^\frac{1}{3} \nonumber \\
&\sim \rbr{\frac{m}{\rho}}^\frac{1}{3} \apG^{-\frac{1}{2}}
\rbr{\frac{\rho}{\rho_0}}^\frac{1}{6} \nonumber \\
&= \lambda_{\ce{e}}\apG^{-\frac{1}{2}}\rbr{\frac{\rho_0}{\rho}}^\frac{1}{6}
\nonumber \\
&\approx \rbr{\frac{\rho_0}{\rho}}^\frac{1}{6}\cdot\SI{5e3}{\kilo\metre}
\end{align}
where on the second line the non-relativistic relation $\rfun{M}{\rho} =
M_\text{C}\rbr{\rho/\rho_0}^{1/2}$, and on the third line $m =
\rho_0\lambda_{\ce{e}}^3$ are inserted. The result is comparable to the size of
Earth!

Because $M = M_\text{C}\rbr{\rho/\rho_0}^{1/2}$, one has
\begin{equation}
MR^3 = M_\text{C} \rbr{\frac{\rho}{\rho_0}}^\frac{1}{2} R_\text{C}^3
\rbr{\frac{\rho_0}{\rho}}^\frac{1}{2},
\end{equation}
or
\begin{empheq}[box=\fbox]{equation}
MR^3 \sim M_\text{C} R_\text{C}^3,
\end{empheq}
in which $R_\text{C} \coloneqq \lambda_{\ce{e}} \apG^{-1/2}$. So that Radii fall
with increasing mass!

\paragraph{Neutron Stars}

$\rho \gtrsim \SI{e8}{\gram\per\cubic\centi\metre}$: gradual transition to
\emph{neutron matter}. Due to the high kinetic energy of electrons:
\begin{equation}
\ce{p + e- -> n + \nu_e},
\end{equation}
which is also called \emph{inverse neutron decay}. $\rho \sim
\SI{e13}{\gram\per\cubic\centi\metre}$ marks the end of transition to neutron
matter and the formation of a \emph{Neutron Star}. Calculation similar to the
case of White Dwarfs, only for that now $m_{\ce{n}} \approx m$ instead of
$m _{\ce{e}}$, where $m_{\ce{n}}$ is the neutron mass.

\subparagraph{Result}
\begin{equation}
\rfun{f}{\rho} = \frac{p}{\rho\lc^2}
= 1\cdot\rbr{\frac{\rho}{\rho_1}}^\frac{n}{3}
\end{equation}
with
\begin{equation*}
\begin{cases}
n = 2, \rho < \rho_1\qquad \text{non-relativistic case} \\
n = 1, \rho > \rho_1\qquad \text{relativistic case}
\end{cases}
\end{equation*}
and
\begin{equation}
\rho_1 \coloneqq \frac{m}{\lambda_{\ce{n}}^3}
\sim \SI{e16}{\gram\per\cubic\centi\metre}.
\end{equation}
Note the factor $1$ instead of $m_{\ce{e}}/m$.
$\lambda_{\ce{n}}$ is the (reduced) Compton wavelength of neutron,
\begin{equation}
\lambda_{\ce{n}} \approx \SI{2e-14}{\centi\metre},
\end{equation}
comparing with
\begin{equation}
\lambda_{\ce{e}} \approx \SI{4e-11}{\centi\metre}.
\end{equation}
Densities between White Dwarf and Neutron Star region lead to unstable
situation.

% end 21.07.2016

% begin 22.07.2016

\subparagraph{Mass spectrum}
\begin{equation}
\rfun{M}{\rho} = M_\text{C} \cdot
\begin{cases}
\rbr{\frac{\rho}{\rho_1}}^\frac{1}{2}\qquad \rho<\rho_1, \\
1\qquad\rho > \rho_1.
\end{cases}
\end{equation}
As for White Dwarfs, since only $M_\text{C}$ enters. But: details of nuclear
physics are important. Detailed studies lead to $M_\text{NS} \lesssim 3
M_\astrosun$.

Order of relativistic effects:
\begin{equation}
\frac{\nG M}{R\lc^2} \sim \frac{p}{\rho\lc^2}
\approx \rbr{\frac{\rho}{\rho_1}}^\frac{n}{3} \sim 1!
\end{equation}
Definitely non-negligible.

\subparagraph{Radii of Neutron Stars}
\begin{equation}
R \sim \lambda_{\ce{n}} \apG^{-\frac{1}{2}}
\rbr{\frac{\rho_1}{\rho}}^\frac{1}{6}
\approx \rbr{\frac{\rho_1}{\rho}}^\frac{1}{6} \SI{5}{\kilo\metre},
\end{equation}
note $\lambda_{\ce{n}}$ instead of $\lambda_{\ce{e}}$ above. That is to say,
\begin{empheq}[box=\fbox]{equation}
\frac{R_\text{NS}}{R_\text{WD}}\sim\frac{\lambda_{\ce{n}}}{\lambda_{\ce{e}}}.
\end{empheq}
Micro-physics determines size of stars!

Historically: Discovery of pulsars 1967 and their identification with neutron
stars.


\subsection{Interior solution for spherically-symmetric stars}
\label{sec:11-2}
Outside: Schwarzschild solution.

Recall old ansatz for spherically-symmetric metric:
\begin{equation}
\dd s^2 = -\ee^{\rfun{\nu}{r}}\dd\, t^2 + \ee^{\rfun{\lambda}{r}}\dd\, r^2
+ r^2\dd\,\Omega^2.
\label{eq:sph-sym-metric}
\end{equation}
But now $T_{\mu\nu} \neq 0$, sao that $\rfun{\nu}{r}$, $\rfun{\lambda}{r}$
have no $t$-dependence because we look for static equilibrium configurations.

Describe matter phenomenologically by an \emph{ideal fluid},
\begin{equation}
T_{\mu\nu} = \rbr{\rho+p}u_\mu u_\nu + pg_{\mu\nu}.
\label{eq:emt-ideal-fluid}
\end{equation}
In rest frame, this reads $\rfun{\diag}{\rho, p, p, p}$. Inserting this ansatz
into the Einstein equation leads to (cf.\ exercise for spherically-symmetric
line element)
\begin{equation}
\ee^{\rfun{\lambda}{r}} = \rbr{1-\frac{2\nG\rfun{M}{r}}{r}}^{-1},
\label{eq:e-to-lambda-of-r}
\end{equation}
similar to Schwarzschild, but now with
\begin{equation}
\rfun{M}{r}\coloneqq 4\pp\int_0^r\rfun{\rho}{\tld{r}}\tld{r}^2\dd\,\tld{r},
\label{eq:mass-of-radius}
\end{equation}
which takes a Newtonian \emph{form}, but see below!
\begin{equation}
\rfun{\nu}{r} = -\rfun{\lambda}{r} + 8\pp\nG\int_{+\infty}^r
\ee^{\rfun{\lambda}{\tld{r}}}\rbr{\rho+p}\tld{r}\dd\,\tld{r},
\label{eq:nu-of-radius}
\end{equation}
where the second term is `new', which vanishes outside the star. Thus
$\rfun{\nu}{r} = -\rfun{\lambda}{r}$ for $r > R$, degenerating to
Schwarzschild.

% end 22.07.2016
% begin 23.07.2016

\begin{equation}
T_\mu{}^\nu{}_{;\nu} = 0
\label{eq:cons-ene-mom-ten}
\end{equation}
leads to the generalisation of the hydrostatic
equilibrium of \cref{ssec:elem-th-ds}. \textbf{Can be more precise!}

Recall
\begin{equation*}
\frde{p}{r} = -\frac{\nG \rfun{M}{r}}{r^2} \rho
\end{equation*}
Here one finds (skip calculation) ($\mu = r$ in \cref{eq:cons-ene-mom-ten})
\begin{empheq}[box=\fbox]{equation}
\frde{p}{r} = -\frac{\nG\rbr{\rfun{M}{r} + 4\pp r^3 \rho}}%
{r^2\rbr{1-2\nG\rfun{M}{r}/r}}\rbr{\rho+p},
\label{eq:TOV}
\end{empheq}
with boundary condition $\rfun{p}{R} = 0$. In \cref{eq:TOV}, $4\pp r^3 \rho$
on the numerator corresponds to the gravitational field generated by pressure;
$2\nG\rfun{M}{r}/r$ means gravitation increases \emph{faster} than the inverse
square law, which is why there cannot be arbitrarily massive Neutron Stars;
and the $p$ term in $\rbr{\rho + p}$ means gravitation also acts on pressure.
\Cref{eq:TOV} is called the \emph{TOV equation}
\cite{PhysRev.55.374,PhysRev.55.364}.

A special case is that $\rho = \text{const.}$ Then one has
\begin{equation}
\rfun{M}{r} = \frac{4\pp}{3}\rho r^3\qquad\rbr{r \le R}.
\label{eq:const-mass-of-radius}
\end{equation}
Calculation gives
\begin{equation}
\frac{p_0}{\rho} = \frac{1-\sqrt{1-2\nG M/R}}{3\sqrt{1-2\nG M/R}-1}
\xrightarrow{2\nG M \ll R}
\frac{\nG M}{2R} \equiv \frac{\rSch}{4R},
\end{equation}
the denominator of the last expression coincides exactly with our earlier
estimation, \cref{eq:heu-estimate}!

We have $p_0 \to +\inf$ for $3\sqrt{1-2\nG M/R}-1 \to 0^+$, i.e.\ for
\begin{empheq}[box=\fbox]{equation}
R\to\frac{9}{8} \rSch,
\label{eq:stable-radius}
\end{empheq}
in the second paper of \citeauthor{Schwarzschild1916gravitationsfeld}
in \cite{Schwarzschild1916gravitationsfeld}. The result is that stars become
unstable for $R \le 9/8\cdot\rSch$; in General Relativity, pressure generates
more pressure. This limit corresponds to the maximal mass
\begin{equation}
M_\text{max} = \frac{4}{9\rbr{3\nG^3 \pp \rho}^{1/2}}.
\end{equation}

One can generally prove the following (e.g.\ \cite[pp.~129]{Wald1984})
\begin{thm}
For $\rfun{\rho}{r} \ge 0$, $\dd \rho/\dd r \le 0$, one has
\begin{enumerate}
\item
for stars with a fixed radius $R$, $M_\text{max} = 4R/9\nG$ (as for constant
$\rho$);
\item
if below a certain density $\rho_*$ one has a \emph{fixed} equation of state
(behaved to be realistic), one has an $M_\text{max}$ independent of the
equation of state for $\rho > \rho_*$.
\end{enumerate}
\end{thm}

% end 23.07.2016

\subsection{Energy density and binding energy}
\label{ssec:energy-density}

% begin 31.07.2016

Recall stellar mass \cref{eq:stellare-mass}
\begin{equation*}
M\rbr{r}\coloneqq 4\pp\int_0^R \dd r\,r^2\rfun{\rho}{r}
\tag{\ref{eq:stellare-mass} revisited}
\end{equation*}
which is the \emph{Newtonian} expression, but used in the (relativistic)
TOV solution. Note:
\begin{enumerate}
\item
\label{it:comp-den}
$\rho \neq \text{baryon density} n$: $\rho = n\rbr{1+\epsilon}$, where
$\epsilon$ is energy density of \emph{compression} ($\rho > n$), which refers
to matter.
\item
Volume element is determined by the metric
\begin{equation}
\dd V = r^2\dd\, r\rbr{1-\frac{2\nG\rfun{M}{r}}{r}}^{-\frac{1}{2}}
\sin \theta \dd\,\theta\dd\, p
\end{equation}
which is a TOV solution. This leads to
\begin{equation}
M = \int_0^R\dd V\,\rfun{\rho}{r}
\rbr{1-\frac{2\nG\rfun{M}{r}}{r}}^\frac{1}{2},
\end{equation}
where the integrad is the effective mass density per volume element $\dd V$,
and is less than $\rho$, because $\dd V$ is larger than the Newtonian
expression. This will lead to negative grativational self energy.
\end{enumerate}

\paragraph{Baryon number of star}
\begin{equation}
A \coloneqq \frac{1}{m} 4\pp\int_0^R\dd r\,r^2\rfun{n}{r}
\rbr{1-\frac{2\nG\rfun{M}{r}}{r}}^{-\frac{1}{2}},
\end{equation}
where $A$ is the total baryoinc mass, $m$ again proton mass, $\rfun{n}{r}$
baryon density above, and $\rbr{\cdot}^{-1/2}$ comes from $\dd V$ above.

In the special case of incompressible matter $n = \rho = \text{const}$
(no effect \cref{it:comp-den} above), one has
\begin{align}
A &= \frac{4\pp\rho}{m}\int_0^R\dd r\,\r^2
\rbr{1-\frac{8\pp\nG r^2\rho}{3}}^{-\frac{1}{2}} \nonumber \\
&\eqqcolon \frac{4\pp\rho}{m}R^3\rfun{F}{\chi^2},
\end{align}
where
\begin{equation}
\chi^2 \coloneqq \frac{2\nG M}{R} = \frac{8\pp\nG R^2 \rho}{3}
= \frac{\rSch}{R},
\label{eq:RS/R}
\end{equation}
and $F$ is dimensionless, which can be written down in close form, but we need
here only its limiting case
\begin{equation}
\rfun{F}{\chi^2} =
\begin{cases}
1 + \frac{3\nG M}{5R}\qquad\chi\ll 1\quad\text{Newtonian limit}, \\
\frac{3}{4}\pp,\qquad\chi = 1.
\end{cases}
\end{equation}
Stability limit from \cref{eq:stable-radius}: $\chi^2 = 8/9$; one finds
$\rfun{F}{8/9} \approx \num{1.64}$.

Call $mA\eqcolon M_0$: `naked mass', the name of which comes from particle
physics, and is the mass of gas cloud from which the star has formed.

% end 31.07.2016

% begin 01.08.2016

$M < M_0$: `mass defect' at star formation (negative gravitational binding
energy)

Relative mass defect:
\begin{align}
\frac{\Delta M}{M} = \frac{M_0-M}{M} &= \rfun{F}{\frac{2\nG M}{R}} - 1
\nonumber \\
\xrightarrow{\rSch \ll R} \frac{3\nG M}{5 R} &\eqqcolon -\frac{E_\text{B}}{M}
\end{align}
one has
\begin{empheq}[box=\fbox]{equation}
E_\text{B} = -\frac{3\nG M^2}{5R}
\end{empheq}
which is a well known result for the binding energy in Newtonian gravitation
(spherically symmetric case).

Since at the stability limit $\rfun{F}{\chi^2} \approx \num{1.64}$, one thus
has
\begin{equation}
\frac{\Delta M}{M} \approx \num{.64}.
\end{equation}
More detailed calculations for Neutron Stars give
\begin{equation}
\frac{\Delta M}{M} \approx \num{.3}.
\end{equation}


\subsection{Geometry of the Schwarzschild solution}

We had (section 9): section $t = \text{const.}$, $\theta = \pp/2$ through
Schwarzschild metric, embedded in an auxiliary $3$-dimensional space:
\begin{equation}
\rfun{z}{r} = \pm \sqrt{8\nG M}\sqrt{r-2\nG M}
\end{equation}

\textbf{diag: Einstein-Rosen bridge or wormhole}

Here: $\rSch = 2\nG M$ not singular. But what happens inside? Look for a new
coordinate system that is as non-singular as possible, which leads to
\emph{Kruskal--Szekeres coordinates} 
\cite{Szekeres2002,PhysRev.119.1743}.

Two steps; details in exercises.
\begin{enumerate}
\item
Consider for $r > \rSch$:
\begin{align}
\dd s^2 &= -\rbr{1-\frac{2\nG M}{r}}\dd\, t^2 + \rbr{1-\frac{2\nG M}{r}}^{-1}
\dd\, r^2 + r^2 \dd\, \Omega^2 \nonumber \\
&\eqqcolon -\rbr{1-\frac{2\nG M}{r}}\rbr{\dd t^2 - \dd r_*^2}
+ r^2 \dd\, \Omega^2
\end{align}
Why $r_*$? Radial light rays ($\dd s^2 = 0$, $\dd \Omega^2 = 0$) move on
straight lines with inclination $\pp / 4$ to the axes: $\dd t = \pm \dd r_*$
so that \emph{causality relations evident}. This requires
\begin{equation}
\dd r_* = \rbr{1-\frac{2\nG M}{r}}^{-1}\dd\, r,
\end{equation}
which can be integrated as
\begin{equation}
r_* = r + 2\nG M \rfun{\ln}{\frac{r}{2\nG M} -1}.
\label{eq:tortoise-r}
\end{equation}
It is called \emph{tortoise coordinate} of Regge and Wheeler; $r_* \to
-\infty$ for $r \to 2\nG M^+$.

% end 01.08.2016

% begin 02.08.2016

\item
Look for $\rbr{t, r_*} \mapsto \rbr{T, X}$, which preserves this property, but
avoids the coordinate singularity, which is the \emph{Kruskal--Szekeres
coordinates}. Let
\begin{align}
T \coloneqq \ee^{r_*/4\nG M}\sinh\frac{t}{4\nG M}
&= \sqrt{\frac{r}{2\nG M}-1}\ee^{r/4\nG M}\sinh\frac{t}{4\nG M},
\label{eq:KScoord-T}\\
X \coloneqq \ee^{r_*/4\nG M}\cosh\frac{t}{4\nG M}
&= \sqrt{\frac{r}{2\nG M}-1}\ee^{r/4\nG M}\cosh\frac{t}{4\nG M},
\label{eq:KScoord-X}
\end{align}
\end{enumerate}

These give
\begin{empheq}[box=\fbox]{equation}
\dd s^2 = \frac{32\rbr{\nG M}^3}{r} \ee^{-\frac{T}{2\nG M}}
\rbr{-\dd T^2 + \dd X^2} + r^2\rbr{T, X}\dd\, \Omega^2,
\end{empheq}
where $\rbr{-\dd T^2 + \dd X^2}$ can only be given implicitly.

From \cref{eq:KScoord-T,eq:KScoord-X} one finds
\begin{align}
X^2 - T^2 &= \rbr{\frac{r}{2\nG M} -1} \ee^{r/2\nG M},
\label{eq:KScoord-X2-T2}\\
\frac{T}{X} &= \tanh \frac{t}{4\nG M}.
\label{eq:KScoord-X/T}
\end{align}
\Cref{eq:KScoord-X2-T2} means that $r = \text{const.}$ determines a hyperbola
$X^2 - T^2 = \text{const.}$, while \eqref{eq:KScoord-X/T} tells one that
$t = \text{const.}$ is a straight line through $O$.

\textbf{diagrams}

Metric is not only regular for $X \ge \vbr{T}$; the \emph{maximally analytic
extension} of the Schwarzschild manifold has been constructed, namely every
geodesic can either be extended to infinite affine parameter values or hits a
singularity which is a statement independent of coordinates.

\textbf{diagram}

In region 2, the coordinate transformations are given by
\begin{align}
T &= \sqrt{1-\frac{r}{2\nG M}}\ee^{r/4\nG M}\cosh\frac{t}{4\nG M},
\\
X &= \sqrt{1-\frac{r}{2\nG M}}\ee^{r/4\nG M}\sinh\frac{t}{4\nG M};
\end{align}
in region 3 and 4, the coordinates can be induced from 1 and 2 by
\begin{equation}
\rbr{T, X} \mapsto \rbr{-T, -X},
\end{equation}
respectively.

\textbf{diagram}

2 and 4 are called black- and white-hole region, respectively.

$r = 2\nG M \equiv \rSch$ is called the \emph{event horizon} which hides real
singularities, whereas $r=0$ is a real singularity; the \emph{Kretschmann
scalar} in Schwarzschild space-time
\begin{equation}
R_{\mu\nu\lambda\sigma}R^{\mu\nu\lambda\sigma} = 48\frac{\rbr{\nG M}^2}{r^6}
\end{equation}
diverge to $+\infty$ as $r \to 0^+$.

\textbf{explain causal relationships! remark on cosmic censorship!}

% end 02.08.2016
% begin 05.08.2016

\begin{nameddef}{Breaking of time-reversal symmetry}

\textbf{diagram collapsing star}: `Black hole'

\textbf{diagram expanding star}: `White hole', excluded for
thermodynamical reasons)

\end{nameddef}

\begin{nameddef}{Back to the full Kruskal diagram}

\textbf{embedding} Einstein-Rosen bridge or wormhole: dynamical picture of
a collapse
\cite{flamm1916beitraege,PhysRev.48.73}
%PhysRevFocus15.11

metric is static only for $r > \rSch$; $r < \rSch$: $r = \text{const.}$ is
space-like surface, corresponding to constant time.
\end{nameddef}

\begin{nameddef}{Penrose-Carter diagram of the Kruskal space-time}

Study of global structure (causal relations) by an appropriate map of infinity
into finite coordinate range (\emph{not} a coordinate transformation)

Usually used for space-times with symmetries ($2$-dimensional cross section)

\emph{exercise}: Penrose diagram for Minkowski space-time.

\emph{here}: Penrose diagram for Kruskal space-time.

radial light-rays
\begin{equation}
U \coloneqq T-X,\qquad V \coloneqq T+X
\end{equation}
out- and ingoing light-rays, respectively.
\begin{equation}
\begin{aligned}
U' &\coloneqq \arctan U \eqqcolon T'-X', \\
V' &\coloneqq \arctan V \eqqcolon T'+X',
\end{aligned}
\end{equation}
which transform infinities into finite range.

Where is the future singularity at $r = 0$, $T > 0$ mapped to? We had
\begin{align}
X^2 - T^2 &= \rbr{\frac{r}{2\nG M} - 1}\ee^{r/2\nG M}
\tag{\ref{eq:KScoord-X2-T2} rev.}\\
&\xrightarrow{r = 0} T^2 - X^2 = 1,
\end{align}
which yields
\begin{equation}
\begin{aligned}
X = \frac{V-U}{2} = \frac{1}{2}\rbr{\tan V' - \tan U'} =
\frac{1}{2}\rbr{\rfun{\tan}{T'+X'}-\rfun{\tan}{T'-X'}}, \\
T = \frac{V+U}{2} = \frac{1}{2}\rbr{\tan V' + \tan U'} =
\frac{1}{2}\rbr{\rfun{\tan}{T'+X'}+\rfun{\tan}{T'-X'}}.
\end{aligned}
\end{equation}
Then $T^2 - X^2 = 1$ gives $\tan^2 T' = 1$, corresponding to
\begin{equation}
T' = \frac{\pp}{4};
\label{eq:penrose-fs-Tp}
\end{equation}and thus the future singularity
corresponds
the range of $X'$ follows from $-\pp/2 \le U' = T'-X' \le \pp/2$,
$-\pp/2 \le V' = T'+X' \le \pp/2$, which gives $X' \le -\pp/4$ and
$X' \le \pp/4$, respectively. Thus
\begin{equation}
-\frac{\pp}{4} \le X' \le \frac{\pp}{4},
\label{eq:penrose-fs-Xp}
\end{equation}
which is also compatible with $T'-\pp/2 < X' < T' + \pp/2$.
\Cref{eq:penrose-fs-Tp,eq:penrose-fs-Xp} give the future singularity in
Penrose diagram.

% end 05.08.2016
% begin 06.08.2016

Analogously, for the past singularity at $r = 0$ and $T < 0$, one has
\begin{equation}
T' = -\frac{\pp}{4},\qquad -\frac{\pp}{4} \le X' \le \frac{\pp}{4}.
\end{equation}
Horizons at $T = \pm X$ are mapped to $T' = \pm X'$.

Diagram: \textbf{diagram}

Future event horizon is the boundary of the space-time region that can be
connected by a light-ray with $\mscrI^+$. Past event horizon is analogous.

2 is never part of the past for an observer in 1 or 3. 4 is never part of the
future for an observer in 1 or 3.

$\mscrI^\pm$ are the future and past null infinity, $i^0$ space-like infinity,
$i^\pm$ future and past time-like infinity. See \cite{Hawking2010}.
\end{nameddef} % Penrose-Carter diagram of the Kruskal space-time

\paragraph{Geometry of the TOV solution}

We had in \cref{sec:11-2}
\begin{align}
\dd s^2 &= -\ee^{\rfun{\nu}{r}}\dd\, t^2 + \ee^{\rfun{\lambda}{r}}\dd\, r^2
+ r^2\dd\,\Omega^2,
\tag{\ref{eq:sph-sym-metric} rev.} \\
T_{\mu\nu} &= \rbr{\rho+p}u_\mu u_\nu + pg_{\mu\nu}
\tag{\ref{eq:emt-ideal-fluid} rev.}
\end{align}
with
\begin{align}
\ee^{\rfun{\lambda}{r}} &= \rbr{1-\frac{2\nG\rfun{M}{r}}{r}}^{-1},
\tag{\ref{eq:e-to-lambda-of-r} rev.} \\
\rfun{M}{r} &= 4\pp\int_0^r\rfun{\rho}{\tld{r}}\tld{r}^2\dd\,\tld{r},
\tag{\ref{eq:mass-of-radius} rev.} \\
\rfun{\nu}{r} &= -\rfun{\lambda}{r} + 8\pp\nG\int_{+\infty}^r
\ee^{\rfun{\lambda}{\tld{r}}}\rbr{\rho+p}\tld{r}\dd\,\tld{r}.
\tag{\ref{eq:nu-of-radius} rev.}
\end{align}
Defining the spacial part in \cref{eq:sph-sym-metric} as
\begin{equation}
\dd\sigma^2 \coloneqq \ee^{\rfun{\lambda}{r}}\dd\, r^2 + r^2\dd\,\Omega^2,
\end{equation}
imposing $\rho = \text{const.}$ gives
\begin{equation}
\rfun{M}{r} = \frac{4\pp}{3}\rho r^3\qquad\rbr{r \le R},
\tag{\ref{eq:const-mass-of-radius} rev.}
\end{equation}
thus
\begin{equation}
\dd\sigma^2 = \rbr{1-\frac{8\pp\nG\rho r^2}{3}}^{-1}\dd\, r^2
+ r^2\dd\,\Omega^2.
\end{equation}
Denoting $\mscrR \coloneqq \sqrt{3/8\pp\nG\rho}$, the cross section at $\theta
= \pp/2$ is (without loss of generality because of symmetry)
\begin{equation}
\fat{\dd\sigma^2}{\theta = \pp/2} = 
\rbr{1-\rbr{\frac{r}{\mscrR}}^2}^{-1} \dd\, r^2 + r^2\dd\,\phi^2.
\end{equation}
This is the line element of a $2$-dimensional sphere $S^2$ with radius
$\mscrR$. If $\theta$ varies, it would be a $S^3$. The $S^2$ case is important
for cosmology, see \cref{sec:found-cosmo}; briefly shown here:

\textbf{diagram}

$\phi$: `longitude'; $\beta$: `latitude'; usually called $\theta$ elsewhere.
for $S^2$,
\begin{equation}
\dd l^2 = \mscrR^2\rbr{\dd\beta^2 + \sin^2\beta\dd\,\theta^2}.
\end{equation}

% end 06.08.2016
% begin 07.08.2016

Choose dimensionless coordinate $r$ ($0 \le r \le 1$) such that the
circumference of a latitude circle ($r = \text{const.}$) is $2\pp \mscrR r$.
Now that
\begin{align}
\sin\beta = \frac{\mscrR r}{\mscrR} = r
\quad &\hookrightarrow \quad
\frde{r}{\beta} = \cos\beta = \sqrt{1-r^2} \nonumber \\
&\hookrightarrow \quad \dd \beta^2 = \frac{\dd r^2}{1-r^2}.
\end{align}
Thus
\begin{equation}
\dd l^2 = \mscrR^2\rbr{\frac{\dd r^2}{1-r^2}+r^2\dd\,\phi^2}.
\end{equation}
Comparison with $\dd \sigma^2$: is the same after $r\to \mscrR r$, i.e.\
\begin{equation}
\dd l^2 = \fat{\dd \sigma^2}{\theta = \pp/2}.
\end{equation}

\paragraph{Cross section}

\textbf{diagram: matter distribution from centre up to stellar radius $R$
($r = 0$ to $r = R/\mscrR$) (only here geometry is $\dd l^2$)}

\begin{equation}
\sin\beta = \frac{\mscrR r}{r} = \frac{R}{\mscrR} = \chi;
\end{equation}
we had
\begin{align}
\chi^2 &= \frac{\rSch}{R} = \frac{2\nG M}{R} = \frac{8\pp\nG R^2 \rho}{3}
\tag{\ref{eq:RS/R} rev.} \\
&= \rbr{\frac{R}{\mscrR}}^2
\end{align}
(recall $\mscrR = \sqrt{3/8\pp\nG\rho}$).

$\chi = 1$: Matter distribution corresponds to half the sphere $r = 1$;
$\chi < 1$: smaller sections of the sphere.

Recall that solutions with $\beta > \arcsin\sqrt{8/9}\approx\ang{63}$ are
unstable. For the full Schwarzschild metric, one thus has the following
\emph{embedding diagram}

\textbf{diagram: same exterior mass $M$.
$\beta > \pp/2$ (not treated above; is unstable)}

continuous connection between interior solution and Schwarzschild solution,
corresponds to the popular picture of the `trampoline'.

\subsection{Charged black holes}

Spherically-symmetric solution of Einstein equation in the presence of an
\emph{electric charge} $q$, called \emph{Reissner\footnote{%
\citeauthor{Reissner1916} in \cite{Reissner1916}, written in
Berlin, was also in Aachen.}--Nordstörm\footnote{%
\citeauthor{nordstroem1918ander} in \cite{nordstroem1918ander}.} solution}.
Cf.\ exercise sheet of last week.

\emph{matter part}: Coulomb field, the energy-momentum tensor for which is
\begin{equation}
4\pp T_{\mu\nu} = F_{\mu\rho}F^\rho{}_\nu - \frac{1}{4} g_{\mu\nu}
F_{\rho\lambda} F^{\rho\lambda},
\end{equation}
see Relativity 1. Evaluating this for
\begin{equation}
\ora{E} = \frac{q}{r^2}\,\what{e}_r\qquad\text{or}\quad E_r = F_{rt}
\end{equation}
in which Gaussian units are used, yields in local inertial system
\begin{equation}
4\pp T_{\mu\nu} = \frac{q^2}{2r^4}\rfun{\diag}{1, -1, 1, 1}.
\end{equation}
The solution is best found using the geometric units of \cref{sec:diff-geo-2},
see exercise sheet of last week.

% end 07.08.2016
% begin 08.08.2016

\begin{nameddef}{Reissner--Nordstörm solution}
\begin{equation}
\dd s^2 = -\rbr{1-\frac{2M}{r}+\frac{q^2}{r^2}}\dd\, t^2
+\rbr{1-\frac{2M}{r}+\frac{q^2}{r^2}}^{-1} \dd\, r^2 + r^2 \dd\, \Omega^2
\label{eq:RN-sol}
\end{equation}
\end{nameddef} % Reissner--Nordstörm solution

Remark: In the presence of a magnetic charge, one would have the same line
element with $q^2 \to q^2 + q_\text{m}^2$.

Remark on units: taking into account $\nG$, one has $\rbr{1-\frac{2\nG M}{r}
+\frac{\nG q^2}{r^2}}$, i.e.\ $q\sqrt{\nG}$ has the dimension of a
\emph{length}.

E.g.\ elementary charge $\ec = \SI{4.8e-10}{esu} = \SI{1.6e-19}{\coulomb}$,
where $\si{esu} = \si{\gram^{1/2}\centi\metre^{3/2}\per\second}$.

Let $\lc = 1$ so that $\SI{1}{\second} = \SI{3e10}{\centi\metre}$,
\begin{equation}
\ec = \frac{\SI{4.8e-10}{\gram^{1/2}\centi\metre^{3/2}}}%
{\SI{3e10}{\centi\metre}} = \SI{1.6e-20}{\gram^{1/2}\centi\metre^{1/2}}.
\end{equation}
In addition, $G = \SI{6.67e-8}%
{\cubic\centi\metre\per\gram\per\square\second}$, so that
\begin{align}
\sqrt{\nG}\ec &= \frac{\sqrt{6.67}\times
\SI{e-4}{\centi\metre^{3/2}\gram^{1/2}}}{\SI{3e10}{\centi\metre}}\times
\SI{1.6e-20}{\gram^{1/2}\centi\metre^{1/2}} \nonumber \\
&\approx \SI{1.38e-34}{\centi\metre} \approx \sqrt{\apE}\pll,
\end{align}
where $\apE \approx 1/137$ fine structure constant of electromagnetic
interaction, $\pll$ Planck length.


\begin{namedrem}{Remark}
Reissner--Nordstörm solution is obtained from the Schwarzschild
solution by the substitution
\begin{equation}
M \to M - \frac{q^2}{2r},
\end{equation}
where $-q^2/2r$ is the electrostatic energy of a spherical shell with radius
$r$ and charge $q$.
\end{namedrem} % Remark

The Birkhoff theorem also holds for the Reissner--Nordstörm solution.

For the Schwarzschild solution, we had the horizon for $g_{tt} = 0$; here,
$g_{tt} = 0$ leads to possibly \emph{two zero points}
\begin{equation}
r_\pm = M\pm\sqrt{M^2 - q^2}.
\end{equation}
One has two different real zeros for $\vbr{q} < M$, two degenerate real zeros
(or one) for $\vbr{q} = M$, and no real zero for $\vbr{q} > M$. For
the last case, $g_{tt} = -g^{rr}$ is always positive, i.e.\ $r$ is always a
space-like coordinate. Curvature singularity at $r = 0$ is \emph{not} hidden
by any event horizon, leading to a \emph{naked singularity}.

\paragraph{Penrose diagram for $\vbr{q} > M$}

\textbf{diagram: naked time-like singularity}

Remark: neglecting spin, for the electron one has
\begin{equation}
\frac{\vbr{q}}{M} \sim \frac{\SI{e-34}{\centi\metre}}{\SI{e-55}{\centi\metre}}
= \num{e21}.
\end{equation}
Is the electron a naked singularity?

\paragraph{Penrose diagram for $\vbr{q} < M$}

\textbf{diagram: naked time-like singularities; example for the world-line
of an observer}

I: our Universe; $r_+$: event horizon; $r_-$: Cauchy horizon, end of
predictability.

Ausführlich? Hehl empfiehlt \cite{Griffiths2009}.

For more details, see e.g.\
\cite{Hawking1973,chandrasekhar1998mathematical}

% end 08.08.2016
% begin 09.08.2016

\begin{unamedrem}
\begin{itemize}
\item
Singularity can be avoided (because it is time-like)
\item
For neutral particle, $r = 0$ turns out to be a potential barrier. Such
particles cannot reach the singularity. Intuitively, Coulomb energy
approaches infinity as $r \to 0^+$; must be compensated by infinite negative
energy to fulfil $E_\text{ADM} = M$.
\item
If $r_-$ is reached, it is reached in finite proper time. Observer sees the
whole `future' of I upon crossing $r_-$.
\item
There are indications that $r_-$ is \emph{unstable} with respect to
perturbations (infinite blue-shift etc.)
\end{itemize}
\end{unamedrem}

\begin{nameddef}{Penrose diagram for $\vbr{q} = M$}

\textbf{figure: $r_+ = r_- = M$; not parts of singularity}

\end{nameddef} % Penrose diagram

Astrophysically, Reissner--Nordstörm black holes have no relevance, but they
are a useful toy model for rotating black holes, to which we now turn.


\subsection{Rotating black holes}

\begin{nameddef}{Kerr solution}
\citeauthor{PhysRevLett.11.237} in \cite{PhysRevLett.11.237}.

Vacuum solution of Einstein equations, characterised by \emph{mass} $M$ and
\emph{angular momentum} $J$. Use is often made of
\begin{align}
a \coloneqq \frac{J}{M}\quad\text{with dimension of length}, \\
\label{eq:a=J/M}
a_* \coloneqq \frac{J}{M^2} = \frac{J\lc}{\nG M^2}\quad
\text{dimensionless}.
\end{align}
Give here the solution in \emph{Boyer--Lindquist coordinates}
\cite{:/content/aip/journal/jmp/8/2/10.1063/1.1705193} without derivation
\begin{empheq}[box=\fbox]{align}
\dd s^2 =& -\rbr{1-\frac{2Mr}{\varrho^2}}\dd\, t^2
-\frac{4Mar\sin^2\theta}{\varrho^2}\dd\,\phi\dd\, t
\nonumber \\
&+\frac{\varrho^2}{\Delta}\dd\, r^2 + \varrho^2\dd\,\theta^2
+\rbr{r^2+a^2+\frac{2Mra^2\sin^2\theta}{\varrho^2}\sin^2\theta\dd\,\phi^2}
\\
=& -\frac{\Delta}{\varrho^2}\rbr{\dd t - a\sin^2\theta\dd\,\theta}^2
+\frac{\sin^2\theta}{\varrho^2}\rbr{\rbr{r^2+a^2}\dd\,\theta-a\dd\, t}^2
\nonumber \\
&+\frac{\varrho^2}{\Delta}\dd\, r^2 + \varrho^2\dd\,\theta^2,
\end{empheq}
with
\begin{align}
\varrho^2 \coloneqq & r^2+a^2\cos^2\theta, \\
\label{eq:def-varrho-Kerr}
\Delta \coloneqq & r^2 - 2Mr + a^2.
\end{align}

For $r \gg M$, $r \gg a$, one finds (exercises next week)
\begin{align}
\dd s^2 \approx & -\rbr{1-\frac{2M}{r}}\dd\, t^2
+ \rbr{1+\frac{2M}{r}}\dd\, r^2
+ r^2\rbr{\dd\theta^2+\sin^2\theta\dd\,\phi^2} \nonumber \\
&- \frac{4Ma}{r^2}\sin^2\theta\,\rbr{r\dd\,\phi}\dd\, t
+ \rfun{\Omicron}{\frac{1}{r^2}}.
\label{eq:Kerr-linear}
\end{align}
% end 09.08.2016
% begin 10.08.2016
In \cref{eq:Kerr-linear}, $M$ can be measured from orbit of distant satellite,
whereas $a$ can be measured from precession of distant orbiting gyroscope,
which is \emph{Thirring--Lense effect} introduced in section 9.7. The
measurement of $M$ and $a$ in the linear form of the Kerr metric justifies
the identification of $M$ as mass and of $J$ as angular momentum.

\Cref{eq:Kerr-linear} describes an \emph{asymptotically flat} space-time,
while the $\dd\phi\dd\, t$-term, which equals (exercise)
\begin{equation*}
-\frac{4I}{r^3}\epsilon_{ijk}\dd\, x^i\omega^j x^k\dd\, t,
\end{equation*}
where $I$ is the moment of inertia (cf.\ section 9.7), if rotation is assumed
with angular velocity $\ora{\omega}$ around $z$-axis.
\begin{proof}
\begin{align}
\frac{4I}{r^3} \epsilon_{ijk}\dd\, x^i\omega^j x^k\dd\, t
=& \frac{4I\omega}{r^3} \epsilon_{ijk}\dd\, x^i\omega^j x^k\dd\, t \nonumber \\
=& \frac{4J}{r^3} r^2\sin^2\theta\dd\,\phi\dd\, t \nonumber \\
=& \frac{4Ma}{r^2} \sin^2\theta\rbr{r\dd\,\phi}\dd\, t,
\end{align}
where in the first, second and third line, $\ora{\omega} =
\begin{pmatrix} 0 \\ 0 \\ \omega\end{pmatrix}$, $J = I\omega$ as well as
$x\dd\, y - y\dd\, x = r^2\sin^2\theta\dd\,\phi$, and $a = J/M$ have been
used, respectively.
\end{proof}

The Kerr metric is independent of $t$, so it is \emph{stationary}; it is also
independent of $\phi$, so it is \emph{axially symmetric}. But it is \emph{not}
static, in that it is \emph{not} invariant under the time reversal
transformation $t\to -t$ which changes the direction of $\ora{J}$; this can be
compensated by $\phi \to -\phi$ in addition. It is \emph{not} spherically
symmetric as well, because $g_{tt}$ and $g_{rr}$ depend on $\theta$.
\end{nameddef} % Kerr solution

\begin{nameddef}{Killing vectors}
It has two Killing vectors
\begin{align}
\xi^\mu \equiv \ul{\xi} &= \begin{pmatrix} 1 \\ 0 \\ 0 \\ 0 \end{pmatrix}
\equiv \rbr{\partial_t}^\mu,
\label{eq:Kerr-killing-1} \\
\psi^\mu \equiv \ul{\psi} &= \begin{pmatrix} 0 \\ 0 \\ 0 \\ 1 \end{pmatrix}
\equiv \rbr{\partial_\phi}^\mu,
\label{eq:Kerr-killing-2}
\end{align}
which correspond to the stationarity and axial symmetry, respectively. The
invariants are $\rfun{g}{\ul{\xi}, \ul{\xi}} = g_{tt}$, $\rfun{g}{\ul{\psi},
\ul{\psi}} = g_{\phi\phi}$ and $\rfun{g}{\ul{\xi}, \ul{\psi}} = g_{t\phi}$.

Because of axial symmetry, the metric is invariant under reflection at the
equatorial plane $\theta = \pp/2$, or $\theta \to \pp - \theta$.

Killing tensor: \cite{PhysRev.174.1559}
\end{nameddef} % Killing vectors

\begin{nameddef}{Exercise}
For $a \to 0$, one obtains the Schwarzschild metric in Schwarzschild
coordinates; For $M \to 0^+$ while $a$ fixed, one obtains flat
space-time.
\end{nameddef} % Exercise

\begin{unamedrem}
By the transformation
\begin{equation}
M \to M-\frac{q^2}{2r},
\end{equation}
one arrives at the \emph{Kerr--Newmann metric}
\cite{:/content/aip/journal/jmp/6/6/10.1063/1.1704350}, which is the most
general solution for a stationary black hole (remark on uniqueness theorems)
(one of next exercise sheets)
\end{unamedrem}

\begin{nameddef}{What about the singularities?}
% end 10.08.2016
% begin 21.08.2016
Occur for
\begin{align}
\varrho &= 0, \qquad \text{and}
\label{eq:varrho=0} \\
\Delta &= 0.
\label{eq:Delta=0}
\end{align}
\Cref{eq:varrho=0} is equivalent to $r^2+a^2\cos^2\theta = 0$, or
\begin{equation}
r = 0,\qquad\theta = \frac{\pp}{2},
\end{equation}
which signals a physical singularity; more precisely, it is a \emph{ring
singularity} or \emph{singular ring}. $r = 0$ is the whole ring, i.e.\ $r$
is not the old radial coordinate. On the other hand, \cref{eq:Delta=0} is
identical with
\begin{equation}
r_\pm = M \pm \sqrt{M^2 - a^2}\qquad\text{if }\vbr{a} \le M,
\end{equation}
which turns out to be \emph{coordinate singularities}. With $a \to 0$,
$r_+$ approaches $2M \equiv \rSch$. This is analogue to Reissner--Nordstörm
solution with the replacement $a \to q$. $\vbr{a} = M$ is again the case for
extremal black holes, which means that for fixed $M$, the angular momentum is
bounded from above by $M^2$ (cf.\ \cref{eq:a=J/M}); this is not the case
for ordinary objects such as Sun. (Exercise next week)
\end{nameddef} % What about the singularities?

\begin{unamedrem}
\begin{itemize}
\item
Kerr solutions does not describe a rotating star. (non-validity of Birkhoff
theorem)
\item
Accretion disk: matter falls into a black hole after spiralling down, which
carries $J$, so that $J$ of black hole increases as well, and bring black
hole closer to extremal limit. (?)
\end{itemize}
\end{unamedrem}

\begin{nameddef}{The event horizon at $r = r_+$}
The horizon of a black hole is the null three-surface interior boundary of the
space-time region from which a light ray can escape to infinity from any
point.

\textbf{diagram: $l^\mu$ plus two tangential vectors}

at any point there is a light-like tangential direction $\rfun{g}{l, l} = 0$
which is orthogonal to two independent space-like tangential directions.
\end{nameddef} % The event horizon at $r = r_+$

\begin{nameddef}{The surface $r_+ = \text{const.}$ is light-like}
(That it is really a horizon; see \ref{nthm:pd-Kerr}.)

Ansatz for tangent vector at $r = r_+ = \text{const.}$:
\begin{equation}
l^\mu = \begin{pmatrix} l^t \\ 0 \\ l^\theta \\ l^\phi \end{pmatrix},
\end{equation}
the $r$-component vanishes because $r$ being constant. The condition for
light-like direction reads
\begin{empheq}[box=\fbox]{equation}
g_{\mu\nu}l^\mu l^\nu = 0.
\end{empheq} % equation
Inserting the Kerr metric, this leads after some calculation to
\begin{equation}
g_{\mu\nu} l^\mu l^\nu = g_{tt} \rbr{l^t}^2 + 2 g_{t\phi}l^t l^\phi
+ g_{\phi\phi}\rbr{l^\phi}^2 + g_{\theta\theta}\rbr{l^\theta}^2 = 0,
\end{equation}
or
\begin{equation}
\rbr{\frac{2Mr_+\sin\theta}{\varrho_+}}^2
\rbr{l^\phi - \frac{a}{2Mr_+}l^t}^2
+ \varrho_+^2\rbr{l^\theta}^2 = 0,
\label{eq:lightlike-Kerr}
\end{equation}
where $\varrho_+ \coloneqq \rfun{\varrho}{r_+,\theta}$; recall
\cref{eq:def-varrho-Kerr}.
Each term in \cref{eq:lightlike-Kerr} must vanish identically. In the first
term, the vanishing condition applying to the second factor gives
$l^\phi = \rbr{a/2Mr_+}l^t$, while the same argument for the second term
yields $l^\theta = 0$. These lead to
\begin{equation}
l^\mu = \begin{pmatrix} l^t \\ 0 \\ 0 \\ \rbr{a/2Mr_+}l^t \end{pmatrix}
\eqqcolon \begin{pmatrix} 1 \\ 0 \\ 0 \\ \Omega_\text{H} \end{pmatrix}l^t,
\end{equation}
where
\begin{equation}
\Omega_\text{H} \coloneqq \frac{a}{2Mr_+} = \frac{a\lc^3}{2\nG M r_+},
\end{equation}
which has a dimension of $\rbr{\text{time}}^{-1}$,
and $l^t$ can be set to unity because $l^\mu$ is unique up to
a non-zero multiplicative constant.

Two other (space-like) directions orthogonal to $l^\mu$ are given by
\begin{equation}
\begin{pmatrix} 0 \\ 0 \\ 1 \\ 0 \end{pmatrix} \quad \text{and} \quad
\begin{pmatrix} 0 \\ 0 \\ 0 \\ 1 \end{pmatrix}.
\end{equation}
$l^\mu$ is not only tangential to the horizon, but also normal to it,
because it is orthogonal to the other tangent vectors.
\end{nameddef} % The surface $r_+ = \text{const.}$ is light-like
% end 21.08.2016

% begin 22.08.2016
\begin{namedrem}{Interpretation of $\Omega_\text{H}$}
Light rays on the horizon starting in direction $l^\mu$ stay on the horizon
(null surface), but rotate with respect to `far-away stars', such that
\begin{equation}
\frde{\phi}{t} = \frac{\dd\phi/\dd\lambda}{\dd t/\dd\lambda}
= \frac{l^\phi}{l^t} = \Omega_\text{H},
\end{equation}
so that $\Omega_\text{H}$ is the `angular velocity of black hole'. Light rays
on the horizon rotate rigidly with $\Omega_\text{H}$ with respect to 
`far-away stars'.
\end{namedrem} % Interpretation of $\Omega_\text{H}$

\begin{namedrem}{Remark}
Although horizon lies at $r = r_+$, the geometry of the horizon is \emph{not}
spherically symmetric (cf.\ exercises), i.e.\
\begin{equation}
\fat{\dd \Sigma^2}{r = r_+, t = \text{const.}} \coloneqq
\varrho_+^2 \dd\,\theta^2 + \rbr{\frac{2Mr_+}{\varrho_+}}^2\sin^2\theta
\dd\,\phi^2
\end{equation}
does \emph{not} have the geometry of a sphere. In fact, the surface area of
$r = r_+$ is
\begin{equation}
A = 8\pp M r_+ = 8 \pp M \rbr{M+\sqrt{M^2-a^2}}.
\end{equation}
\end{namedrem} % Remark

\begin{nameddef}{Penrose diagramme for Kerr balck hole}
\label{nthm:pd-Kerr}
Show here $\vbr{a} < M$ only. $\vbr{a} > M$: nacked singularity, c.f. RN case.

\textbf{diagram}: $r_- = M - \sqrt{M^2 - a^2}$, Cauchy horizon.
`ring singularity' at $\varrho^2 = 0$. `Our Universe'. `Anti-gravitation
universe', does also contain closed time-like curves. Can travel through ring
singularity into anti-gravitation universe, not present in RN solution.
\end{nameddef} % Penrose diagramme for Kerr balck hole

\begin{nameddef}{Geodesics in the Kerr geometry}
More complex than in the Schwarzschild case, where conservation of angular
momentum for a test particle guarantees that the orbit stay within one plane.
Here, general orbits do not stay in one plane, except those which start in the
equatorial plane $\theta = \pp/2$, since angular momentum conservation holds
with respect to the symmetry axis (symmetry of Kerr metric under $\phi \to
\pp - \theta$); restrict yourself here to such orbits!

For $\theta = \pp/2$, we had $\varrho^2 = r^2 + a^2\cos^2\theta = r^2$, so
that
\begin{equation}
\fat{\dd s^2}{\theta = \pp/2} = -\rbr{1-\frac{2M}{r}}\dd\, t^2
- \frac{4aM}{r}\dd\, t\dd\,\phi + \frac{r^2}{\Delta}\dd\, r^2
+ \rbr{r^2 + a^2 + \frac{2Ma^2}{r}}\dd\,\phi^2.
\end{equation}
\end{nameddef} % Geodesics in the Kerr geometry

\begin{nameddef}{Conserved quantities}
\begin{equation}
F\coloneqq -\ul{\xi}\cdot\ul{u} \equiv -g_{\mu\nu}\xi^\mu u^\nu
\end{equation}
is the energy per unit mass, and
\begin{equation}
l \coloneqq \ul{\psi}\cdot\ul{u} \equiv g_{\mu\nu}\psi^\mu u^\nu
\end{equation}
is the angular momentum per unit mass with respect to symmetry axis, which is
also the total angular momentum for equatorial orbits.
% end 22.08.2016
% begin 23.08.2016

Express $F$ and $l$ by the metric:
\begin{align}
F &= -g_{\mu\nu}\xi^\mu u^\nu = g_{t\nu}u^\nu = -g_{tt}u^t - g_{t\phi}u^\phi,
\label{eq:F-Kerr}\\
l &= g_{\mu\nu}\psi^\mu u^\nu = g_{\phi\nu}u^\nu
= g_{\phi t}u^t + g_{\phi\phi}u^\phi,
\label{eq:l-Kerr}
\end{align}
where
\begin{equation}
\xi^\mu = \begin{pmatrix} 1 \\ 0 \\ 0 \\ 0 \end{pmatrix},\qquad
\psi^\mu = \begin{pmatrix} 0 \\ 0 \\ 0 \\ 1 \end{pmatrix}.
\end{equation}

Solving \cref{eq:F-Kerr,eq:l-Kerr} with respect to $u^t$ and $u^\phi$ and
inserting Kerr metric for $\theta = \pp/2$ give rise to 
\begin{align}
u^t = \frde{t}{s} &= \frac{1}{\Delta}\sbr{\rbr{r^2+a^2+\frac{2Ma^2}{r}}F
- \frac{2Ma}{r}l},
\label{eq:u^t-Kerr}\\
u^\phi = \frde{\phi}{s} &= \frac{1}{\Delta}\sbr{\rbr{1-\frac{2M}{r}}l
+\frac{2Ma}{r}F}.
\label{eq:u^phi-Kerr}
\end{align}

Use now $\ul{u}\cdot\ul{u} \equiv g_{\mu\nu}u^\mu u^\nu = -1$ (time-like
geodesic):
\begin{equation}
g_{tt}\rbr{u^t}^2 + 2 g_{t\phi} u^t u^\phi + g_{\phi\phi}\rbr{u^\phi}^2
+ g_{rr}\rbr{u^r}^2 = -1,
\end{equation}
where $u^\theta = 0$ since $\theta = \pp/2$. One inserts
\cref{eq:u^t-Kerr,eq:u^phi-Kerr} for $u^t$ and $u^\phi$, and replaces $u^r$
with $\dd r/\dd s$, yielding the following \emph{energy law} (cf.\ situation
for Schwarzschild metric)
\begin{empheq}[box=\fbox]{equation}
\frac{1}{2}\rbr{\frde{r}{s}}^2 + \rfun{V_\text{eff}}{r} = \frac{F^2-1}{2}
=\text{const},
\end{empheq} % equation
where
\begin{equation}
\rfun{V_\text{eff}}{r} \coloneqq -\frac{M}{r}+\frac{l^2-a^2\rbr{F^2-1}}{2r^2}
-\frac{M\rbr{l-aF}^2}{r^3}.
\end{equation}

Observe that
\begin{equation}
\rfun{V_\text{eff}}{r} \to -\frac{M}{r} + \frac{l^2}{2r^2} - \frac{Ml^2}{r^3},
\qquad a\to 0,
\end{equation}
which is same as the Schwarzschild case.

For null geodesics, use
\begin{equation}
\ul{u}\cdot\ul{u} \equiv g_{\mu\nu}\frde{x^\mu}{\lambda}\frde{x^\nu}{\lambda}
= 0;
\end{equation}
an analogous calculation leads to
\begin{empheq}[box=\fbox]{equation}
\frac{1}{l^2}\rbr{\frde{r}{\lambda}}^2 = \frac{1}{b^2}
- \rfun{W_\text{eff}}{r},
\end{empheq} % equation
where
\begin{empheq}[box=\fbox]{equation}
b \coloneqq \vbr{\frac{l}{F}}
\end{empheq} % equation
is the \emph{impact parameter}, and
\begin{equation}
\rfun{W_\text{eff}}{r} \coloneqq \frac{1}{r^2}\sbr{1-\rbr{\frac{a}{b}}^2
-\frac{2M}{r}\rbr{1-\sigma\frac{a}{b}}^2},
\end{equation}
% end 23.08.2016
% begin 02.09.2016
where
\begin{equation}
\sigma \coloneqq \rfun{\sgn}{l},
\end{equation}
so that $\sigma = \pm 1$ corresponds to co- and counter-rotation,
respectively. One also has
\begin{equation}
\rfun{W_\text{eff}}{r} \to r^{-2} - 2Mr^{-3}, \qquad a \to 0,
\end{equation}
which is identical to the Schwarzschild case.

Note that, unlike in the Schwarzschild case, $V_\text{eff}$ and $W_\text{eff}$
here depend explicitly on $F$ and on the sign of $l$, the latter of which give
rise to the difference between `co-roarting' and `counter-rorating'.
\end{nameddef} % Conserved quantities

\begin{nameddef}{Important property for astrophysics}%
[Binding energy of the innermost stable circular orbit (ISCO)]
Circular orbit at $r = R$:
\begin{equation}
\frde{r}{s} = 0 \quad \Rightarrow \quad
\rfun{V_\text{eff}}{R} = \frac{F^2-1}{2};
\end{equation}
also $V_\text{eff}$ has a minimum at $r = R$, so that
\begin{equation}
\fat{\frde{V_\text{eff}}{r}}{r=R} = 0,\qquad
\frde{^2 V_\text{eff}}{r^2} > 0.
\end{equation}
From these one finds $F$, $l$, $R$ for an innermost stable circular orbit
(use $\dd^2 V_\text{eff}/\dd r^2 = 0$: just on the verge of being unstable).

Simple case for an ISCO: Extremal Kerr black hole $\vbr{a} = M$. One finds
\begin{equation}
F = \frac{1}{\sqrt{3}},\qquad l = \frac{2M}{\sqrt{3}}
\end{equation}
and
\begin{equation}
R = M
\label{eq:ISCOr=M}
\end{equation}
for the co-rotating case. \Cref{eq:ISCOr=M} means that the orbit can be very
`close' (at least in the coordinate distance) to the black hole. For the
proper distance, let
\begin{equation}
\vbr{a} = M\rbr{1-\delta}\quad\Rightarrow\quad R = M\rbr{1+2\sqrt{\delta}},
\end{equation}
so that
\begin{equation}
\fat{\rbr{R-r_+}}{\text{proper distance}} \to M\rfun{\log}{1+\sqrt{2}},
\qquad\delta\to 0^+,
\end{equation}
which is still `very close'.

Binding energy of the orbit per unit mass:
\begin{equation}
1-F,
\end{equation}
where $1 = m/m$ is the rest energy at infinity, $F$ the energy on orbit
(measured from infinity). Maximally one has
\begin{equation}
1 - \frac{1}{\sqrt{3}} \approx \SI{42}{\percent}
\end{equation}
of the rest energy being able to be released, which is much more than the
release in nuclear processes (few perent) [realistically, up to $\approx
\SI{30}{\percent}$]

\textbf{diagram}: Schwarzschild value: $r_\text{ISCO} = 6M = 3\rSch$

\textbf{diagram}: Binding energy for ISCOs
\end{nameddef} % Important property for astrophysics

\begin{nameddef}{Ergosphere}[Not a `sphere'; better `ergoregion'?]
Special region between $r_+$ and `asymptotic' region.

We had the killing vectors
\begin{align}
\xi^\mu \equiv \ul{\xi} &\equiv \rbr{\partial_t}^\mu =
\begin{pmatrix} 1 \\ 0 \\ 0 \\ 0 \end{pmatrix},
\tag{\ref{eq:Kerr-killing-1} rev.} \\
\psi^\mu \equiv \ul{\psi} &\equiv \rbr{\partial_\phi}^\mu =
\begin{pmatrix} 0 \\ 0 \\ 0 \\ 1 \end{pmatrix}.
\tag{\ref{eq:Kerr-killing-2} rev.}
\end{align}
Usually, $g_{tt} = g_{\mu\nu}\xi^\mu \xi^\nu$ is negative, in which $\xi^\mu$
is a time-like vector tangential to a stationary observer. For Kerr metric,
\begin{equation}
g_{tt} = -\frac{\Delta - a^2\sin^2\theta}{\varrho^2}
\end{equation}
can become \emph{positive}:
\begin{equation}
\Delta - a^2\sin^2\theta = r^2 - 2Mr + a^2 - a^2\sin^2\theta =
r^2 - 2Mr + a^2\cos^2\theta \lls 0,
\end{equation}
which is the case for $r_2 < r < r_1$, where
\begin{align}
r_2 &= M - \sqrt{M^2 - a^2\cos^2\theta} < r_- = M - \sqrt{M^2-a^2},
\\
r_1 &= M + \sqrt{M^2 - a^2\cos^2\theta} \eqqcolon r_\text{E} > r_+.
\end{align}
One sees that $r_2$ lies within the Cauchy horizon and is thus irrelevant,
whereas $r_\text{E}$ lies \emph{outside} the event horizon and thus \emph{is}
relevant!
% end 02.09.2016
% begin 04.09.2016

The region $r_+ < r < r_\text{E}$ is called \emph{ergosphere}\footnote{ergon
(έργο) means work; will become clear below.}.

Here: $\xi_\mu \xi^\mu > 0$, so that $\xi^\mu$ space-like.

\textbf{diagram:} $r_+$; $\rfun{r_\text{E}}{\theta}$ not an embedding
diagram!

An observer in the ergosphere cannot remain static, i.e.\ his four-velocity,
which must be time-like, must exhibit non-vanishing spatial components;
angular velocity with respect to rest frame at infinity
\begin{equation}
\Omega \coloneqq \frde{\phi}{t} = \frac{\dd \phi/\dd s}{\dd t/\dd s}
= \frac{u^\phi}{u^t},
\label{eq:Omega=dphi/dt}
\end{equation}
where $u^\mu = \dd x^\mu/\dd s$; $s$ proper time along observer's curve.

Assume that observer has $r = \text{const.}$, $\theta = \text{const.}$, but
\emph{not} $\phi = \text{const}$. Consider four-velocity
\begin{align}
\ul{u} &= u^t\ul{\xi} + u^\phi\ul{\psi} \nonumber \\
&= u^t\ul{\xi} + \Omega u^t\ul{\psi} \nonumber \\
&= u^t \rbr{\ul{\xi} + \Omega \ul{\psi}},
\end{align}
where on the second line \cref{eq:Omega=dphi/dt} was used. Now,
\begin{equation}
g_{\mu\nu} u^\mu u^\nu \equiv \ul{u} \cdot \ul{u} = -1
= -\rbr{u^t}^2\rbr{\ul{\xi}+\Omega\ul{\psi}}^2,
\end{equation}
so that
\begin{equation}
u^t = \frac{1}{\vbr{\ul{\xi} + \Omega\ul{\psi}}}
\end{equation}
and
\begin{empheq}[box=\fbox]{equation}
\ul{u} = \frac{\ul{\xi} + \Omega\ul{\psi}}{\vbr{\ul{\xi} + \Omega\ul{\psi}}},
\label{eq:Kerr-ulu}
\end{empheq} % equation
which means observer follows orbits of $\ul{\xi} + \Omega\ul{\psi}$.

\begin{equation}
\ul{\chi} \coloneqq \ul{\xi} + \Omega\ul{\psi}
\end{equation}
is again a Killing field because it is a sum of two Killing fields. When
evaluated at horizon,
\begin{equation}
\fat{\chi^\mu}{\text{horizon}} = \xi^\mu + \Omega_\text{H}\psi^\mu \equiv
l^\mu = \begin{pmatrix} 1 \\ 0 \\ 0 \\ \Omega_\text{H} \end{pmatrix},
\end{equation}
where the angular velocity of the horizon
\begin{equation}
\Omega_\text{H} = \frac{a}{2Mr_+} = \frac{a}{r_+^2 + a^2} \to \frac{1}{2M},
\qquad a\to M^-.
\end{equation}
\end{nameddef} % Ergosphere

\begin{namedrem}{Notes}
\begin{enumerate}
\item
\begin{equation}
\Omega = \frde{\phi}{t},
\tag{\ref{eq:Omega=dphi/dt} rev.}
\end{equation}
where the derivative is \emph{not} with respect to $s$; $t$ is the asymptotic
time of the `far-away stars'.
\item
\begin{equation}
\Omega \to \Omega_\text{H},\qquad r \to r_+^+,
\end{equation}
which means observer on the horizon must co-rotate with it, which is an
extreme form of the Thirring--Lense effect.
\item
$\ul{u}$ is time-like, leading to restriction of allowed range for $\Omega$
(recall \cref{eq:Kerr-ulu})
\begin{align}
0 &\ggt \rbr{\ul{\xi} + \Omega\ul{\psi}}^2 =
g_{\mu\nu} \rbr{\xi^\mu + \Omega \psi^\mu} \rbr{\xi^\nu + \Omega \psi^\nu}
\nonumber \\
&= \xi_\mu \xi^\mu + 2\Omega g_{\mu\nu} \xi^\mu \psi^\nu +
\Omega^2 \psi_\mu \psi^\mu.
\end{align}
Inserting $g_{tt}$, $g_{t\phi}$ and $g_{\phi\phi}$ leads to
\begin{empheq}[box=\fbox]{equation}
\Omega_\text{min} < \Omega < \Omega_\text{max},
\end{empheq} % equation
where
\begin{align}
\Omega_\text{min}
&\coloneqq \omega - \sqrt{\omega^2 - \frac{g_{tt}}{g_{\phi\phi}}}, \\
\Omega_\text{max}
&\coloneqq \omega + \sqrt{\omega^2 - \frac{g_{tt}}{g_{\phi\phi}}}, \\
\omega \coloneqq -\frac{g_{\phi t}}{g_{\phi\phi}}
&= \frac{2Mra}{\rbr{r^2 + a^2}^2 - \Delta a^2 \sin^2\theta}
= \frac{\Omega_\text{min} + \Omega_\text{max}}{2}.
\end{align}
% end 04.09.2016
% begin 06.09.2016
For $g_{tt} < 0$ (outside ergosphere), $\Omega_\text{min} < 0$ (since
$g_{\phi\phi} > 0$), i.e.\ $\Omega = 0$ lies in the allowed range and static
observer is allowed; for $g_{tt} > 0$ (inside ergosphere), $\Omega_\text{min}
> 0$ and $\Omega = 0$ is impossible, i.e.\ observer \emph{must} co-rotate.
\item
Horizon:
\begin{equation}
\Delta = r_+^2 - 2Mr_+ + a^2 = 0;
\tag{\ref{eq:Delta=0} rev.}
\end{equation}
solving for $\omega$ gives rise to
\begin{align}
\omega &= \frac{2 M r_+ a}{\rbr{r_+^2 + a^2}^2} = \frac{2 M r_+ a}{4M^2 r_+^2}
= \frac{a}{2 M r_+} \nonumber \\
&= \Omega_\text{H} = \Omega_\text{min} = \Omega_\text{max}.
\end{align}
At the horizon, \emph{only} $\Omega_\text{H}$ is allowed! A locally inertial
observer rotates in the ergosphere with $\Omega = \rfun{\omega}{r, \theta}$.
\end{enumerate}
\end{namedrem} % Notes

\begin{nameddef}{Penrose process}
Because of the presence of the ergosphere, energy can be extracted from a
\emph{rotating} black hole \citeauthor{Penrose:1969pc}
\cite{Penrose2002,PENROSEFLOYD1971}\footnote{%
Possible astrophysical application: Blandford--Znajek mechanism
\cite{Blandford01071977}, possible explanation of \emph{%
jets} from active galactic nuclei (AGN).}.

Energy of a test particle:
\begin{equation}
E = - p_\mu \xi^\mu \equiv - g_{\mu\nu} p^\mu \xi^\nu = -g_{\mu t}p^\mu,
\end{equation}
which is conserved for geodesics. $E$ can become negative in the ergosphere,
since there $\xi_\mu \xi^\mu > 0$.

Viewed from above: \textbf{diagram}
Particle with energy $E_0$ is sent into the ergosphere; there breaks up into
two other particles, on of which falls into the black hole, the other escapes
to infinity.

Use \emph{energy conservation} for local process; in a freely falling system,
the physics is the same as in flat space-time.
\begin{equation}
p_0^\mu = p_1^\mu + p_2^\mu;
\end{equation}
contracting with $-\xi_\mu$ yields
\begin{equation}
E_0 = E_1 + E_2,
\end{equation}
in which $E_1$ can become negative, leading to $E_2 < E_0$. The wave analogue
would be super-radiance.

At infinity, one has $E_0 + \vbr{E_1}$; for the black hole, one has $M -
\vbr{E_1}$.
\end{nameddef} % Penrose process

\begin{namedrem}{Note}
The locally measured energy must always be \emph{positive} due to the
equivalence principle; $E_1 < 0$ then means that $E_1$ corresponds to a
spatial component of the four-momentum since observer in ergosphere cannot be
static, and cannot measure a negative energy. One needs
\begin{align}
0 &\lle -p^\mu \chi_\mu = -p^\mu\rbr{\xi_\mu + \Omega \psi_\mu} \nonumber \\
&= -p^\mu \xi_\mu - \Omega p^\mu \psi_\mu \eqqcolon E - \Omega L,
\end{align}
where $\chi_\mu \propto u_\mu$, $mF \equiv E \coloneqq -p^\mu \xi_\mu$,
$F = -\ul{\xi}\cdot\ul{u}$; $ml \equiv L \coloneqq p^\mu \psi_\mu$,
$l = \ul{\psi}\cdot\ul{u}$. One has
\begin{equation}
L \le \frac{E}{\Omega}.
\end{equation}
% end 06.09.2016
% begin 07.09.2016
If $E = E_1 < 0$: $L < 0$ since $\Omega > 0$. One has
\begin{enumerate}
\item
Particle with energy $E_1$ has negative angular momentum, so that through the
Penrose process, angular momentum is extracted from the black hole:
\begin{equation}
\dva J \le \frac{\dva M}{\Omega_\text{H}},
\end{equation}
where $\dva J ≙ L$, $\dva M ≙ E$.
\item
Angular momentum of black hole can be reduced to zero, where ergosphere
has disappeared.
\end{enumerate}
\end{namedrem} % Note

\begin{nameddef}{Area of Kerr black hole}
cf.\ exercises.
\begin{equation}
A = 4\pp \rbr{r_+^2 + a^2} \eqqcolon 16\pp\rbr{M_\text{irr}}^2,
\label{eq:area-Kerr}
\end{equation}
where $M_\text{irr}$ is the \emph{irreducible mass}; when $a = 0$, $M_%
\text{irr} = M$.

Recall $a = J/M$;
\begin{equation}
M_\text{irr}^2 = \frac{M^2 + \sqrt{M^4 - J^2}}{2} = \frac{Mr_+}{2},
\end{equation}
so that
\begin{equation}
M^2 = M_\text{irr}^2 + \frac{J^2}{4M_\text{irr}^2} \ge M_\text{irr}^2.
\label{eq:area-Kerr-1}
\end{equation}
If mass and angular momentum of the black hole are changed by $\dva M$
and $\dva J$ respectively, $M_\text{irr}$ changes by
\begin{equation}
\dva M_\text{irr}^2 = \frac{Mr_+}{\sqrt{M^2-a^2}}\rbr{\dva M -
\Omega_\text{H}\dva J},
\label{eq:area-Kerr-2}
\end{equation}
where $\dva M - \Omega_\text{H}\dva J \ge 0$ in the Penrose process.
\begin{proof}
\begin{equation}
\dva M_\text{irr}^2 = M\dva M + \frac{2M^3\dva M - J \dva J}%
{2\sqrt{M^4 - J^2}},
\end{equation}
in which
\begin{enumerate}
\item
$\dva M$-term:
\begin{equation}
M + \frac{M^3}{\sqrt{M^4 - J^2}} = \frac{M\sqrt{M^2 - a^2} + M^2}%
{\sqrt{M^2 - a^2}} = \frac{M r_+}{\sqrt{M^2 - a^2}};
\end{equation}
\item
$\dva J$-term:
\begin{equation}
-\frac{J}{2\sqrt{M^4-J^2}} = -\frac{J}{2M\sqrt{M^2-a^2}},
\end{equation}
where
\begin{equation}
\frac{J}{2M} = \frac{a}{2} = \frac{a}{2Mr_+} Mr_+ \equiv \Omega_\text{H} Mr_+.
\end{equation}
\end{enumerate}
\end{proof}
Combining \cref{eq:area-Kerr-1,eq:area-Kerr-2}, one concludes that through the
Penrose process, $M$ cannot be reduced below $\rfun{M_\text{irr}}{M_0, J_0}$.
% end 07.09.2016
% begin 21.09.2016

\textbf{diagram} Maximal reduction $M_0$ $\rfun{M_\text{irr}}{M_0, J_0}$.
Maximally extraction of rotation energy: $M_0 -
\rfun{M_\text{irr}}{M_0, J_0}$.

Extremal black hole: $J_0 = M_0^2$, so that
\begin{equation}
\rfun{M_\text{irr}^2}{M_0, J_0} = \frac{M_0^2}{2},
\end{equation}
and
\begin{equation}
M_0 - \rfun{M_\text{irr}}{M_0, J_0} = M_0\rbr{1-\frac{1}{\sqrt{2}}} \approx
0.29M_0,
\end{equation}
i.e.\ maximally $\approx\SI{29}{\percent}$ of the initial mass can be
extracted\footnote{Sun: maximally $\approx\SI{.6}{\percent}$ of its mass can
be extracted by nuclear fusion!}.

The relation \cref{eq:area-Kerr-2} can also be formulated by using $\dva A$
instead of $\dva M$. Recall
\begin{equation}
A = 16\pp\rbr{M_\text{irr}}^2,
\tag{\ref{eq:area-Kerr} rev.}
\end{equation}
\begin{equation}
\dva A = \frac{8\pp}{\kappa}\rbr{\dva M - \Omega_\text{H}\dva J},
\label{eq:delta-A-Kerr}
\end{equation}
where
\begin{equation}
\kappa\coloneqq\frac{M^2-a^2}{2Mr_+}
\end{equation}
is the \emph{surface gravity},
\begin{equation}
\kappa \to \frac{1}{2\rSch} = \frac{1}{4M} = \frac{M}{\rSch^2}\qquad a\to 0.
\end{equation}
From \cref{eq:delta-A-Kerr} one concludes that
\begin{empheq}[box=\fbox]{equation}
\dva A \ge 0,
\label{eq:delta-A-ge-0}
\end{empheq} % equation
or \emph{area of event horizon cannot decrease during the Penrose process}!
\cite{PhysRevLett.25.1596} \textbf{Citation correct?}
\end{nameddef} % Area of Kerr black hole

$\dva A \ge 0$ is valid much more generally.

\begin{namedthm}{Hawking's area law}
Assuming the weak energy condition\footnote{$R_{\mu\nu} k^\mu k^\nu \ge 0,
\forall k^\mu$ light-like.}, the area of a future event horizon in an
asymptotically flat space-time is non-decreasing. \cite{Hawking1972}
\end{namedthm} % Hawking's area law

\begin{namedrem}{Remark for the theorem}
\begin{itemize}
\item
Presence of event horizon assumed (cosmic censorship!)
\item
Time direction comes into play by referring to the \emph{future} horizon.
\item
For white holes, one has non-increasing horizon.
\end{itemize}
\end{namedrem} % Remark for the theorem

\textbf{diagram: Coalescence of two black holes}, $A_3 \ge A_1 + A_2$.

\begin{nameddef}{Black hole thermodynamics}
In thermodynamics, one has non-decreasing \emph{entropy}
\begin{empheq}[box=\fbox]{equation}
\dva S \ge 0
\end{empheq}
for isolated systems, analogue with \cref{eq:delta-A-ge-0}. Not too far
fetched?

There are indeed analogies to \emph{all} Laws of thermodynamics \cite%
{Bardeen1973}
\begin{itemize}
\item[0th Law]
Surface gravity $\kappa$ is constant on the event horizon, i.e.\ $\kappa ≙ T$;
\item[1st Law] In Penrose process,
\begin{equation}
\dd M = \frac{\kappa}{8\pp}\dd\, A + \Omega \dd\, J - \rbr{\Phi \dd\, q},
\label{eq:bh-first-law}
\end{equation}
analogous with
\begin{equation}
\dd H = T\dd\, S + \Omega \dd\, J - \rbr{\Phi \dd\, q}
+ \rbr{V\dd\, p + \mu\dd\, N},
\end{equation}
where $A ≙ S$. Note the $V\dd\, p$ as well as $\mu\dd\, N$ terms are missing
in \cref{eq:bh-first-law}.
\item[2nd Law]
$\dva A \ge 0$
\item[3rd Law]
$\kappa = 0$ cannot be reached in a finite number of steps (cf.\ Nernst 1905
\textbf{which paper?} with $\kappa \to T$);

Stronger version by Planck 1911 \textbf{which paper?}
\begin{equation}
S \to 0\qquad T \to 0,
\end{equation}
which can be visited there and here;
\begin{equation}
S\to \frac{A}{4\pll^2},\qquad \kappa\to 0.
\end{equation}
\end{itemize}
% end 21.09.2016
\end{nameddef} % Black hole thermodynamics

% begin 23.09.2016
\begin{namedthm}{Uniqueness (No-hair) Theorem for stationary black holes}
Stationary black holes are uniquely characterised in Einstein--Maxwell theory
by the parameters $M$, $J$, $q$, and $q_\text{M}$ if there are magnetic
monopoles.
\cite{Heusler1996,lrr-2012-7}
\end{namedthm} % Uniqueness (No-hair) Theorem for stationary black holes

\begin{unamedrem}
Interpretation of analogy with thermodynamics only by brining quantum theory
into play\footnote{see lecture notes on Quantum Field Theory in Curved
Space-Time.}, with \emph{Hawking temperature} \cite{HAWKING1974}
\begin{equation}
T_\text{H} = \frac{\phs\kappa}{2\pp\bk\lc}
\end{equation}
and Bekenstein--Hawking entropy \cite{PhysRevD.7.2333}
\begin{equation}
S_\text{BH} = \frac{A\lc^3}{4\nG\phs} = \rbr{\frac{A}{2\pll}}^2.
\end{equation}
For a Schwarzschild black hole, these read
\begin{align}
T_\text{H} &= \frac{\phs\lc^3}{8\pp\bk\nG M}\approx
\num{6.17e-8}\rbr{\frac{M_\astrosun}{M}}\,\si{\kelvin},\\
S_\text{BH} &= \bk\frac{\pp\rSch^2}{\nG\phs} \approx
\num{1.07e77}\rbr{\frac{M_\astrosun}{M}}^2\bk.
\end{align}
\end{unamedrem}

\subsection{Gravitational collapse to black holes}

\begin{nameddef}{ }
Equilibrium possible by degeneracy pressure from electrons (to White Dwarfs)
respectively neutrons (to Neutron Stars) only possible for $M \lesssim 3
M_\astrosun$. For bigger masses, star will collapse to a black hole\footnote{%
within General Relativity; alternative scenarios in speculative physics.}.

Assume here \emph{spherical symmetry} for simplicity, so that Schwarzschild
black holes will be formed. Consider density of an object at $R = \rSch = 2\nG
M/\lc^2$:
\begin{equation}
R = \rbr{\frac{3M}{4\pp\rho}^{1/3}} \eeq \frac{2\nG M}{\lc^2},
\end{equation}
where $\rho = \text{const.}$; justification for flat coordinates? e.g.\
Painlevé--Gullstrand coordinate \cite{painleve1921mecanique,%
Gullstrand:1922tfa}. One has
\begin{equation}
\rho = \frac{3\lc^6}{32\pp\nG^3 M^2}\approx \num{2e16}\rbr{\frac{M_\astrosun}%
{M}}^2\si{\gram\per\cubic\centi\metre}.
\end{equation}
\begin{itemize}
\item
$M \approx M_\astrosun$, where $\rho$ is high, so that complicated pressures
during collapse are expected, under which non-gravitational interactions
become important.
\item
$M\sim\num{e11}M_\astrosun$, which is at the galaxy scale and $\rho\sim\SI{%
e-6}{\gram\per\cubic\centi\metre}$, so that non-gravitational interactions
are negligible and collapse can be considered as \emph{free fall} ($p=0$).
In this case one gets a simple model.
\end{itemize}

First model by \citeauthor{PhysRev.56.455} \cite{PhysRev.56.455}: collapse
of a pressure-less star (outside Schwarzschild; inside Friedmann or TOV).

\textbf{diagram:} stellar radius
% end 23.09.2016
% begin 30.09.2016

Geodesic equation: after finite proper time, the surface falls through $\rSch$
and reaches $r = 0$ after
\begin{equation}
s = \pp\sqrt{\frac{R^3}{8\nG M}},
\end{equation}
which is called \emph{free-fall time} in astronomy.
For the external observer, on the contrary, the crossing of $\rSch$
corresponds to $t\to+\infty$!

For $\rho = \text{const.}$ so that $R^3 = 3\nG M/4\pp\rho$,
\begin{align}
s &= \frac{\pp}{2}\sqrt{\frac{3}{8\pp\nG M\rho}} \\
&\approx \num{2e3}\frac{\rho}{\si{\gram\per\cubic\centi\metre}}\si{\second},
\end{align}
which depends only on $\rho$. For example, for $\rho\sim\SI{1}{\gram\per%
\cubic\centi\metre}$, $s\sim\SI{1}{\hour}$ \textbf{some unrecognisable script}

\end{nameddef} % (null)

To appropriately study collapse, introducing the ingoing version of
\begin{nameddef}{Edditon--Finkelstein coordinates}
First, as before, one transforms the radial $r$ to
\begin{equation}
r_* = r + 2\nG M \rfun{\ln}{\frac{r}{2\nG M} -1},
\tag{\ref{eq:tortoise-r} rev.}
\end{equation}
so that the out- and in-going radial null geodesics propagate on
\begin{align}
\tld{u} &\coloneqq t - r_* = \text{const.}, \\
\tld{v} &\coloneqq t + r_* = \text{const.},
\end{align}
respectively, which shows the advantage of chossing $r_*$.

Second, choose $\rbr{r, \tld{v}}$ instead of $\rbr{r, t}$ so that ingoing rays
lie on straight lines, and the metric takes the form
\begin{equation}
\dd s^2 = -\rbr{1-\frac{2\nG M}{r}}\dd\,\tld{v}^2 + 2\dd\,\tld{v}\dd\, r
+ r^2\dd\,\Omega^2,
\end{equation}
where the second term on the right hand side avoid the coordinate singularity
at the future horizon.
\end{nameddef} % Edditon--Finkelstein coordinates

\begin{unamedrem}
$r\to+\infty$ gives Minkowski space in null coordinate $\tld{v}
\equiv v = t+r$ and $\dd s^2 = -\dd\,\tld{v}^2 + 2\dd\,\tld{v}\dd\, r
+ r^2\dd\,\Omega^2$.
\end{unamedrem}

\begin{nameddef}{Radial light rays}
\begin{equation}
\dd s^2 = 0;\qquad \dd \theta = 0 = \dd \phi,
\end{equation}
so that
\begin{equation}
\rbr{1-\frac{2\nG M}{r}}\dd\,\tld{v}^2 = 2\dd\,\tld{v}\dd\, r.
\end{equation}

Solutions:
\begin{enumerate}
\item
$\dd\tld{v}/\dd r = 0$, so that $\tld{v} = \text{const.}$, which are radially
ingoing light rays;

\item \label{item:vtilde}
$\dd\tld{v}/\dd r \ne 0$, then one has
\begin{equation}
\frde{\tld{v}}{r} = \frac{2}{1-2\nG M/r}.
\end{equation}
Integration yields
\begin{align}
\tld{v} &= 2\rbr{r+2\nG M\rfun{\ln}{\frac{r}{2\nG M} -1}} + \text{const.} \\
&=2r_* + \text{const.} \equiv t+r_*,
\end{align}
or $r_* = t+\text{const.}$ for outgoing light rays.

\item
$\dd r = 0$: $r = 2\nG M \equiv \rSch$ is also a solution.
\end{enumerate}
% end 30.09.2016
% begin 11.10.2016

For \cref{item:vtilde},
\begin{enumerate}
\item
\[\frde{\tld{v}}{r} \to 2^- \qquad r \to +\infty;\]
writing
\begin{equation}
\tld{t}\coloneqq \tld{v} - r
\end{equation}
gives
\begin{equation}
\frde{\tld{t}}{r} = \frde{\tld{v}}{r} -1 \to 1^-.
\end{equation}

\item 
\[\frde{\tld{v}}{r} \to \pm\infty \qquad r \to 2\nG M,\]
so that
\begin{equation}
\frde{\tld{t}}{r} \to \pm\infty.
\end{equation}

\item 
\[\frde{\tld{v}}{r} \to 0^- \qquad r \to 0^+,\]
so that
\begin{equation}
\frde{\tld{t}}{r} \to -1^-.
\end{equation}
\end{enumerate}
\textbf{diagram: $\tld{t} = \tld{v} - r$. Ingoing light rays.}

There is also an increasing \emph{redshift} (exercise?)
\begin{equation}
z \coloneqq \frac{\Delta\lambda}{\lambda} = \ee^{t/4\nG M}
\end{equation}
for radial light rays.

\emph{Decrease in luminosity} from red-shift and photon arrives in ever
increasing time intervals \cite{Ames1968}
\begin{equation}
L \propto L_0\rfun{\exp}{-\frac{t}{3\sqrt{3}\nG M}}.
\end{equation}
Characteristic time:
\begin{equation}
t_\text{decrease}\approx 3\sqrt{3}\nG M
\sim\num{2.5e-5}\rbr{\frac{M}{M_\astrosun}}\si{\second},
\end{equation}
which means star becomes rapidly invisible!

\end{nameddef} % Radial light rays

\begin{namedrem}{Remarks on general gravitational collapse}
See e.g.\ \cite{StephenHawking2015,Hawking2000Raum}

Singularity only for high symmetry of collapsing object?

No, is much more general (Penrose, Hawking, etc., from 1965 on)

``Singularity theorems''.

Typical assumptions for the proofs:
\begin{enumerate}
\item An energy condition
\label{en:energy-cond}
\item A condition on the global structure
(e.g.\ non-existence on closed time-like curves)
\item Existence of a `trapped surface'
\label{en:trapped-surf}
\end{enumerate}
% end 11.10.2016
% begin 19.10.2016

\begin{nameddef}{ }
A space-time is called \emph{singular} if it is time-like- or null-geodesically
incomplete and if it cannot be embedded in a bigger space-time.
(can be generalised to non-geodesic motion)
\end{nameddef} %

\begin{namedrem}{About \cref{en:energy-cond}}
\begin{enumerate}[label=\alph*]
\item
\emph{weak energy condition}
\begin{equation}
T_{\mu\nu}\xi^\mu\xi^\nu \ge 0\qquad \forall \text{ time-like } \xi^\mu,
\end{equation}
so that observer with four-velocity $\xi^\mu$ measures non-negative energy.

For ideal fluid
\begin{align}
T_{\mu\nu} &= \rbr{\rho+p}u_\mu u_\nu + pg_{\mu\nu} \\
&\overset{*}{=} \rfun{\diag}{\rho,p,p,p}\qquad\text{isotropic case} \\
&\overset{*}{=} \rfun{\diag}{\rho,p_1,p_2,p_3}
\qquad\text{anisotropic case} \\
\end{align}
where $\rho \ge 0$; $\rho + p_i \ge 0$, $i = 1, 2, 3$, which follows after some
calculation with a general time-like vector.

\item \emph{strong energy condition}, typically used for singularity theorems.
\begin{equation}
T_{\mu\nu}\xi^\mu\xi^\nu \ge -\frac{1}{2} g^{\mu\nu}T_{\mu\nu}
\qquad \forall \text{ time-like } \xi^\mu.
\end{equation}
For ideal fluid, $\rho + \sum_{i = 1}^3 p_i \ge 0$ and $\rho + p_i \ge 0$, $i
 = 1, 2, 3$. $\rho + 3p \ge 0$ in isotropic case.

\item \emph{dominant energy condition}, velocity of energy flux for matter
(local sound velocity) no more than the speed of light, leading to
\begin{equation}
\rho \ge \vbr{p_i},\qquad i = 1, 2, 3.
\end{equation}
More precisely, for all future directed time-like $\xi^\nu$, $-T^\mu{}_\nu
\xi^\nu$, which is the energy-momentum four-current density, should be a future
directed time-like or null vector. One has: dominant energy condition implies
weak energy condition, but otherwise the conditions are independent.

\end{enumerate}
\end{namedrem} % About \cref{en:energy-cond}

\begin{namedrem}{About \cref{en:trapped-surf}}
Trapped surface bounds a region where both in- \emph{and} out-going light rays
converge, already the case in Kruskal space-time.
\end{namedrem} % About \cref{en:trapped-surf}

Naked singularities? We already had: uniqueness theorems (assuming cosmic
censorship); collapse ends on a Kerr black hole (for $q = 0$).

\end{namedrem} % Remarks on general gravitational collapse
% end 19.10.2016
% begin 31.10.2016
\subsection{Astrophysical black holes}

\subsubsection{Stellar black holes}
Typically: `observable' as partner of a visible star (most stars are in multiple
systems)

\paragraph{Accretion disk} gas flows from visible star to black hole (angular
momentum of visible star!) Inspiral up to the innermost stable circular orbit,
then radial infall

Inspiral because of dissipative processes, emission of X-rays (for supermassive
black holes, also \emph{radio} emission)

Gravitational energy of infalling matter transformed into X-ray luminosity
(historically: Cygnus X-1 1971 \cite{Kristian1971})

How can black holes be distinguished from other dark compact objects (e.g.\
neutron stars)?
\begin{enumerate}
\item event horizon instead of rigid surface, where infalling material can
bounce back
\item mass ($M \gtrsim 3 M_\astrosun$)
\end{enumerate}

\paragraph{Example} approximately circular orbits and Newtonian dynamics,
which is sufficient for mass determination. Good approximations are often.

\textbf{diagramme}

\begin{empheq}[box=\fbox]{equation}
v_{1,2} = \frac{2\pp a_{1,2}}{P},
\end{empheq}
where $v_1$ is observable, $P$ is the period.

\begin{equation}
a\coloneqq a_1 + a_2,\qquad M\coloneqq M_1 + M_2
\end{equation}

\textbf{diagramme}

$v_\text{ob}$ determined from periodic Doppler shift
\begin{equation}
v_\text{ob} = v_1 \sin i = \frac{2\pp a_1}{P} \sin i = \frac{2\pp a}{P}
\frac{M_2}{M}\sin i,
\end{equation}
where $a_1 = a M_2/M$. Kepler III:
\begin{equation}
\nG M = \rbr{\frac{2\pp}{P}}^2 a^3.
\end{equation}
Eliminating the unknown $a$ yields
\begin{equation}
v_\text{ob}^3 = \frac{2\pp}{P} \rbr{M_2 \sin i}^3 \frac{\nG}{M^2},
\end{equation}
or
\begin{equation}
\frac{v_\text{ob}^3}{2\pp\nG} = \frac{\rbr{M_2 \sin i}^3}{M^2}.
\end{equation}

Define the `\emph{mass function}'
\begin{align}
\rfun{f}{M_1, M_2, i} &\coloneqq \frac{\rbr{M_2 \sin i}^3}{M^2}
\label{eq:mass-fun-info}\\
&= \frac{v_\text{ob}^3}{2\pp\nG},
\label{eq:mass-fun-obs}
\end{align}
where one gets information about the combination from \cref{eq:mass-fun-info},
while the observables quantities are in \cref{eq:mass-fun-obs}. Estimate of
$M_1$ from spectroscopy.

Bound:
\begin{equation}
f \le \frac{M_2^3}{\rbr{M_1+M_2}^2} < \frac{M_2^3}{M_2^2} = M_2.
\end{equation}
$f$ yields lower bound for $M_2$, safe black hole candidate if $f > 3
M_\astrosun$.

Sometimes one can get information about $\sin i$ (e.g.\ $i \approx \pp/2$ from
variabilities due to orbital eclipses)

At present, $> 50$ candidates for stellar black holes in our Milky Way, where
perhaps $> \num{e5}$ black holes exist.

Distinction into high mass X-ray binaries (HMXB, companion is of high mass) and 
low mass X-ray binaries (LMXB, companion $\lesssim 1 M_\astrosun$) -> `X-ray 
transients':
flare up to high luminosities.
% end 31.10.2016
% begin 03.11.2016

Cygnus X-1: HMXB, $M_* \approx \numrange{24}{42}\,M_\astrosun$, $f\approx 
\num{.25}$, $M_\text{BH}\approx\numrange{11}{21}\,M_\astrosun$;

V404 Cygnus: LMXB, $M_* \approx \num{.6}\,M_\astrosun$, $f\approx 
\num{6.07}$, $M_\text{BH}\approx\numrange{10}{15}\,M_\astrosun$. It is a 
`variable star' which flares (swallows material from $M_*$) in irregular 
periods: 1938, 1956, 1989, 2015; also called `microquasar'.

\paragraph{Intermezzo} `Eddington luminosity' or `Eddington limit', which is 
important for size of accretion disk for both stellar and supermassive black 
holes.

Need: pressure of outgoing radiation $\lesssim$ gravitational attraction 
(otherwise accretion disk resolves)

\paragraph{Rough estimate} Be $L$ the luminosity of the object

$\xrightarrow\quad L/4\pp r^2$: energy flux through surface with area $4\pp r^2$

$≙ L/4\pp\lc r^2$: Momentum flux ($E = p\lc$)

Scattering of this (outgoing) radiation on infalling matter: estimate momentum 
transfer on matter via \emph{Thomson scattering} (scattering of low-energy 
photons at electrons):
\begin{equation}
\sigma_\text{T} = \frac{8\pp}{3}\rbr{\frac{\ec^2}{m_e \lc^2}}^2 \approx
\SI{.665e-24}{\square\centi\metre},
\end{equation}
where $\ec^2/m_e \lc^2$ is the classical electron radius from $\ec^2/r = m_e 
\lc^2$.

So the momentum transfer on electron at radius $r$ per unit time:
\begin{equation}
\sigma_\text{T}\times\text{momentum flux:}\qquad
\frac{\sigma_\text{T} L}{4\pp\lc r^2},
\end{equation}
with dimension of `force'.

Set this `force' equal to gravitational force:
\begin{equation}
\frac{\nG m_\text{pr} M}{r^2} \eqqcolon \frac{\sigma_\text{T} 
L_\text{Edd}}{4\pp \lc r^2},
\end{equation}
where the left-hand side $\approx \num{1}$ nucleon per electron. One derives
\begin{equation}
L_\text{Edd} = \frac{4\pp \nG \lc m_\text{pr} M}{\sigma_\text{T}}
\approx \num{1.3e38}\,\frac{M}{M_\astrosun}\si{erg\per\second},
\end{equation}
which can be compared with $L_\astrosun \approx \SI{4e33}{erg\per\second} 
\equiv \SI{4e25}{\W}$.


\subsubsection{Supermassive black holes}

1960s: postulated to explain the gigantic energy emission from `active galactic 
nuclei' (AGN), which models quasars, radio galaxies, etc.
(Novikov \& Zel'dovich)
% end 03.11.2016
% begin 08.11.2016

Observed are \SIrange{e44}{e48}{erg\per\second} ($≙ \numrange{e11}{e15} 
\,\L_\astrosun$)!

$L < L_\text{Edd}$: $M > \numrange{e6}{e10}\,M_\astrosun$ (NGC 1277: $> 
\num{e10}\,M_\astrosun$)

Often observed: variabilities of very short time ($\sim \SI{1}{hour}$), so that
mass is concentrated in small region, and thus only black holes fulfil this.

Perhaps there is an SMBH in the centre of every galaxy (comprising $\approx 
14\,\%$ of the stellar mass of the elliptical part [bulge] of the galaxy)

Up to now more than \num{50} candidates, e.g.\ M87 (elliptical galaxy in the 
centre of the Virgo cluster): $M \approx \num{6.6e9}\,M_\astrosun$ (bulge mass 
$\sim \num{e12}\,M_\astrosun$); M106 (NGC4258) (best studied example outside 
Milky Way): \ce{H2O} \textbf{unrecognisable} laser emission at $\lambda = 
\SI{1.35}{\centi\metre}$, so that radio interferometer; $M \approx 
\num{3.6e7}\,M_\astrosun$; NGC 1332: $M\approx \num{6.6e8}\,M_\astrosun$ (Alma 
in Chile), and from motion of gas cloud around the SMBH.
% end 08.11.2016
% begin 14.11.2016

$M_\text{SMBH}$ correlates with bulge masses (merger origin?)

Best studied SMBH: black hole in the centre of our own Galaxy (Sagittarius A*, 
Sgr A*), which is a compact radio source; distance $\approx \SI{8}{kpc}$.

From influence on orbiting stars, one infers its mass to be
\begin{equation}
M\approx \num{4.3(3)e6}\,M_\astrosun.
\end{equation}

Observation of periodic signals from Sgr A* (plus width of an iron spectral 
line) from very close to the black hole suggests that the black hole seems to 
\emph{rotate} (limit of observable region given by the innermost stable 
circular obit, with radius less than the Schwarzschild radius of a non-rotating 
hole)

Kerr:
\begin{equation}
a^* = \frac{a}{M} = \frac{J\lc}{\nG M^2},
\end{equation}
and observations indicate that $a^* \gtrsim \num{.4}$. Direct check of Kerr 
metric?

(Mention also gas cloud that approached the Galactic black hole in 2015)

Theoretical upper limit for luminous SMBH:
\begin{equation}
M_\text{max} \approx \num{5e10}\,M_\astrosun,
\end{equation}
above which luminous accretion cannot preceed, since ISCO outside 
self-gravitation of the disk.

\subsubsection{Primordial black holes?}

Not yet observed; speculative.

From \emph{density fluctuation} in the early Universe
\begin{equation}
\rho = \rho_\text{c} + \delta_\rho,
\end{equation}
where $\rho_\text{c}$ is the cosmological critical density. Collapse of this 
fluctuation if potential energy is larger than the inner energy caused by 
pressure \textbf{diagram}
\begin{equation}
U\sim \frac{\nG M^2}{R} \sim \nG \rho^2 R^5 \gge p R^3,
\end{equation}
so that
\begin{equation}
R^2 \ge \frac{p}{\nG \rho^2}.
\end{equation}
Assume $p = w\rho$ ($w = 1/3$ in the radiation-dominated early Universe):
\begin{equation}
R \ge \sqrt{w}\cdot\frac{1}{\sqrt{\nG \rho}},
\end{equation}
where $\rbr{\nG \rho}^{-1/2}$ is the \emph{Jeans length}.

Upper bound: $R \lesssim \rbr{\nG \rho}^{-1/2}$ from the demand that $R < $ 
curvature radius of over-dense region.

From cosmology ($\rho_\text{PBH} \approx \rho_\text{c}$), one has
\begin{equation}
M_\text{PBH}\,\sbr{\mathrm{g}} \sim \num{e38}t\,\sbr{\mathrm{s}}
\end{equation}
in a radiation-dominated Universe,
\begin{align}
\rfun{M_\text{PBH}}{t}
&\approx M_\astrosun \rbr{\frac{t}{\SI{5e-6}{\second}}} \nonumber \\
&\approx M_* \rbr{\frac{t}{\SI{e-24}{\second}}},
\end{align}
where $M_* \approx \SI{5e14}{\gram}$, which should have been evaporated by now.

Remark on Fermi Gamma-ray Space Telescope, launch: June 2008

\subsubsection{Black holes from Gravitational-wave observations}
Recall Gravitational-wave emission from binary stars $\rbr{M_1, M_2}$. 
Quadrupole formula leads to the amplitude (for $\epsilon \lesssim \num{.2}$)
\begin{align}
h &= 8\rbr{\frac{2}{15}}^{1/2} \frac{\nG^{5/3}\mu}{r\lc^4} \rbr{\pp M f}^{2/3}
\nonumber \\
&= \num{8.7e-21} \rbr{\frac{\mu}{M_\astrosun}} \rbr{\frac{M}{M_\astrosun}}^{2/3}
\rbr{\frac{\SI{100}{pc}}{r}} \rbr{\frac{f}{\SI{e-3}{\Hz}}}^{2/3},
\end{align}
where $\mu = M_1 M_2/M$, $f = 2$ \textbf{cannot read}

Second post-Newtonian (2PN) order: take $\rbr{\nG M/r\lc^2}^2 \sim 
\rbr{v/\lc}^4$
into account, the \emph{chirp mass}
\begin{equation}
\mscrM\coloneqq\frac{\rbr{M_1 M_2}^{3/5}}{\rbr{M_1 + M_2}^{1/5}}
= \frac{\lc^3}{\nG}\rbr{\frac{5\dot{f}}{96\pp^{8/3}f^{11/3}}}^{3/5},
\end{equation}
has been observed.

Event GW150914: direct Gravitational-wave detection with LIGO 
\cite{PhysRevLett.116.061102}.

$f$, $\dot{f}$ observed over $\SI{.2}{\second}$, the signal increases from 
$\num{35}$ to $\SI{150}{\Hz}$; one infers that $\mscrM \approx 30 M_\astrosun$, 
so that $M = M_1 + M_2 \gtrsim 70 M_\astrosun$.

\begin{quote}
This leaves black holes as the only known objects compact enough to resolve an 
orbital frequency of \SI{75}{\Hz} without contact.
\end{quote}
\begin{quote}
\dots decay of waveform consistent with damped oscillation of a black hole 
relaxing to a final stationary Kerr configuration.
\end{quote}

Black hole masses:
% end 14.11.2016
% begin 15.11.2016
\begin{align}
M_1 &= 36\substack{+5 \\ -4}\,M_\astrosun,\\
M_2 &= 29\substack{+4 \\ -4}\,M_\astrosun,\\
M_\text{BH, final} &= 62\substack{+4 \\ -4}\,M_\astrosun,\\
a^*_{BH, final} &= \num{.67}\substack{+\num{.05} \\ -\num{.07}}\,M_\astrosun,\\
d_L &= \num{410}\substack{+\num{160} \\ -\num{180}}\,\si{Mpc}.
\end{align}
$E_\text{tot}$ in GW:
\begin{equation}
3.0\substack{+\num{.5} \\ -\num{.5}}\,M_\astrosun \lc^2 !!
\end{equation}

\url{http://www.faz.net/aktuell/wissen/weltraum/%
binaersystem-v404-cygni-ein-schwarzes-loch-erwacht-13690122.html}
\url{http://www.faz.net/aktuell/wissen/weltraum/die-masse-eines-schwarzen-loch-%
in-ferner-galaxie-praezise-bestimmt-14224214.html}

% end 15.11.2016