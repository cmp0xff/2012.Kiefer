\section{Differential Geometry 2: Modern Formulation}
\label{sec:diff-geo-2}
The modern formulation is mainly based on the work by Élie Cartan
(1869 --- 1951) in the 1920s.

\subsection{Tangent space and tangent vectors}
\begin{figure}
\centering
\includegraphics{graphics/10/Diag1}
\caption{Tangent space\label{fig:tangent-space}}
\end{figure}
Up to now, vectors have been given with respect to a \emph{coordinate system},
also called \emph{natural} or \emph{holonomic} basis. It is often more
convenient to introduce \emph{general basis}, or \emph{anholonomic} basis, if
they were not provided by a coordinate system. The reasons to do this include
\begin{itemize}
\item	Adaptation to symmetries
\item	Can always choose orthonormal basis
\item	Necessity for spinor fields (formulation of Dirac equation)
\end{itemize}

To begin with, we construct a tangent vector at $P$ in a coordinate and basis
independent way, see \cref{fig:construct-t-vec}.
Let $F$ be a $C^\infty$ function, mapping $\mathcal{M}^n$ to
$\mathbb{R}$, defined in a neighbourhood $U$ of $P$, and being independent of
coordinate system. Let $\mathcal{F}$ be the space of all such functions. Let
$f = F\circ \sigma^{-1} :\quad \mathbb{R}^n \to \mathbb{R}$ be the coordinate
expression for $F$.

\begin{figure}
\centering
\includegraphics{graphics/10/Diag2}
\caption{Construction of a tangent vector at $P$ in a coordinate and basis
independent way\label{fig:construct-t-vec}}
\end{figure}

\begin{defn}
A \emph{tangent vector} $v$ at $P$ is a \emph{linear map} \[\mathcal{F} \to
\mathbb{R},\qquad F \mapsto v(F),\] which obeys the chain rule. That is, for
\[G = g\left(G_1, G_2, \ldots\right),\qquad
G,\,G_i \in \mathcal{F}, \rfun{g}{G_i}\qquad \text{smooth},\]
\begin{equation}
v(G) = \sum_i \left.\frpa{g}{G_i}\right|_P \rfun{v}{G_i}.
\label{eq:chain-rule}
\end{equation}
\end{defn}
In components, $\rfun{v}{G}$ is something like $G_{1k}v^k$, where $G =
\rfun{g}{G_1}$, $G_{1k} = \frpa{g}{G_1}G_{1,k}$, so that $G_{1k}v^k =
\frpa{g}{G_1}G_{1,k}v^k$.
\begin{defn}
If \cref{eq:chain-rule} is fulfilled by $u$ and $v$, then it would also be
fulfilled by $α u + β v$, $\forall α,\,β \in\mathbb{R}$, so
that these tangent vector span a \emph{vector space}, called the \emph{tangent
space} at $P$, denoted by $T_P$.
\end{defn}


Next we introduce the \emph{frame} at $P$. Given the coordinate system $\sigma$, one can apply the chain rule to $F\circ\sigma^{-1} = f\left(x^1, x^2, \ldots\right)$:
\begin{align}
\rfun{v}{F} &= \rfun{v}{f\rbr{x^1, x^2, \ldots}} = \fat{\sum_μ\frpa{f}{x^μ}}{P} v(x^μ) \nonumber \\
& \equiv v^μ\fat{\frpa{f}{x^μ}}{P}.
\end{align}
Here $v^μ$'s are components with respect to the coordinate system $\sigma$, and $\fat{\frpa{f}{x^μ}}{P}$ defines $n$ tangent vectors
\begin{equation}
\fat{\partial_μ}{P}\rbr{F} \coloneqq \fat{\frpa{f}{x^μ}}{P}.
\label{eq:def-partial}
\end{equation}
It is now clear that $\rfun{v}{f\rbr{\ldots}}$ serves as a `directional derivative', and we can write down
\begin{empheq}[box=\fbox]{equation}
\fat{v}{P} \equiv v^μ \fat{\partial_μ}{P}.
\end{empheq}

\begin{rem}
$\fat{\partial_μ}{P}$ is the \emph{notation} for a tangent vector, which is \emph{defined} by \cref{eq:def-partial}. When viewed as $n$ basis vectors, $\cbr{\fat{\partial_μ}{P}}$ are a special \emph{$n$-Bein} (literally $n$-leg in German) or \emph{frame} at $P$.
\end{rem}

\begin{defn}
A \emph{Vector field} is vectors at each point.
\end{defn}

\begin{defn}
An \emph{$n$-Bein field} is a frame field. It is denoted by
$\cbr{\partial_μ}$ if provided by a coordinate system, and is also named as
\emph{holonomic\footnote{The word is composed from the Greek ὅλος meaning
\textit{whole, entire} and νόμ-ος meaning \textit{law}} frame.} in this case.
In general, it is denoted by $\cbr{e_μ}$.
\end{defn}

The Lie bracket of a general $n$-Bein field $\cbr{e_μ}$ satisfies
\begin{equation}
\sbr{e_μ, e_ν} \eqqcolon C^ρ_{μν} e_ρ,
\end{equation}
where $C^ρ_{μν}$ is called \emph{anholonomy} or \emph{structure
coefficients}. Obviously, $\sbr{e_μ, e_ν} = 0$ is necessary for the basis
being holonomic\footnote{Locally, this is also sufficient.}.

\subsection{Differentials and cotangent space}

Define the following linear functional on $T_P$
\begin{equation}
\dd f \coloneqq f_{,μ}u^μ = u\rbr{f},
\label{eq:def-df}
\end{equation}
which has the meaning of a \emph{differential}; see \cref{defn:ext-der}.
It associates with each
$u \in T_P$ the corresponding directional derivative of $f$. Note that
$\dd f$ is \emph{not} an \emph{infinitesimal}, but formally has this property.

\begin{defn}
The linear functionals over a vector space $V$ (here $T_P$) with
values in $\mathbb{R}$ form the \emph{dual vector space} $V^*$, here
called the \emph{cotangent space} $T_P^*$.
\end{defn}

Consider $a \in V^*$ (so that $a\cbr{v}\in\mathbb{R}$), $u,\,v \in V$,
$α,\,β\in\mathbb{R}$.
\begin{defn}
The \emph{interior product} is defined as
\begin{equation}
a\rbr{v} \equiv \iota_va\rbr{v} \equiv v \intprod a.
\end{equation}
\end{defn}
Sometimes this is also denoted as $\abr{a, v}$, but there is a danger of
confusion with \emph{scalar product}, which is a more special notion, because
it needs a metric.

Next we construct the dual basis in $V^*$. Acting
$a\in V^*$ on $v$,
\begin{equation}
a\rbr{v} = a\rbr{v^μ e_μ} = v^μ a\rbr{e_μ}
\coloneqq a\rbr{e_μ} θ^μ\rbr{v},
\end{equation}
where $θ^μ$ is defined such that $θ^μ\rbr{v} = v^μ$. Now
$\forall a\in V^*$ can be decomposed as
\begin{equation}
a = a\rbr{e_μ}θ^μ \coloneqq a_μ θ^μ,
\end{equation}
where $\cbr{θ^μ}$\footnote{Also called $\cbr{e^μ}$ or
$\cbr{ω^μ}$ in the literature.} is called the \emph{dual basis} of
$V^*$. When viewed as vector field, it is also called the \emph{dual frame}.

$a\rbr{v} = a\rbr{e_μ} θ^μ\rbr{v} = a_μ v^μ$, where the lower
$_μ$ is called \emph{covariant} index, and the upper $^μ$ 
\emph{contravariant} index. In particular,
\[
θ^μ\rbr{e_ν} = θ^μ\rbr{e_ρ}θ^ρ\rbr{e_ν},
\]
so that
\begin{empheq}[box=\fbox]{equation}
\rfun{θ^ρ}{e_ν} = δ^ρ_ν
\label{eq:dual-basis}
\end{empheq}
must hold.

Consider \cref{eq:def-df}. Choose $f = x^ν$,
\begin{equation}
\dd x^ν\rbr{v} = \rfun{v}{x^ν} = v^ν = θ^ν\rbr{v},
\end{equation}
so that $\cbr{\dd x^ν}$ is the basis dual to $\cbr{\partial_ν}$. Combining
\cref{eq:dual-basis} yields
\begin{equation}
\rfun{\dd x^ρ}{\partial_ν} = δ^ρ_ν.
\end{equation}

$a\in T_P^*$: $a = a_μ \dd\, x^μ$ (compare $v = v^μ \partial_μ$).
Covector field: $\rfun{a}{x} = \rfun{a_μ}{x}\dd\, x^μ$.
In general frame: $a = a_μ θ^μ$, where $a_μ$ is not equal to an
$f_{,μ}$ for an anholonomic basis.

\subsection{Tensor algebra and metric}

\textbf{Hey what is going on here, sometimes tensor sometimes Cartesian 
product.}

$v\in V$, $w\in W$; $a\in V^*$, $b\in W^*$.
The \emph{tensor product} $v\otimes w \in V\otimes W$. The tensor
product defined on $V^*\times W^*$ is
\begin{align}
\rbr{v\times w}\rbr{a, b} &\coloneqq a\rbr{v}\,b\rbr{w} \nonumber \\
&= \rbr{a_μ b^μ}\rbr{b_ν w^ν},
\end{align}
which is a bilinear functional on the Cartesian product $V^* \times
W^*$.

Let $\cbr{e_μ}$, $\cbr{f_ν}$ be basis of $V$ and $W$,
respectively.
\[\rbr{e_μ\otimes f_ν}\rbr{a, b} = a\rbr{e_μ} b\rbr{f_ν} =
a_μ b_ν,\]
so that $e_μ\otimes f_ν$ is a basis in $V\otimes W$.

$T\in V\otimes W$:
\begin{align}
T\rbr{a, b} &= T\rbr{a_μ θ^μ, b_ν \eta^ν} \nonumber \\
&= \underbrace{a_μ b^ν}_{\rbr{e_μ\otimes f_ν}\rbr{a, b}}
\underbrace{T\rbr{θ^μ, \eta^ν}}_{\eqqcolon T^{μν}}.
\end{align}
we get
\begin{empheq}[box=\fbox]{equation}
T = T^{μν} e_μ\otimes f_ν
\end{empheq}.

Dimension of $V\otimes W$ is $mn$ if $m = \mathrm{dim}\,V$, $n =
\mathrm{dim}\,W$. for $m=n$, also called \emph{dyadic product}.

\begin{defn}
Tensor of type $\rbr{p, q}$ over $V$
\begin{empheq}[box=\fbox]{equation}
T = T^{μ_1 \ldots μ_p}{}_{ν_1 \ldots ν_q}\,
\bigotimes_{i=1}^p e_{μ_i}\otimes\bigotimes_{i=1}^pθ^{ρ_i}
\end{empheq}
is an element of $V^{\otimes p} \otimes \rbr{V^*}^{\otimes q}$.
\end{defn}

\begin{defn}
The \emph{metric tensor} or \emph{metric} $g$ is of type $\rbr{0, 2}$
denoted by
\begin{equation}
\rfun{g}{u, v} = g_{μν}u^μ u^ν
\eqqcolon u\cdot v\quad\text{or}\quad\abr{u, v}
\end{equation}
satisfying
\begin{align}
\rfun{g}{v, v} &= 0\quad\forall v\in V\quad\Rightarrow\quad v=0
\qquad\text{(Non-singular)},\\
g_{μν} &= \rfun{g}{e_μ, e_ν} = e_μ\cdot e_ν = g_{νμ}
\qquad\text{(symmetric)}.
\end{align}
\end{defn}

For $n$ fixed: $g\rbr{u, \cdot}$ is a linear functional on $V$, that
is, a covector; that is, one can associate each vector $u$ with a
covector $g\rbr{u, \cdot}$, which is the abstract formulation of
\emph{lowering index},
\begin{equation}
u_μ \equiv g_{μν}u^ν.
\end{equation}

$V = T_P$: $g\rbr{\partial_μ, \partial_ν} = g_{μν}$ (as in
section 5), but now more flexible: can have $g\rbr{e_ν, e_ν} =
\eta_{μν}$ by the choice of an orthonormal basis $\cbr{e_μ}$.

\begin{defn}
\emph{Tensor algebra over $V$}: direct sum of the space with
multiplication = tensor product.
\begin{equation}
V^p{}_q\coloneqq V^{\otimes p} \otimes \rbr{V^*}^{\otimes q},
\end{equation}
\end{defn}
so that $V^1{}_0 = V$, 
$V^0{}_1 = V^*$, $V^0{}_0 = \mathbb{R}$.

\subsection{The exterior algebra}
Important subspace of the tensor algebra: \emph{exterior algebra}
(Graßmann\footnote{Herrmann Günther Graßmann (1809 --- 1877), inventor of
linear algebra as it is taught today, Sanskrit dictionary of 1873
(\textbf{really?}).} number) $\Lambda$ over $V$, is \emph{defined} as the
direct sum over $\Lambda^p(V)$, where $\Lambda^p$ denotes the space of all
antisymmetric tensors\footnote{$T\rbr{a, b} = -T\rbr{b, a}$, $a, b\in V^*$}
of type $(p, 0)$\footnote{Because over $V$; below over $V^*$.};
$\Lambda^0 = \mathbb{R}$, $\Lambda^1 = V$.

(below: $\Lambda^p\rbr{T_p^*}$, called \emph{differential forms}, antisymmetric
tensor of type $\cbr{0, p}$.)

Need a product that does not lead outside $\Lambda$, i.e.\ that again leads to
antisymmetric tensors. Let $D\in\Lambda^p$, $T\in\Lambda^q$,
$a_i\in V^*,\,i = 1, 2, \ldots, p+q$.
\begin{defn}
The \emph{exterior} or \emph{wege product} $\wedge$
\begin{equation}
\rbr{D\wedge T}\rbr{a_1, \ldots, a_{p+q}}\coloneqq
\frac{1}{p!q!}\sum_{π} \mathrm{sgn}\,π\, \rbr{D\otimes T}
\rbr{a_{π(1)}, \ldots, a_{π\rbr{p+q}}},
\end{equation}
where $π$ denotes all permutations of $1, \ldots, p+1$,
$\mathrm{sgn}\,π = \pm 1$ for even/odd permutations, $1/p!q!$ makes the
product associative.
\end{defn}
\begin{equation}
D\wedge T = \rbr{-1}^{pq}T\wedge D.
\end{equation}

\begin{exmp}
Let $v, w\in V \equiv \Lambda^1$, then
\begin{equation}
v\wedge w = v\otimes w - w\otimes v \in \Lambda^2
\end{equation}
which obeys
\begin{equation}
v\wedge w = -w\wedge v,\qquad v\wedge v = 0.
\end{equation}
when $V = \mathbb{R}^3$, $\wedge$ corresponds to \emph{vector product}.
\end{exmp}

Orthonormal basis in $\Lambda^2$: $e_μ\wedge e_ν$ ($μ < ν$). There are
$\binom{n}{2}$ independent ones, so that $\dim \Lambda^2 =
\binom{n}{2}$.

\begin{exmp}
In $n = 4$, the basis are
\[\begin{matrix}
e_0\wedge e_1 & e_0\wedge e_2 & e_0\wedge e_3 \\
 & e_1\wedge e_2 & e_1\wedge e_3 \\
 & & e_2\wedge e_3 \\
\end{matrix}.\]
Total number is $\binom{4}{2} = 6$.
\end{exmp}

One can write
\begin{align}
v\wedge w = v^μ e_μ \wedge w^ν e_ν &= \frac{1}{2!}
\rbr{v^μ w^ν - v^ν w^μ}e_μ\wedge e_ν \nonumber \\
&= \sum_{μ < ν}\rbr{v^μ w^ν - v^ν w^μ}e_μ\wedge e_ν,
\end{align}
where $v\wedge w$ is also called \emph{bivector}.

\begin{exmp}
For $n=2$,
\[ v\wedge w = \rbr{v^1 w^2 - v^2 w^1}e_1\wedge e_2.\]
\end{exmp}

\begin{rem}
Interesting observation by William Clifford (1845 --- 1879): set $v =
v_1 + \ii v_2$ where $v, v_1, v_2 \in \mathbb{R}$,
\begin{align}
vw^* &= \rbr{v_1+\ii v_2}\rbr{w_1-\ii w_2} \nonumber \\
&= \rbr{v_1w_1+v_2w_2}+\ii\rbr{v_2w_1-v_1w_2},
\end{align}
where the real and imaginary part correspond to the inner and wedge
product, respectively. This is called \emph{Cliffold's geometric
product}. See also \cite{Lasenby2010}.
\end{rem}

Generally, for a basis of $\Lambda^p$,
\begin{align}
e_H &\coloneqq e_{h_1} \wedge \ldots \wedge e_{h_p}\qquad\text{with}\\
H &\coloneqq\cbr{h_1,\ldots h_p| h_i<h_j,\quad\forall i < j}
\end{align}
so that $\dim \Lambda^p = \binom{n}{p}$ because we have $\binom{n}{p}$
such products.
\begin{exmp}
\[\dim\Lambda = \sum_{p=0}^n\binom{n}{p} = \rbr{1+1}^n = 2^n.\]
\end{exmp}

Orthonormal basis $\cbr{e_μ}$ in $V$: $e_μ\cdot e_ν =
\rfun{g}{e_μ, e_ν} = \eta_{μν}$; $e_i\cdot e_j = δ_{ij}$ in the
Euclidean case.

Let $v = v_1\wedge \ldots \wedge v_p$, $w = w_1\wedge\ldots\wedge w_p$, with
$v_i, w_j \in V$, $i, j = 1, \ldots, p$.
\begin{defn}
\begin{equation}
v\cdot w \coloneqq \det v_i\cdot w_j \equiv \rfun{g}{v_i, w_j}.
\end{equation}
\end{defn}
\begin{exmp}
\begin{align*}
e&\coloneq e_1\wedge\ldots\wedge e_n\quad\in\Lambda^n \\
\Rightarrow\qquad
e\cdot e &= \det e_μ\cdot e_ν =
\mathrm{diag}\rbr{-1, 1, \ldots, 1} = -1.
\end{align*}
\end{exmp}

\begin{defn}
The \emph{length} of $u\in V$ is
\begin{equation}
\vbr{u}\coloneqq\sqrt{\vbr{u\cdot u}}.
\end{equation}
\end{defn}
\begin{exmp}
\[\vbr{e} = 1 = \prod_{i = 1}^n \vbr{e_i}.\]
\end{exmp}

Let $v_i \in V$, $i = 1, \ldots, n$.
\begin{defn}
\begin{equation}
v_1\wedge\ldots\wedge v_n \eqqcolon \mathcal{V}\cdot e,
\end{equation}
where $\vbr{\mathcal{V}}$ is the volume of parallelepiped,
%\footnote{Pronunced
%{/ˌpærəlɛlˈɛpᵻpɛd/}
%{\textsc{parr}-ə-lel-\textsc{ep}-i-ped}
in accordance with its
%etymology in Greek παραλληλ-επίπεδον, a body ``having parallel planes''.},
spanned by $n$ arbitrary vectors $\cbr{v_i}$, and $\mathrm{sgn}\,\mathcal{V}$
is the relative orientation of the $v_i$ with respect to the $e_μ$.
\end{defn}
\begin{exmp}
some diagrams
\end{exmp}

One can show
\begin{equation}
\vbr{\mathcal{V}} = \sqrt{\vbr{\det g_{ij}}}\qquad\text{where }
g_{ij} = v_i\cdot v_j,
\end{equation}
which is important for \emph{integration theory on manifolds} (see ealier).


The \emph{Levi-Civita symbol} $\epsilon$ is a further important quantity.
\begin{exmp}
In $n = 4$,
\[a^μ \coloneqq \frac{1}{3!} g^{μ\xi}\epsilon_{\xiνρλ}
a^{νρλ},\]
where $a^{νρλ} = a^{\sbr{νρλ}}$ is totally antisymmetric.
\textbf{(Notiz auf Deutsch?)}
\end{exmp}
\begin{defn}
The \emph{$*$-operator} or \emph{Hodge star} maps
\begin{empheq}[box=\fbox]{equation}
\Lambda^p\xleftrightarrow{*}\Lambda^{n-p}
\end{empheq}
in this way, where $\dim\Lambda^p = \binom{n}{p} = \binom{n}{n-p} =
\dim\Lambda^{n-p}$.
\end{defn}

\subsection{Exterior $p$-forms}
\label{ssec:extpf}

\emph{Exterior algebra} can be built over covector fields (more
important than over vector fields because of \emph{exterior
 derivative}, which can be defined without metric, see below)

\begin{defn}
An \emph{exterior} or \emph{differential} $p$-form maps every point
$P$ to an element of $\Lambda^p$ ($T_P^*$) (here: totally
antisymmetric tensors of type $(0, p)$). Specifically, $q$-form $w$:
\[ω = \sum_{H}ω_H\rbr{x^μ}\dd\, x^H\]
with respect to \emph{holonomic} $ω$ basis, where $H$ stands for a
sequence of indices $\cbr{h_1 < h_2 < \ldots < h_q}$, and $\dd x^H =
\dd x^{h_1} \wedge \ldots \wedge \dd x^{h_q}$.
\end{defn}
\begin{exmp}
$0$-forms are scalar functions.
\end{exmp}
\begin{exmp}
$n=3$: $0$-form: $\Phi = \rfun{f}{x, y, z}$;
$1$-form: $λ = a\dd\, x + b\dd\, y + c\dd\, z$;
$2$-form: $μ = f\dd\, y\wedge\dd z + g\dd\, z\wedge\dd x +
	h\dd\, x\wedge\dd y$;
$3$-form: $ν = a\dd\, x \wedge\dd y\wedge\dd z$;
higher forms are identically zero.

Dimension of $\Lambda$ is $1+3+3+1 = 2^3 = 8$.
\label{exmp:vec-anal-0}
\end{exmp}

\begin{defn}
\emph{Exterior derivative} $\dd $ maps $q$-forms linearly into $(q+1)$-forms
and is uniquely determined by the following properties:
\begin{enumerate}
\item \label{enum:ext-der-1}
$\dd \wedge\rbr{ω+\eta} = \dd \wedgeω+\dd \wedge\eta$;
\item \label{enum:ext-der-2}
$\dd \wedge\rbr{ω\wedge\eta} = \rbr{\dd \wedgeω}\wedge\eta
+\rbr{-1}^{\opdeg ω}ω\wedge\rbr{\dd \wedge\eta}$;
\item \label{enum:ext-der-3}
$\dd \wedge\rbr{\dd \wedgeω} = 0$;
\item \label{enum:ext-der-4}
$\dd \wedge f = \dd f$ for a scalar function $f$.
\end{enumerate}
The $\wedge$ is often skipped after $\dd $.
\label{defn:ext-der}
\end{defn}
\begin{exmp}
For the above form $ω = \sum_H ω_H\rbr{x^μ}\dd\, x^H$,
\begin{empheq}[box=\fbox]{equation}
\dd ω = \sum_H \frpa{ω_H}{x^ν}\dd\, x^ν\wedge\dd x^H
= \sum_H \dd ω_H\wedge\dd x^H.
\end{empheq}
One can confirm that this fulfils all conditions.
\end{exmp}

\begin{exmp}
An example for $\wedge$.
$n=3$, $ω = xy^2\dd\, x\wedge\dd y - z\dd\, y\wedge\dd z$ $2$-form,
$\zeta = x\dd\, x+y\dd\, y+z\dd\, z$ $1$-form. $ω\wedge\zeta$ is a
$3$-form and must therefore be proportional to $\dd x\wedge\dd y\wedge\dd z$:
\begin{align*}
ω\wedge\zeta =
&x^2y^2\,\underbrace{\dd x\wedge\dd y\wedge\dd x}_{0}
+xy^3\,\underbrace{\dd x\wedge\dd y\wedge\dd y}_{0} \\
+&xy^2z\dd\, x\wedge\dd y\wedge\dd z
-xz\,\underbrace{\dd y\wedge\dd z\wedge\dd x}_%
{\dd x\wedge\dd y\wedge\dd z} \\
-&yz\,\underbrace{\dd y\wedge\dd z\wedge\dd y}_{0}
-z^2\,\underbrace{\dd y\wedge\dd z\wedge\dd z}_{0} \\
=&\rbr{xy^2z-xz}\dd\, x\wedge\dd y\wedge\dd z.
\end{align*}
\end{exmp}

\begin{exmp}
An example for $\dd $.
$n=2$, $ω = P\dd\, x+Q\dd\, y$ $1$-form.
\begin{align*}
\Rightarrow\quad
\dd ω &= \dd P\wedge\dd x+\dd Q\wedge\dd y \\
&= \rbr{\frpa{P}{x}\dd\, x+\frpa{P}{y}\dd\, y}\wedge \dd x
+\rbr{\frpa{Q}{x}\dd\, x+\frpa{Q}{y}\dd\, y}\wedge \dd y \\
&= \frpa{P}{y}\dd\, y\wedge\dd x +\frpa{Q}{x}\dd\, x\wedge\dd y
=\rbr{\frpa{Q}{x}-\frpa{P}{y}}\dd\, x\wedge\dd y
\end{align*}
is a $2$-form, where the first line used \ref{enum:ext-der-2} and
\ref{enum:ext-der-3} ($\dd \dd x = 0$) in \cref{defn:ext-der}, and
the second line used \ref{enum:ext-der-4}. The result
reminds one of $\nabla\times\ora{a}$.
\label{ex:ext-der-curl}
\end{exmp}

\begin{exmp}
Recall \cref{exmp:vec-anal-0}. Connection with vector analysis in $n=3$:
\begin{enumerate}
\item $\Phi = \rfun{f}{x, y, z}$ ($0$-form) \\
\begin{align*}
\Rightarrow\quad &= \frpa{f}{x}\dd\, x+\frpa{f}{y}\dd\, y+\frpa{f}{z}\dd\, z
\\
&\equiv\rbr{\grad f}\cdot\dd \ora{x}
\equiv\rbr{\nabla f}\cdot\dd \ora{x}.
\end{align*}

\item
$λ = a\dd\, x+b\dd\, y+c\dd\, z\equiv\ora{a}\cdot\dd \ora{x}$
($1$-form).
\begin{align*}
\Rightarrow\quad\dd λ&=
\rbr{\frpa{c}{y}-\frpa{b}{z}}\dd\, y\wedge\dd z
+\rbr{\frpa{a}{z}-\frpa{c}{x}}\dd\, z\wedge\dd x
+\rbr{\frpa{b}{x}-\frpa{a}{y}}\dd\, x\wedge\dd y \\
&\equiv\rbr{\curl \ora{a}}\cdot\dd \ora{O}
\equiv\rbr{\nabla\times\ora{a}}\cdot\dd \ora{O},\qquad\text{where}
\end{align*}
\[
\ora{a} = \begin{pmatrix} a \\ b \\ c \end{pmatrix},\qquad
\dd \ora{O} = \begin{pmatrix}
\dd y\wedge\dd z \\
\dd z\wedge\dd x \\
\dd x\wedge\dd y \end{pmatrix}
\quad\text{vector-valued $2$-form}.
\]

\item
\begin{align*}
μ &= f\dd\, y\wedge\dd z+g\dd\, z\wedge\dd x+h\dd\, x\wedge\dd y
\qquad\text{($2$-form)} \\
&\equiv \ora{v}\cdot\dd \ora{O},\qquad
\ora{v} = \begin{pmatrix} f \\ g \\ h \end{pmatrix}.
\end{align*}
$\int \ora{v}\cdot\dd \ora{O}$: flux of vector field through directed surface
\begin{align*}
\Rightarrow\quad
\dd μ &=
\rbr{\frpa{f}{x}+\frpa{g}{y}+\frpa{h}{z}}\dd\, x\wedge\dd y\wedge \dd z \\
&\equiv \rbr{\opdiv \ora{v}}\dd\,^3 x
\equiv\rbr{\nabla\cdot\ora{v}}\dd\,^3 x.
\end{align*}
\end{enumerate}
\label{exmp:vec-anal-1}
\end{exmp}

\begin{defn}
$ω$ \emph{exact} if $ω = \dd α$,
$ω$ \emph{closed} if $\dd ω = 0$.
\end{defn}
Since $\dd \rbr{\dd α}=0$: $ω$ exact $\Rightarrow$ $ω$ closed. 
Inverse?
\begin{defn}
In a \emph{star-shaped} region, from a \emph{given} point, every other
point can be reached by a straight line.
\label{defn:star-shaped}
\end{defn}
\begin{rem}
In a \emph{convex} region, however, the property in
\cref{defn:star-shaped} holds for \emph{any} two points.
\end{rem}
\begin{lem}[Poincaré Lemma]
If $ω$ is closed, then $ω$ is exact in a star-shaped region.
\label{lem:poincare}
\end{lem}
\begin{rem}
Gilt allgemeinhin als ``simply connected''.
\end{rem}
Study of closed forms which are not exact leads to the \emph{de Rham
cohomology} and \emph{Betti numbers}.

\begin{exmp}
Necessary for holonomic cobasis: $\dd θ^μ = 0$.
For example, $\dd x^μ$.
\end{exmp}
\begin{exmp}
Recall \cref{exmp:vec-anal-1}. $\dd \rbr{\dd ω} = 0$ yields for
\begin{enumerate}
\item
$ω = λ = a\dd\, x + b\dd\, y + c\dd\, z$:
\begin{align*}
\dd \rbr{\curl \ora{a}\cdot\dd \ora{O}} &= 0 \\
\Rightarrow\qquad\opdiv \curl \ora{a} &= 0.
\end{align*}

\item
$ω = \Phi = \rfun{f}{x, y, z}$:
\begin{align*}
\dd \rbr{\grad f\cdot\dd \ora{x}} &= 0 \\
\Rightarrow\qquad\curl \grad f &= 0.
\end{align*}
\end{enumerate}

\nameref{lem:poincare}: locally, one has
\begin{align*}
\opdiv \ora{v}&=0\quad\Rightarrow\quad
&\ora{v}&=\curl \ora{a}, \\
\curl \ora{a}&=0\quad\Rightarrow\quad
&\ora{a}&=\grad f.
\end{align*}
\end{exmp}

\begin{defn}
Tensor-valued $q$-forms of type $\rbr{0, b}$ maps every point $P$ to an
element from
\begin{equation}
\underbrace{V^a{}_b}_\text{tensor-valued}\otimes
\underbrace{\Lambda^q\rbr{T_P^*}}_\text{$q$-form}
\end{equation} 
\end{defn}

\subsection{Connections and Riemann geometry}

We had in \cref{ssec:extpf}: exterior derivative

Concept of differentiation that can be introduced in a basis-independent way
without having \emph{additional distinguished structures}, see
\cref{fig:add-dist-struct}.
\begin{figure}
\centering
\begin{tabular}{rcl}
\emph{metric $g$} &\multicolumn{2}{l}{in Riemann geometry} \\
\multicolumn{1}{c}{$\downarrow$} && \\
\emph{connection} & $\leftrightarrow$ & covariant derivative
\end{tabular}
\caption{An example of additional distinguished structure
\label{fig:add-dist-struct}}
\end{figure}

Recall: $\DD_μ v^ν = \partial_μ v^ν + Γ^{ν}{}_{μλ} v^λ$

A connection enables one to compare vectors in the tangent spaces of
infinitesimally neighboured points (`parallel transport')

One can define a connection independent of any metric (\emph{if} defined from
a metric: Levi-Civita connection, or `Christoffel symbol')

\paragraph{Construction of a connection} \mbox{} \\

$\left\{e_μ\right\}$: $n$-bein field; $e_μ(Q)$: at $Q$, etc.\ \\
$\left\{θ^μ\right\}$: dual frame

\begin{figure}
\centering
\includegraphics{graphics/10/Diag3}
\caption{Construction of a connection\label{fig:construct-connect}}
\end{figure}

$e'_μ(P)$: $n$-bein transported from $Q$ to $P$.

\begin{align}
T_p \ni \DD_{\epsilon u} e_μ &\coloneqq e'_μ(P)-e_μ(P) \nonumber \\
&=: \rfun{ω^ν{}_μ}{\epsilon u}\,e_ν(P)
\end{align}
where $\rfun{ω^ν{}_μ}{\epsilon u}$ is the \emph{connection}
(Zusammenhang, Übertragung); here also \emph{spin connection}, if with respect
to an orthonormal frame field.

We only consider \emph{linear connections}, where $\DD_{\epsilon u}e_μ$ and
$\rfun{ω^ν{}_μ}{\epsilon u}$ depend linearly on $\epsilon u$, so that $\epsilon$
cancels out.
\begin{empheq}[box=\fbox]{equation}
\DD_u e_μ = \rfun{ω^ν{}_μ}{\epsilon u}\, e_ν,
\label{eq:cDuemu}
\end{empheq}
\emph{absolute} or \emph{covariant} differential.

We then have a map
\begin{equation}
u \mapsto \rfun{ω^ν{}_μ}{u},
\end{equation}
where $ω^ν{}_μ$ defines a matrix of one-forms (for each $(μ, ν)$ one one-form)

Decomposition into dual basis
\begin{empheq}[box=\fbox]{equation}
ω^ν{}_μ = L^ν{}_{μλ}\,θ^λ,
\end{empheq}
where $L^ν{}_{μλ}$ are \emph{connection coefficients}.

Without $u$, \cref{eq:cDuemu} can be written as
\begin{empheq}[box=\fbox]{align}
\DD e_μ &= ω^ν{}_μ\otimes e_ν \nonumber \\
&= L^ν{}_{μλ}\,θ^λ \otimes e_ν
\end{empheq}

Covariant derivative of an arbitrary vector field $v = v^μ e_μ$: demand the
validity of the product rule, additivity, and $\DD f \coloneqq \dd f$ for 
scalar
fields; then
\begin{align}
\DD v = \DD\left(v^μ e_μ\right) &=
\underbrace{\dd v^μ}_\text{one-form for each $μ$} \otimes e_μ
	+ v^μ \DD\, e_μ \nonumber \\
&= \dd v^μ \otimes e_μ + v^μ L^ν{}_{μλ}\,
	θ^λ \otimes e_ν \nonumber \\
&= \left(\dd v^μ + ω^ν{}_μv^ν\right)\otimes e_μ,
\end{align}
where
\begin{align}
\dd v^μ &= v^μ{}_{,λ}\dd\, x^λ\qquad
&\text{(coordinate basis)} \\
\dd v^μ &= \rfun{e_λ}{v^μ}\,θ^λ\qquad
&\text{(general basis)}.
\end{align}
Note that in general, $\dd f = \rfun{e_λ}{f}\,θ^λ$. Therefore,
\begin{align}
\DD v &= \rfun{e_λ}{v^μ}\,θ^λ\otimes e_μ +
v^ν L^μ{}_{νλ}\,θ^λ\otimes e_μ \nonumber \\
&= \rbr{\rfun{e_λ}{v^μ}+v^νL^μ{}_{νλ}}\,θ^μ\otimes e_μ \nonumber \\
&\equiv\DD_λ v^λ\,θ^λ\otimes e_μ
\equiv v^μ{}_{;λ}\,θ^λ\otimes e_μ.
\end{align}
In a coordinate basis, this reads
\begin{equation}
\DD v = \rbr{\frpa{v^μ}{x^λ} + L^μ{}_{νλ}{v^ν}}
\dd\, x^λ\otimes\partial_μ.
\end{equation}
$L^ν{}_{μλ}$ where $μ$ and $λ$ are different types
of indices, defined independently of the metric; do not have to be symmetric
in $μ$ and $λ$, even if $e_μ = \partial_μ$, unless torsion
vanishes, see below.

$\DD θ^μ$?
\begin{equation}
θ^μ\rbr{e_ν} = δ^μ_ν\quad
\xrightarrow{\substack{\text{demand} \\ \text{Leibniz} \\ \text{rule}}}\quad
\DDθ^μ\rbr{e_ν}+θ^μ\rbr{\DD e_ν} = 0,
\end{equation}
where $\DD e_ν = ω^ρ{}_ν\otimes e_ρ$.
\begin{align*}
\Rightarrow\quad\DD θ^μ\rbr{e_ν} &=
-ω^ρ{}_ν\otimes
\underbrace{θ^μ\rbr{e_ρ}}_{δ^μ_ρ} =
-ω^μ{}_ν \\
\Rightarrow\quad\DDθ^μ &=
-ω^μ{}_ν\otimesθ^ν \\
&= -L^μ{}_{νλ}\,θ^λ\otimesθ^ν
\end{align*}
Let $a$ be a $1$-form
\begin{align*}
\DD a &= \DD\rbr{a_μ θ^μ} =
	\underbrace{\dd a_μ}_{e_ν\rbr{a_μ}θ^ν}\otimesθ^μ +
	a_ν\,\underbrace{\DDθ^μ}_%
	{-L^μ{}_{νλ}\,θ^λ\otimesθ^ν} \\
&= e_ν\rbr{a_μ}\,θ^ν\otimesθ^μ -
	a_λL^λ{}_{μν}\,θ^ν\otimesθ^μ \\
&= \rbr{e_ν\rbr{a_μ}-L^λ{}_{μν}a_λ}\,
	θ^ν\otimesθ^μ \\
&\equiv \DD_ν a_μ\,θ^ν\otimesθ^μ \\
&\equiv a_{μ;ν}\,θ^ν\otimesθ^μ.
\end{align*}


\paragraph{Transformation of $ω^μ{}_ν$ under a change of
basis}\mbox{} \\
\begin{equation}
e_{\overline{α}} = p^ν{}_\ol{α}\,e_ν \quad
%\xhookrightarrow{\text{inverse}} \quad
\xrightarrow{\text{inverse}} \quad
e_ν = p^\ol{α}{}_ν\,e_{\overline{α}};
\end{equation}
then
\begin{align}
e_ν &= p^\ol{α}{}_ν\,e_{\overline{α}}
=\underbrace{\overbrace{p^\ol{α}{}_ν}^{p}
\overbrace{p^μ{}_\ol{α}}^{p^{-1}}}_{\eeq
 δ^μ_ν}\,e_μ \\
e_\ol{α} &= p^ν{}_\ol{α}\,e_ν
=\underbrace{p^ν{}_\ol{α}p^\ol{β}{}_ν}_{\eeq
 δ^\ol{β}_\ol{α}}\,e_\ol{β}
\end{align}
so that
\begin{align*}
\DD e_\ol{α} &= \rbr{\dd p^ν{}_\ol{α}} \otimes e_ν
+p^ν{}_\ol{α}\,\underbrace{\DD e_ν}_{ω^μ{}_ν \otimes e_μ} \\
&= \rbr{\dd p^ν{}_\ol{α}+p^μ{}_\ol{α}
ω^ν{}_μ}\otimes e_ν \\
&\equiv ω^\ol{β}{}_\ol{α}\otimes
e_\ol{β} \\
&= ω^\ol{β}{}_\ol{α}p^ν{}_\ol{β}
\otimes e_ν
\end{align*}
Comparison:
\begin{align}
ω^\ol{β}{}_\ol{α} p^ν{}_\ol{β} &=
p^μ{}_\ol{α} ω^ν{}_μ +\dd p^ν{}_\ol{α} \nonumber \\
\xRightarrow{\cdot p^\ol{γ}{}_ν}\quad ω^\ol{β}{}_\ol{α}
\underbrace{p^ν{}_\ol{β} p^\ol{γ}{}_ν}_{δ^\ol{γ}_\ol{β}}
&= p^μ{}_\ol{α} p^\ol{γ}{}_νω^ν{}_μ
+ p^\ol{γ}{}_ν\dd\, p^ν{}_\ol{α} \nonumber \\
\Rightarrow\quad
ω^\ol{γ}{}_\ol{α} &= p^\ol{γ}{}_νp^μ{}_\ol{α} ω^ν{}_μ
+\underbrace{p^\ol{γ}{}_ν\dd\, p^ν{}_\ol{α}}_%
{-p^ν{}_\ol{α}\dd\, p^\ol{γ}{}_ν}
\label{eq:star-connection}
\end{align}
So finally
\begin{equation}
\Rightarrow\quad
ω^\ol{β}{}_\ol{α} =
p^\ol{β}{}_ν p^μ{}_\ol{α} ω^ν{}_μ -p^ν{}_\ol{α}\dd\, p^\ol{β}{}_ν
\end{equation}
Call $p^\ol{β}{}_ν$ the matrix $p$. In matrix notation,
\begin{equation}
ω \mapsto p ω p^{-1} - p^{-1}\dd\, p
\end{equation}
This is the typical transformation law for a connection.

Decompose again $ω^\ol{β}{}_\ol{α} =
L^\ol{β}{}_{\ol{α}\ol{γ}}\,θ^\ol{γ}$,
then one gets from \cref{eq:star-connection} that
\begin{equation}
L^\ol{α}{}_{\ol{β}\ol{γ}} =
L^μ{}_{νλ} p^\ol{α}{}_μ p^ν{}_\ol{β} p^λ{}_\ol{γ}
+p^\ol{α}{}_ν\, e_\ol{γ}\rbr{p^ν{}_\ol{β}}
\end{equation}
which is not the transformation law of a tensor; resembles the
transformation law for $Γ^μ{}_{νλ}$. What is the exact
relation between these two? $\dd f(e) = e(f)$

Choose a holonomic basis $e_μ = \partial_\mu$, $e_\ol{\alpha} =
\partial_\ol{\alpha}$, $p^\nu{}_\ol{\alpha} = \frpa{x^\nu}{x^\ol{\alpha}}$;
then,
\begin{equation}
L^\ol{\alpha}{}_{\ol{\beta}\ol{\gamma}} = 
L^μ{}_{\nuλ} \frpa{x^\ol{\alpha}}{x^\mu} \frpa{x^\nu}{x^\ol{\beta}}
\frpa{x^λ}{x^\ol{\gamma}} + \frpa{x^\ol{\alpha}}{x^\nu}
\frpa{^2 x^\nu}{x^\ol{\beta}\,\partial x^\ol{\gamma}}
\label{eq:trans-L}
\end{equation}
does not need to be symmetric in $\ol{\beta}$ and $\ol{\gamma}$ (as the
Christoffel symbol would be)

\begin{equation}
L^μ{}_{\nuλ} - L^μ{}_{λ\nu} \equiv 2L^μ{}_{\sbr{\nuλ}}
\eqqcolon T^μ{}_{\nuλ}
\end{equation}
transforms as a tensor, because
$\frpa{^2 x^\nu}{x^\ol{\beta}\,\partial x^\ol{\gamma}}$-term in
\cref{eq:trans-L} does not contribute.

$T^μ{}_{\nuλ}$: components of the \emph{torsion tensor}.

In Riemannian geometry, torsion vanishes identically; then,
\begin{equation}
L^μ{}_{\rbr{\nuλ}} = L^μ{}_{\nuλ} = \Gamma^μ{}_{\nuλ}
\end{equation}
(Christoffel symbol or Levi-Civita connection)

Only in Riemannian geometry, and only in a holonomic basis, is
$L^μ{}_{\nuλ} = \Gamma^μ{}_{\nuλ}$ and
$L^μ{}_{\nuλ} = L^μ{}_{λ\nu}$.

Because $T^μ{}_{\nuλ}$ is antisymmetric in $\nu$ and $λ$, one
can intepret it as the coefficients of a vector-valued $2$-form,
\begin{align}
\frac{1}{2} T^μ{}_{\nuλ}\dd\, x^λ\wedge\dd x^\nu
&\equiv \underbrace{L^μ{}_{\nuλ}\dd\, x^λ}_{ω^μ{}_\nu}
\wedge\dd x^\nu \nonumber \\
&= ω^μ{}_\nu\wedge \underbrace{\dd x^\nu}_{\text{holonomic co-basis,
obeys $\dd \wedge\dd x^\nu = 0$}}
\end{align}
\emph{general basis}: \emph{Define} the torsion $2$-form
\begin{align}
\Theta^μ &\coloneqq \overbrace{\dd \theta^\mu}^{\text{not present in
holonomic basis}} + ω^μ{}_\nu \wedge \theta^\nu \\
&\eqqcolon\frac{1}{2} T^μ{}_{\nuλ}\,\theta^λ\wedge\theta^\nu
\end{align}
Justification of this definition comes from the behaviour under basis change
$\theta^μ\mapsto\theta^\ol{\alpha}$,
\begin{equation}
\Theta^\ol{\alpha} = p^\ol{\alpha}{}_μ\,\Theta^\mu
\end{equation}
which is left as an exercise. Now it is clear that $T^μ{}_{\nuλ}$
are really the components of a tensor (which is a \emph{holonomic basis}
coinciede with $2L^μ{}_{\sbr{\nuλ}}$)

One defining equation of \emph{Riemannian geometry} is the \emph{vanishing of
torsion},
\begin{empheq}[box=\fbox]{equation}
\Theta^μ \equiv 0
\end{empheq}
or
\begin{empheq}[box=\fbox]{equation}
\dd \theta^μ + ω^μ{}_\nu\wedge\theta^\nu = 0
\end{empheq}
\emph{First Cartan structure equation}. gauge field corresponds to
$ω^μ{}_\nu$; gauge potential corresponds to Lorentz transformation
\textbf{What does it mean here}

($\DD e_μ = \underbrace{ω^\nu{}_\nu}_{L^\nu{}_{\muλ}\,
\theta^λ}\otimes e_\nu$)

Define now curvature $2$-forms as follows
\begin{empheq}[box=\fbox]{align}
\Omega^μ{}_\nu &\coloneqq \dd ω^μ{}_\nu + ω^μ{}_{λ}
	\wedgeω^λ{}_\nu \\
&\eqqcolon \frac{1}{2} R^μ{}_{\nuρ\sigma}\,
\theta^ρ\wedge\theta^\sigma
\end{empheq}
\emph{Second Cartan structure equation}. $\Omega^μ{}_\nu$ are components of
a tensorial $2$-form.

One can show that under a basis change,
\begin{equation}
\Omega^\ol{\alpha}{}_\ol{\beta} = p^\ol{\alpha}{}_μ p^\nu{}_\ol{\beta}
\Omega^μ{}_\nu,
\end{equation}
that is, the $R^μ{}_{\nuρ\sigma}$ are the components of a tensor; in a
holonomic basis, these are the components of the Riemann curvature tensor
(exercise) (use $L^μ{}_{\nuλ} = \Gamma^μ{}_{\nuλ}$)

Metricity: Validity of product rule if $\DD$ is applied to a scalar product,
\begin{equation}
\DD \underbrace{\rfun{g}{v, w}}_{\text{scalar}} = \dd \rfun{g}{v, w} \eeq
\rfun{g}{\DD v, w} + \rfun{g}{v, \DD w}.
\end{equation}
Applying this to $v = e_\mu$, $w = e_\nu$:
\begin{align}
\dd \rfun{g}{e_\mu, e_\nu} &= \dd g_{μ\nu} \eeq \rfun{g}{\underbrace{\DD
	e_\mu}_{ω^λ{}_μ e_λ}, e_\nu} + \rfun{g}{e_\mu,
	\underbrace{\DD e_\nu}_{ω^λ{}_\nu e_λ}} \\
&= \underbrace{ω^λ{}_μ g_{λ\nu}}_{\eqqcolon
	ω_{μ\nu}} + \underbrace{ω^λ{}_\nu g_{\muλ}}_{
	\eqqcolon ω_{\nu\mu}},
\end{align}
where $ω_{\cdot\cdot}$ are defined here, since $ω^λ{}_\mu$ is
not a tensor. Now the \emph{metricity} is defined as
\begin{empheq}[box=\fbox]{equation}
\dd g_{μ\nu} = ω_{μ\nu} + ω_{\nu\mu},
\end{empheq}
where $\dd g_{μ\nu} \equiv \rfun{e_λ}{g_{μ\nu}}\,\theta^λ$.

Define also $L_{ρ\nuλ}\coloneqq g_{ρ\mu} L^μ{}_{\nuλ}$
(cf. Christoffel symbols of the first kind)

Let us evaluate the metricity condition for a \emph{holonomic} basis;
\begin{align}
\dd g_{\mu\nu} &=
\omega^\lambda{}_\mu g_{\lambda\nu} + \omega^\lambda{}_\nu g_{\mu\lambda} \\
&= \Gamma^\lambda{}_{\mu\rho}\dd\, x^\rho g_{\lambda\nu}
+ \Gamma^\lambda{}_{\nu\rho}\dd\, x^\rho g_{\mu\lambda} \\
&= \rbr{\Gamma_{\nu\mu\rho}+\Gamma_{\mu\nu\rho}}\dd\, x^\rho
\equiv g_{\mu\nu,\rho}\dd\, x^\rho
\end{align}
\begin{equation}
\hookrightarrow
g_{\mu\nu,\rho} = \Gamma_{\nu\mu\rho} + \Gamma_{\mu\nu\rho}
\end{equation}
\begin{equation}
\hookrightarrow \underbrace{g_{\mu\nu,\rho}
-\Gamma_{\mu\nu\rho}-\Gamma_{\nu\mu\rho}}_{g_{\mu\nu;\rho}} = 0
\end{equation}
In an anholonomic basis, one has $\rfun{e_\rho}{g_{\mu\nu}}$ instead of
$g_{\mu\nu,\rho}$.

`Metricity' in a holonomic basis gives back the `old' identity
$g_{\mu\nu;\rho} = 0$, which means scalar products remain constant under
parallel transport,
\begin{equation}
\frDe{\rbr{g_{\mu\nu}u^\mu u^\nu}}{s} = 0,
\end{equation}
because $\DD u^\mu/\DD s = 0$ under parallel transport.

One can show (here withour proof) that metricity and vanishing of torsion lead
to a \emph{uniquely determined connection} or \emph{Maurer-Cartan equation}:
\begin{equation}
\dd \theta^\mu \eqqcolon -\frac{1}{2} C^\mu{}_{\nu\lambda}\,
\theta^\nu\wedge\theta^\lambda,
\end{equation}
where $\dd\theta^\mu$ are $2$-forms, $C^\mu{}_{\mu\nu}$ are anholonomy
coefficients, c.f.\ $\sbr{e_\mu, e_\nu} = C^\lambda{}_{\mu\nu}\, e_\lambda$,
$C^\lambda{}_{\mu\nu} = - C^\lambda{}_{\nu\mu}$.

Cartan $\Lambda$ (\textbf{?}), metricity lead to the unique expression
(exercise)
\begin{equation}
\begin{split}
2 L_{\mu\nu\lambda} =
&C_{\mu\lambda\nu} + C_{\nu\mu\lambda} - C_{\lambda\nu\mu} \\
-\rfun{e_\mu}{g_{\nu\lambda}}+\rfun{e_\nu}{g_{\mu\lambda}}
+\rfun{e_\lambda}{g_{\mu\nu}},
\end{split}
\end{equation}
where for holonomic basis the first line vanishes, and the second line become
partial derivatives ($L_{\mu\nu\lambda} = \Gamma_{\mu\nu\lambda}$).

Of central importance are \emph{orthonormal bases}, where
$g_{\mu\nu} = \eta_{\mu\nu}$, which is not possible to achieve for coordinate
basis is  curvature is present. Only locally for a freely falling observer,
such an observer can be directly described by such an anholonomic basis.

\begin{rem}
Generalised theories of gravitation make use of non-metricity $Q_{\alpha\beta}
\coloneq -\DD g_{\alpha\beta}$.
\end{rem}

$g_{\mu\nu}=\eta_{\mu\nu}$ leads to $\dd g_{\mu\nu} = 0$, which in turn
gives $\omega_{\mu\nu} + \omega_{\nu\mu} = 0$. The last equation means the
matrix of connection forms is antisymmetric; $\Omega_{\mu\nu} = -
\Omega_{\nu\mu}$ always (follows from metricity), and only $n\rbr{n-1}/2$
forms need to be calculated; for $n=4$, this is $6$. In turn,
$n^2\rbr{n-1}/2$ coefficients $L_{\lambda\mu\nu}$; for $n=4$, it is $24$
instead of $4\times \rbr{4\times 5}/2 = 40$ Christoffel symbols.

This is very important for practical calculations! Will be our main
application.

\begin{rem}
So far enough for what follows; also important: \emph{integration on
manifolds}. Central: Stokes Theorem.

\begin{figure}
\centering
\includegraphics{graphics/10/stokes}
\caption{Stokes}
\label{fig:stokes}
\end{figure}

$\sigma\in\Lambda^{p-1}$, so that
$\dd \sigma \in \Lambda^p$,
\begin{empheq}[box=\fbox]{equation}
\int_{\partial G}\sigma = \int_{G}\dd\sigma.
\end{empheq}
$\dim\partial G = p-1$, $\dim G = p$.
Ordinary Gauß and Stokes theorems follow as special cases.
\end{rem}

\begin{rem}
Einstein-Hilbert action can be written in the form
\begin{equation}
S_\text{EH} = -\frac{1}{32\pp\nG}\int_\mathcal{M}\Omega^\mu{}_\nu\wedge
*\rbr{\theta^\nu\wedge\theta_\mu},
\end{equation}
where the integrand is a $4$-form, and the index in $\theta_\mu$ is lowered by
$g_{\mu\nu}$.

In an orthonormal frame, this can be written as
\begin{equation}
-\frac{1}{32\pp\nG}\int_\mathcal{M}\Omega^{\mu\nu}\wedge
\theta^\rho\wedge\theta^\xi\,\epsilon_{\mu\nu\rho\xi}
\end{equation}
add $\int\Ld_\text{m}$. Can also add $\Lambda$-term:
\begin{equation}
\frac{\Lambda}{4!}\int_\mathcal{M}
\theta^\mu\wedge\theta^\nu\wedge\theta^\rho\wedge\theta^\xi
\,\epsilon_{\mu\nu\rho\xi}.
\end{equation}
Variation of $\theta^\mu$ leads to the \emph{Einstein field equation}
\begin{empheq}[box=\fbox]{equation}
\Omega^{\mu\nu}\wedge\theta^\rho\,\epsilon_{\mu\nu\lambda\rho}
= -16\pp\nG\tau_\lambda,
\end{empheq}
where $\tau_\lambda$ is a vector-valued $3$-form, analogy to electromagnetic
current $j$\footnote{$\dd*F = j$, $\dd F = 0$.}; $*\tau_\lambda \eqqcolon
T_{\lambda\rho}\theta^\rho$, integrate $\tau_\lambda$ over $3$-space
volume yields the energy-momentum of matter in this volume.
\end{rem}

\begin{rem}
Recall $\Theta^\mu = \dd\theta^\mu+\omega^\mu{}_\nu\wedge\theta^\nu
= \frac{1}{2} T^\mu{}_{\nu\lambda}\,\theta^\lambda\wedge\theta^\nu$; variation
with respect to the connection $\omega$ can be performed independently and
leads to
\begin{empheq}[box=\fbox]{equation}
\Theta^\mu\wedge\theta^\nu\,\epsilon_{\rho\sigma\mu\nu} =
-8\pp\nG S_{\rho\sigma},
\end{empheq}
where $S_{\mu\nu}$ are (like $\tau_\mu$) $3$-forms, the `spin currents' of
matter ($\sfun{\Ld_\text{m}}{\omega+\lambda}-\sfun{\Ld_\text{m}}{\omega}
= -\frac{1}{2} \chi^{\mu\nu} \wedge S_{\mu\nu} + \rfun{\Omicron}{\chi^2}$).

\emph{Spin is the source of torsion}.

$S_{\mu\nu}$ and $\Theta^\mu$ vanish if $\Ld$ does not depend on
$\omega^\mu{}_\nu$, which happens for example for scalar fields and Yang-Mills
fields. $\omega^\mu{}_\nu$ are important for fermions!

\emph{Einstein-Cartan theory}: torsion does not propagate here; completely
fixed by $S_{\mu\nu}$.
\end{rem}

References: \cite{Hehl1985,Gockeler1987,GronwaldHehl2009} and lecture notes
via our website.